%%%%%%%%%%%%%%%%%%%%%%%%%%%%%%%%%%%%%%%%%%%%%%%%%%%%%%%%%%%%%%%%%%%%%%%
%% Тезисы доклада на XIV «Владикавказская молодежная математическая школа»
%%  Россия, РСО-А, с. Цей, 16-21 июля 2018 г.
%%%%%%%%%%%%%%%%%%%%%%%%%%%%%%%%%%%%%%%%%%%%%%%%%%%%%%%%%%%%%%%%%%%%%%%%

% Загрузка файла style.tex со служебными командами
\input style.tex

%%%%%%%%%%%%%%%%%%%%%%%%%%%%%%%%%%%%%%%%%%%%%%
% Пожалуйста, не меняйте порядок команд!
%%%%%%%%%%%%%%%%%%%%%%%%%%%%%%%%%%%%%%%%%%%%%%
%
% Название доклада

УДК 511.95 + 514.112


%\Title{ПОД}
\Title{ПОДМНОЖЕСТВА ТОЧЕК НА ПЛОСКОСТИ С ЦЕЛОЧИСЛЕННЫМИ РАССТОЯНИЯМИ} % обязательное поле!



%%{{ А вот этот фрагмент обеспечит использование автоматической нумерации формул.
%%   Обычно её запрещают из-за того, что при авоматической нумерации номера формул у всех авторов будут "общие":
%%   если тезисы n-го по порядку автора имели последнюю формулу с номером m,
%%   то первая формула (n+1)-го автора будет иметь первую формулу с номером m+1,
%%   чего быть не должно.
%%   Блок строчек ниже сбрасывает нумерацию формул, рисунков, таблиц и сносок.
%%   Настоятельно рекомендую добавить эту последовательность команд в определение команды \Title в файле style.tex
	\setcounter{equation}{0}
	\setcounter{figure}{0}
	\setcounter{table}{0}
	\setcounter{footnote}{0}
	\renewcommand{\theenumi}{\arabic{enumi}}
	\renewcommand{\labelenumi}{\theenumi)}
	\selectlanguage{russian}
%%}}


%
% Авторы

\Author{Авдеев Н.~Н.$^{1}$, Семенов Е.~М.$^{2}$}
% обязательное поле!
\\
\affiliation{$^{1}$Воронежский госуниверситет, Воронеж, Россия}
\email{nickkolok@mail.ru}
\\
\affiliation{$^{2}$Воронежский госуниверситет, Воронеж, Россия}
\email{nadezhka\_ssm@geophys.vsu.ru}

%\inputencoding{utf8}


Oбозначим через $\mathfrak{M}$ множество таких подмножеств
$
	M = \{M_1, M_2, ... \}\subset \mathbb{R}^2
$,
что
$
	|M_i - M_j|\in\mathbb{N}
$
для всех $i$, $j$, где $\mathbb{N}$~--- множество натуральных чисел
и $|M_i - M_j|$~--- расстояние между точками $M_i$ и $M_j$.
П. Эрдёш доказал~\cite{erdos1945integral},
что справедлива следующая

\begin{theorem}
	\label{thmErdos}
	Всякое бесконечное подмножество $M\in\mathfrak{M}$
	содержится в некоторой прямой $L\in\mathbb{R}^2$.
\end{theorem}

Теорему~\ref{thmErdos} можно доказать, основываясь на двух элементарных леммах.
Интересно, что работа~\cite{erdos1945integral} была опубликована через несколько
месяцев после совместной работы Эннинга и Эрдёша~\cite{anning1945integral},
в которой было дано другое, более сложное доказательство.

\begin{lemma}
	\label{thmQuadCurves}
	Пусть $A$ и $B$~--- две кривые второго порядка на плоскости и $A$ не является парой прямых.
	Тогда $A\cap B$ содержит не более 4 точек.
\end{lemma}

\begin{proof}
	Без ограничения общности кривая $A$ определяется уравнением
	\begin{equation*}
		x^2 = ay^2 + by +c
		,
	\end{equation*}
	а кривая $B$ определяется уравнением второго порядка общего вида.
	Система этих двух уравнений легко сводится к уравнению с одной неизвестной 4-го порядка.
	Если это уравнение не является тривиальным, то оно имеет не более 4 корней.
\end{proof}

Если $A$ и $B$~--- пары прямых $xy=0$ и $xy=x$,
то $A\cap B$ содержит ось игреков.
Поэтому предположение в лемме~\ref{thmQuadCurves} о том,
что одна из кривых не является парой прямых, существенно.

\begin{lemma}
	\label{thm:quadCurveFamily}
	Пусть $M_1, M_2 \in \mathbb{R}^2$ и
	$|M_1 - M_2| = n \in \mathbb{N}$.
	Тогда
	\begin{equation}
		\label{eq:quadCurveFamily}
		\left\{
			M: |M-M_1|\in\mathbb{N}, |M-M_2|\in\mathbb{N}
		\right\}
		=\bigcup_{i=1}^n Q_i
		,
	\end{equation}
	где $Q_1$ есть пара взаимно перпендикулярных прямых (крест)
	и $Q_2, Q_3,...,Q_n$~--- гиперболы.
\end{lemma}

\begin{proof}
	По неравенству треугольника
	\begin{equation*}
		\bigl| |M-M_1| - |M-M_2| \bigr|
		\leqslant
		|M_1 - M_2| = n
		.
	\end{equation*}
	Так как $\bigl| |M-M_1| - |M-M_2| \bigr|$
	есть целое неотрицательное число, то
	\begin{equation*}
		\bigl| |M-M_1| - |M-M_2| \bigr|
		\in\{0,1,2,...,n\}
		.
	\end{equation*}
	Ясно, что множество
	\begin{equation}
		\label{eq:centrPerp}
		\left\{ M: \bigl| |M-M_1| - |M-M_2| \bigr| = 0 \right\}
	\end{equation}
	принадлежит прямой, проходящей через середину отрезка
	$[M_1, M_2]$ перпендикулярно этому отрезку,
	\begin{equation}
		\label{eq:tails}
		\left\{ M: \bigl| |M-M_1| - |M-M_2| \bigr| = n \right\}
	\end{equation}
	содержится в прямой, проходящей через точки $M_1, M_2$,
	\begin{equation}
		\label{eq:hyperbola}
		\left\{ M: \bigl| |M-M_1| - |M-M_2| \bigr| = k \right\}
	\end{equation}
	есть гипербола для любого $k=1,2,...,n-1$.
	Обозначая объединение множеств \eqref{eq:centrPerp} и \eqref {eq:tails}
	через $Q_1$,
	а множества \eqref{eq:hyperbola} через $Q_i$,
	мы получаем требуемое представление.
	Таким образом, \eqref{eq:quadCurveFamily}
	есть объединение $n$ кривых второго порядка.
\end{proof}

\begin{proof}[Доказательство теоремы \ref{thmErdos}]
	Предположим противное,
	т.е. $\{M_1, M_2, ... \}\in\mathfrak{M}$
	и точки $M_1$, $M_2$, $M_3$ не лежат на одной прямой.
	Применяя лемму~\ref{thm:quadCurveFamily}
	к паре точек $M_1, M_2$ и к паре точек $M_1, M_3$,
	найдём такие кривые 2-го порядка
	$Q_1, Q_2,...,.Q_n$ и $R_1, R_2,...,.R_m$, что
	\begin{equation*}
		\left\{
			M: |M-M_1|\in\mathbb{N}, |M-M_2|\in\mathbb{N}
		\right\}
		=\bigcup_{i=1}^n Q_i
		,
	\end{equation*}
	\begin{equation*}
		\left\{
			M: |M-M_1|\in\mathbb{N}, |M-M_3|\in\mathbb{N}
		\right\}
		=\bigcup_{j=1}^m R_j
		.
	\end{equation*}
	Отсюда
	\begin{multline*}
		\left\{
			M:
			|M-M_1|\in\mathbb{N},
			|M-M_2|\in\mathbb{N},
			|M-M_3|\in\mathbb{N}
		\right\}
		\subset
		\\ \subset
		\left(\bigcup_{i=1}^n Q_i\right)
		\cap
		\left(\bigcup_{j=1}^m R_j\right)
		=
		\bigcup_{i=1}^n \bigcup_{j=1}^m \left( Q_i \cap R_i \right)
		.
	\end{multline*}
	Из предположения о том, что $M_1$, $M_2$, $M_3$
	не лежат на одной прямой, вытекает,
	что $Q_1\cap R_1$ не содержит прямую.
	Поэтому $|Q_1\cap R_1|\leqslant 4$,
	и, следовательно, в силу леммы~\ref{thmQuadCurves}
	$|Q_i\cap R_j|\leqslant 4$
	для всех $i=1,2,...,n$, $j=1,2,...,m$.
	Отсюда
	\begin{equation*}
		|M| \leqslant 4nm
		,
	\end{equation*}
	т.е. $M$ конечно,
	что проиворечит предположению о бесконечности $M$.
	Полученное проиворечие доказывает, что все точки $M$
	принадлежат некоторой прямой $L$.
	В качестве $L$ можно взять прямую,
	проходящую через пару любых точек из $M$.
\end{proof}

Предположение о бесконечности $M$ в теореме~\ref{thmErdos} существенно.
Для заданного $n\in\mathbb{N}$ обозначим через $\mathfrak{M}_n$
множество таких $M\in\mathfrak{M}$, что
$|M|=n$ и $M \not\subset L$ для любой прямой $L \subset\mathbb{R}^2$.

\begin{theorem}
	\label{thm:power_exist}
	Для любого $n\in\mathbb{N}$ выполнено $\mathfrak{M}_n\neq\varnothing$.
\end{theorem}

\begin{proof}
	Из теоремы Пифагора следует, что треугольник со сторонами
	$2k+1$, $2k^2+2k$, $2k^2+2k+1$
	является прямоугольным для любого $k\in\mathbb{N}$.
	Выпишем стороны первых трёх треугольников,
	а затем стороны подобных им треугольников с совпадающей первой стороной:
	\begin{equation}
		\label{eq:PifagTriangles}
		\begin{array}{ll}
			3,  4,  5 & \quad 105, 140, 175 \\
			5, 12, 13 & \quad 105, 252, 273 \\
			7, 24, 25 & \quad 105, 360, 375.
		\end{array}
	\end{equation}
	Ясно, точки $(0,0)$, $(0,105)$, $(140,0)$, $(252,0)$, $(360,0)$
	образуют множество из $\mathfrak{M}_5$.
	Очевидно, если взять не 3, а $n$ точек, и затем найти подобные
	прямоугольные треугольники с совпадающей первой стороной,
	то можно построить множество из $\mathfrak{M}_{n+2}$.
\end{proof}

Другой пример множества из $\mathfrak{M}_{n}$ для заданного $n$
можно найти в доказательстве теоремы~\ref{thm:power_char_exist},
уверждение которой усиливает утверждение теоремы~\ref{thm:power_exist}.


\begin{lemma}
	\label{lemma_points_on_line}
	Пусть $S\in\mathfrak{M}_n$ для некоторого $n \geqslant 3$,
	$\{M_1, M_2, M_3, M_4\} \subset S$, как минимум три из этих точек различны,
	$|M_1 - M_2| = |M_3 - M_4| = 1$.
	Тогда никакая прямая $m$ не содержит точки $M_1$, $M_2$, $M_3$ и $M_4$ одновременно.
\end{lemma}

\begin{proof}
	Предположим противное.
	Обозначим через $m_{12}$ и $m_{34}$ серединные перпендикуляры к отрезкам $M_1 M_2$ и $M_3 M_4$ соответственно.

	Пусть точка $M\in S$.
	Тогда либо $|M - M_1| - |M - M_2| = 0$ и $M\in m_{12}$, либо $\Bigl||M - M_1| - |M - M_2|\Bigr| = 1$ и $M\in m$,
	т.е. в любом случае $M\in m \cup m_{12}$.
	Аналогично $M\in m \cup m_{34}$.
	Следовательно, $M\in (m \cup m_{12}) \cap (m \cup m_{34}) = m \cup (m_{12} \cap m_{34}) = m$,
	т.к. $m_{12} \cap m_{34} = \varnothing$ (перпендикуляры к одной прямой, проведённые в разных точках, не пересекаются между собой).

	В силу произвольности выбора $M \in S$ получаем $S \subset m$, что противоречит $S\in\mathfrak{M}_n$.
\end{proof}

\begin{corollary}
	\label{corollary:max_points_on_line}
	Пусть $S\in\mathfrak{M}_n$, $\operatorname{diam} S = d$.
	Тогда ни на какой прямой не лежит более $(d+3)/2$ точек из $S$.
\end{corollary}

\begin{proof}
	Предположим противное.
	Очевидно, что на прямой $m$ может быть не более $d+1$ точек системы $S$.

	Пусть сначала $d$ чётно.
	Тогда из каждой пары соседних точек, кроме, быть может, одной пары,
	нужно выбросить хотя бы одну точку в силу леммы~\ref{lemma_points_on_line}.
	Итого мы выбросим не менее $d/2$ точек,
	останется не более $(d+2)/2$ точек.

	Пусть теперь $d$ нечётно.
	Тогда нужно выбросить не менее $(d-1)/2$ точек.
	Останется не более $(d+3)/2$ точек.
\end{proof}

\begin{definition}
	Будем говорить, что подмножество плоскости $S\subset\mathbb{R}^2$
	вписано в квадрат со стороной $d$, если
	можно построить такой квадрат $Q$ со стороной $d$,
	что каждая точка $S$ окажется либо внутри $Q$, либо на одной из его сторон.
\end{definition}

\begin{lemma}
	\label{lemma:square_container}
	Пусть $S\in\mathfrak{M}_n$, $\operatorname{diam} S = d$.
	Тогда $S$ можно вписать в квадрат со стороной $d$.
\end{lemma}

\begin{proof}
	Пусть $M_1, M_2 \in S$, $|M_1 - M_2| = d$,
	т.е. диаметр системы $S$ достигается на $M_1$  $M_2$.
	Введём прямоугольную декартову систему координат на плоскости таким образом, что
	$M_1 = (-d/2; 0)$, $M_1 = (d/2; 0)$.

	Тогда прочие точки $M_i \in S$ имеют координаты $M_i=(x_i, y_i)$.
	Очевидно, что для $i>2$ имеем $|x_i| < d/2$
	(иначе $|M_i - M_1| > d$ или $|M_i - M_2| > d$).
	Более того, $|y_i| < d$ (иначе $|M_i - M_1| > d$ или $|M_i - M_2| > d$).

	Пусть $y_+ = \max_{i} y_i$, $y_- = \min_{i} y_i$, $M_+=(x_+, y_+)$, $M_-=(x_-, y_-)$
	(если максимум досигается на нескольких точках, возьмём любые).
	Тогда
	\begin{multline}
		d \geq |M_+ M_-| = \sqrt{(x_+ - x_-)^2 + (y_+ - y_-)^2}
		\geq \sqrt{(y_+ - y_-)^2} = y_+ - y_-
	\end{multline}
	Следовательно, $S$ можно вписать в квадрат со сторонами $x=\pm d/2$,
	$y=y_+$, $y=y_+ - d$.
\end{proof}

\begin{remark}
	Скорее всего, улучшить оценку на сторону квадрата, сделав её меньше диаметра,
	не получится: часто встречаются конструкции систем Эрдёша, расположенных на окружности
	~\cite{anning1915discussions,harborth1993upper,piepmeyer1996maximum,kurz2008bounds,our-vvmsh-2018}.
\end{remark}

\begin{definition}
	Крестом точек $M_1$ и $M_2$ будем называть объединение прямой,
	проходящей через эти точки,
	и серединного перпендикуляра к отрезку $M_1 M_2$
	и обозначать $cr(M_1,M_2)$.
\end{definition}

\begin{proposition}
	\label{proposition:intervals_cross}
	Если интервалы $M_1 M_2$ и $M_3 M_4$ не пересекаются,
	$M_1 \neq M_2$, $M_3 \neq M_4$,
	то $cr(M_1,M_2) \cap cr(M_3,M_4)$~--- либо не более 4 точек, либо прямая.
\end{proposition}

\begin{proof}
	Пересечением $cr(M_1,M_2) \cap cr(M_3,M_4)$ может быть либо от 2 до 4 точек, либо прямая,
	либо, если эти кресты совпадают, их пересечение совпадает с каждым из этих крестов.
	Но $cr(M_1,M_2) \neq cr(M_3,M_4)$, потому что середины отрезков $M_1 M_2$ и $M_3 M_4$
	совпасть не могут, т.к. интервалы $M_1 M_2$ и $M_3 M_4$ не пересекаются по условию.
\end{proof}

\begin{lemma}
	\label{lemma_preliminary_size}
	Пусть $S$~--- система Эрдёша,
	$M_1, M_2, M_3, M_4 \in S$,
	$M_1 \neq M_2$, $M_3 \neq M_4$,
	интервалы $M_1 M_2$ и $M_3 M_4$ не пересекаются,
	$d = \operatorname{diam} S > 5$.
	Тогда $|S| \leq 4 \cdot |M_1 - M_2| \cdot |M_3 - M_4| + \frac{d-5}{2}$.
\end{lemma}

\begin{proof}
	Если $|(cr(M_1, M_2) \cap cr(M_3, M_4))| < \infty$,
	то (см. доказательство теоремы~\ref{thmErdos} или~\cite[часть 2, неравенство (1)]{solymosi2003note})
	$|S| \leq 4 \cdot |M_1 - M_2| \cdot |M_3 - M_4|$.
	Иначе (в силу утверждения~\ref{proposition:intervals_cross}) $cr(M_1, M_2) \cap cr(M_3, M_4) = m$,
	где $m$~--- прямая.
	В силу следствия~\ref{corollary:max_points_on_line} на этой прямой лежит не более $(d+3)/2$ точек;
	кроме того, из общего количества точек нужно вычесть 4,
	которые бы дали эти 2 креста в случае дискретного переcечения.
	Получаем верхнюю оценку
	\begin{multline}
		|S| \leq 4 \cdot |M_1 - M_2| \cdot |M_3 - M_4| - 4 + \frac{d+3}{2}
		=
		\\=
		4 \cdot |M_1 - M_2| \cdot |M_3 - M_4| + \frac{d-5}{2}
		.
	\end{multline}
\end{proof}

Определим теперь коэффициенты упаковки точек в квадрат $\varphi_k$.
Пусть
\begin{equation*}
	\Phi_k = \{ P \subset [0;1]^2 : |P|=k\}
	,
\end{equation*}
где $[0;1]^2$~--- замкнутый единичный квадрат на плоскости.
Тогда
\begin{equation*}
	\varphi_k = \max_{P \in \Phi_k} \min_{A,B \in P} |A - B|
	.
\end{equation*}
Иначе говоря, через $\varphi_k$ будем обозначать наибольшее число, такое,
что в единичном квадрате нельзя разместить $k$ точек так,
чтобы расстояние между любыми двумя точками было не менее $\varphi_k$.
Проблема отыскания $\varphi_k$ носит название проблемы упаковки точек в квадрат~\cite{locatelli2002packing,costa2013valid}.


\begin{lemma}
	Пусть $S$~--- система Эрдёша,
	$d = \operatorname{diam} S > 5$,
	$k \geq 2$,
	$m \geq 2$,
	$ |S| = (k-1)m^2 + 2$.

	Тогда
	\begin{equation}
		d \geq \mu (|S| - 2),
	\end{equation}
	где
	\begin{equation}
		\mu = \frac{\sqrt{64\varphi_k^2 (k-1)+1}-1}{16\varphi_k^2 (k-1)}
	\end{equation}
\end{lemma}

\begin{proof}
	Впишем $S$ в квадрат со стороной $d$ (см. лемму~\ref{lemma:square_container})
	и разобьём этот квадрат на $m^2$ маленьких квадратов со стороной $d/m$.
	Тогда по принципу Дирихле либо:

	а) хотя бы в два маленьких квадрата $Q_1$ и $Q_2$ попало не менее чем по $k$ точек;

	б) хотя бы в один маленький квадрат $Q_1$ попало не менее чем $k+1$ точка.
	\\
	В случае (а) в $Q_1$ выберем точки $M_1$ и $M_2$,
	а в $Q_2$ выберем точки $M_3$ и $M_4$
	так, что $|M_1 - M_2| \leq \varphi_k d /m$, $|M_3 - M_4| \leq \varphi_k d/m$
	(это всегда можно сделать в силу определения $\varphi_k$).

	В случае (б) в $Q_1$ выберем точки $M_1$ и $M_2$ так, что
	$|M_1 - M_2| \leq \varphi_{k+1} d /m \leq \varphi_k d /m$.
	Из всех точек, кроме $M_1$, выберем $M_3$ и $M_4$ так, что
	$|M_3 - M_4| \leq \varphi_k d/m$.
	Возможное совпадение $M_2$ и $M_3$ не помешает дальнейшему доказательству.

	TODO: реверанс про выпуклый четырёхугольник.

	По лемме~\ref{lemma_preliminary_size}
	\begin{equation}
		|S| \leq 4 \left( \frac{d}{m} \varphi_k \right)^2 + \frac{d}{2} - \frac{5}{2}
	\end{equation}
	или, с учётом того, что $ |S| = (k-1)m^2 + 2$,
	\begin{equation}
		 4 \left( \frac{d}{m} \varphi_k \right)^2 + \frac{d}{2} - \left( (k-1)m^2 + \frac{9}{2}\right) \geq 0
	\end{equation}

	Положим $d = \lambda m$:
	\begin{equation}
		 4 \lambda^2 \varphi_k^2 + \frac{m}{2} \lambda - \left( (k-1)m^2 + \frac{9}{2}\right) \geq 0
	\end{equation}

	Решим это квадратное неравенство относительно $\lambda$,
	зная, что $\lambda > 0$.

	Непосредственно выпишем дискриминант:
	\begin{multline}
		D =
		\frac{m^2}{4} + 4 \cdot 4 \varphi_k^2 \cdot \left( (k-1)m^2 + \frac{9}{2}\right)
		=
		\frac{m^2}{4} + 16 \varphi_k^2 \cdot \left( (k-1)m^2 + \frac{9}{2}\right)
		>\\>
		\frac{m^2}{4} + 16 \varphi_k^2 \cdot (k-1)m^2
		=
		\frac{m^2}{4} \cdot (1 + 64 \varphi_k^2 \cdot (k-1))
		> 0
	\end{multline}

	Итак,
	\begin{multline}
		\frac{d}{m} = \lambda >
		\frac{-\frac{m}{2} + \sqrt{\frac{m^2}{4} \cdot (1 + 64 \varphi_k^2 \cdot (k-1))} }{8 \varphi_k^2}
		=\\=
		\frac{-m + \sqrt{m^2  (1 + 64 \varphi_k^2 \cdot (k-1))} }{16 \varphi_k^2}
		=
		m\frac{ \sqrt{ 64 \varphi_k^2 (k-1) + 1} -1 }{16 \varphi_k^2}
	\end{multline}
	Разделим обе части неравенства на положительное число $m(k-1)$:
	\begin{equation}
		\frac{d}{(k-1)m^2}
		>
		\frac{ \sqrt{ 64 \varphi_k^2 (k-1) + 1} -1 }{16 \varphi_k^2 (k-1)}
	\end{equation}
	Положим
	\begin{equation}
		\mu = \frac{ \sqrt{ 64 \varphi_k^2 (k-1) + 1} -1 }{16 \varphi_k^2 (k-1)},
	\end{equation}
	тогда
	\begin{equation}
		\frac{d}{(k-1)m^2}
		>
		\mu
	\end{equation}
	откуда незамедлительно
	\begin{equation}
		d > \mu \cdot (|S|-2)
		.
	\end{equation}
\end{proof}


В~\cite{costa2013valid} приведены значения $\varphi_k$ до $k=10$ включительно

TODO: разобраться, насколько эти значения точны и откуда они взяты =/

По их расчётам получаем $\varphi_{10} = 0.4214$,
что даёт оценку $d>0.3583(|S|-2)$.

Это, конечно, для $|S|$ специального вида, но множитель $(m+1)^2 / m^2$ съестся на бесконечности,
а итоговые коэфициенты и так будем брать с запасом (кто их там знает, как они округляли).

Это для $|S| \geq 41$, но для меньших мощностей посчитано (и не один раз, в том числе нами).

Кстати, для $|S| = 3$ не проходит (и не должно).

А особенная приятность в том, что вычисление новых $\varphi_k$ может давать автоматическое улучшение оценки
(может, правда, и не давать).
Однако, как ни печально, константу не получится сделать даже 0.42 (есть нижняя оценка на $\varphi_k$).

Следующий прикол:
де-факто нам нужна не какая попало расстановка точек в единичном квадрате,
а расстановка с рациональными расстояниями (они потенциально могут превратиться в целые при умножении на $d/m$).
А возможность аппроксимировать произвольную расстановку расстановкой с рациональными расстояниями
уже имеет прямое отношение к проблеме Улама-Эрдёша (ссылка!!).


%\inputencoding{cp1251}

%%%%%%%%%%%%%%%%%%%%%%%%%%%%%%%%%%%%%%%%%%%%%%

% Для оформления теорем, пожалуйста, используйте следующий образец
%
\begin{theorem}
	Всякое
\end{theorem}


 \textbf{Теорема 1.} \textsl{Текст формулировки теоремы.}
%
% Утверждения и следствия оформляются так же, как и теоремы.
%
%
%
% Для оформления определений, замечаний и примеров, пожалуйста, используйте
% следующий образец
%

\begin{remark*}
	Всякое
\end{remark*}


\textsc{Замечание.} Текст замечания.
%
%
%
% Для оформления благодарностей, пожалуйста, используйте следующий образец
%
% {\small Работа выполнена при поддержке Российского фонда фундаментальных
% исследований (\No~08-01-00888).}
%
%
%
% Для оформления списка литературы, пожалуйста, используйте следующий образец
%

\par\bigskip\centerline{\bf Литература}\smallskip

 \begin{enumerate}

 \itemsep=0pt\parskip=0pt


\bib{Фамилия1 И1.~О1.}{Название книги.---Город: Издательство, год.---С.~??.}

\bib{Фамилия1 И1.~О1., Фамилия2 И2.~О2.} {Название статьи в
журнале~/\!/ Название журнала.---Год.---Т.~?, No~?.---С.~??--??.}

\bib{Фамилия1 И1.~О1., Фамилия2 И2.~О2., Фамилия3 И3.~О3.}
{Название статьи в сборнике~/\!/ Название сборника.---Город:
Издательство, год.---С.~??--??.}

 \end{enumerate}




\end{document}
