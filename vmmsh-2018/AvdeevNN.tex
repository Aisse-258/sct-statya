%%%%%%%%%%%%%%%%%%%%%%%%%%%%%%%%%%%%%%%%%%%%%%%%%%%%%%%%%%%%%%%%%%%%%%%
%% Тезисы доклада на XIV «Владикавказская молодежная математическая школа»
%%  Россия, РСО-А, с. Цей, 16-21 июля 2018 г.
%%%%%%%%%%%%%%%%%%%%%%%%%%%%%%%%%%%%%%%%%%%%%%%%%%%%%%%%%%%%%%%%%%%%%%%%

% Загрузка файла style.tex со служебными командами
\input style.tex

\mag 1468

%%%%%%%%%%%%%%%%%%%%%%%%%%%%%%%%%%%%%%%%%%%%%%
% Пожалуйста, не меняйте порядок команд!
%%%%%%%%%%%%%%%%%%%%%%%%%%%%%%%%%%%%%%%%%%%%%%
%
% Название доклада

УДК 511.95 + 514.112


\Title{МНОЖЕСТВА ТОЧЕК С ЦЕЛОЧИСЛЕННЫМИ РАССТОЯНИЯМИ НА ПЛОСКОСТИ И В ЕВКЛИДОВОМ ПРОСТРАНСТВЕ} % обязательное поле!



%%{{ А вот этот фрагмент обеспечит использование автоматической нумерации формул.
%%   Обычно её запрещают из-за того, что при автоматической нумерации номера формул у всех авторов будут "общие":
%%   если тезисы n-го по порядку автора имели последнюю формулу с номером m,
%%   то первая формула (n+1)-го автора будет иметь первую формулу с номером m+1,
%%   чего быть не должно.
%%   Блок строчек ниже сбрасывает нумерацию формул, рисунков, таблиц и сносок.
%%   Настоятельно рекомендую добавить эту последовательность команд в определение команды \Title в файле style.tex
	\setcounter{equation}{0}
	\setcounter{section}{0}
	\setcounter{figure}{0}
	\setcounter{table}{0}
	\setcounter{footnote}{0}
	\renewcommand{\theenumi}{\arabic{enumi}}
	\renewcommand{\labelenumi}{\theenumi)}
	\selectlanguage{russian}
%%}}




%
% Авторы

\Author{Авдеев Н.~Н.$^{1}$, Семенов Е.~М.$^{2}$}
% обязательное поле!
\\
\affiliation{$^{1}$Воронежский госуниверситет, Воронеж, Россия}
\email{nickkolok@mail.ru}
\\
\affiliation{$^{2}$Воронежский госуниверситет, Воронеж, Россия}
\email{nadezhka\_ssm@geophys.vsu.ru}

%\inputencoding{utf8}


Oбозначим через $\mathfrak{M}$ множество таких подмножеств
$
	M = \{M_1, M_2, ... \}\subset \mathbb{R}^2
$,
что
$
	|M_i - M_j|\in\mathbb{N}
$
для всех $i$, $j$, где $\mathbb{N}$~--- множество натуральных чисел
и $|M_i - M_j|$~--- расстояние между точками $M_i$ и $M_j$.
П. Эрдёш доказал~\cite{erdos1945integral},
что справедлива следующая

\begin{theorem}
	\label{thmErdos}
	Всякое бесконечное подмножество $M\in\mathfrak{M}$
	содержится в некоторой прямой $L\in\mathbb{R}^2$.
\end{theorem}

Теорему~\ref{thmErdos} можно доказать, основываясь на двух элементарных леммах.
Интересно, что работа~\cite{erdos1945integral} была опубликована через несколько
месяцев после совместной работы Эннинга и Эрдёша~\cite{anning1945integral},
в которой было дано другое, более сложное доказательство.

\begin{lemma}
	\label{thmQuadCurves}
	Пусть $A$ и $B$~--- две кривые второго порядка на плоскости и $A$ не является парой прямых.
	Тогда $A\cap B$ содержит не более 4 точек.
\end{lemma}

\begin{proof}
	Без ограничения общности кривая $A$ определяется уравнением
	\begin{equation*}
		x^2 = ay^2 + by +c
		,
	\end{equation*}
	а кривая $B$ определяется уравнением второго порядка общего вида.
	Система этих двух уравнений легко сводится к уравнению с одной неизвестной 4-го порядка.
	Если это уравнение не является тривиальным, то оно имеет не более 4 корней.
\end{proof}

Если $A$ и $B$~--- пары прямых $xy=0$ и $xy=x$,
то $A\cap B$ содержит ось игреков.
Поэтому предположение в лемме~\ref{thmQuadCurves} о том,
что одна из кривых не является парой прямых, существенно.

\begin{lemma}
	\label{thm:quadCurveFamily}
	Пусть $M_1, M_2 \in \mathbb{R}^2$ и
	$|M_1 - M_2| = n \in \mathbb{N}$.
	Тогда
	\begin{equation}
		\label{eq:quadCurveFamily}
		\left\{
			M: |M-M_1|\in\mathbb{N}, |M-M_2|\in\mathbb{N}
		\right\}
		=\bigcup_{i=1}^n Q_i
		,
	\end{equation}
	где $Q_1$ есть пара взаимно перпендикулярных прямых (крест)
	и $Q_2, Q_3,...,Q_n$~--- гиперболы.
\end{lemma}

\begin{proof}
	По неравенству треугольника
	\begin{equation*}
		\bigl| |M-M_1| - |M-M_2| \bigr|
		\leqslant
		|M_1 - M_2| = n
		.
	\end{equation*}
	Так как $\bigl| |M-M_1| - |M-M_2| \bigr|$
	есть целое неотрицательное число, то
	\begin{equation*}
		\bigl| |M-M_1| - |M-M_2| \bigr|
		\in\{0,1,2,...,n\}
		.
	\end{equation*}
	Ясно, что множество
	\begin{equation}
		\label{eq:centrPerp}
		\left\{ M: \bigl| |M-M_1| - |M-M_2| \bigr| = 0 \right\}
	\end{equation}
	принадлежит прямой, проходящей через середину отрезка
	$[M_1, M_2]$ перпендикулярно этому отрезку,
	\begin{equation}
		\label{eq:tails}
		\left\{ M: \bigl| |M-M_1| - |M-M_2| \bigr| = n \right\}
	\end{equation}
	содержится в прямой, проходящей через точки $M_1, M_2$,
	\begin{equation}
		\label{eq:hyperbola}
		\left\{ M: \bigl| |M-M_1| - |M-M_2| \bigr| = k \right\}
	\end{equation}
	есть гипербола для любого $k=1,2,...,n-1$.
	Обозначая объединение множеств \eqref{eq:centrPerp} и \eqref {eq:tails}
	через $Q_1$,
	а множества \eqref{eq:hyperbola} через $Q_i$,
	мы получаем требуемое представление.
	Таким образом, \eqref{eq:quadCurveFamily}
	есть объединение $n$ кривых второго порядка.
\end{proof}

\begin{proof}[Доказательство теоремы \ref{thmErdos}]
	Предположим противное,
	т.е. $\{M_1, M_2, ... \}\in\mathfrak{M}$
	и точки $M_1$, $M_2$, $M_3$ не лежат на одной прямой.
	Применяя лемму~\ref{thm:quadCurveFamily}
	к паре точек $M_1, M_2$ и к паре точек $M_1, M_3$,
	найдём такие кривые 2-го порядка
	$Q_1, Q_2,...,.Q_n$ и $R_1, R_2,...,.R_m$, что
	\begin{equation*}
		\left\{
			M: |M-M_1|\in\mathbb{N}, |M-M_2|\in\mathbb{N}
		\right\}
		=\bigcup_{i=1}^n Q_i
		,
	\end{equation*}
	\begin{equation*}
		\left\{
			M: |M-M_1|\in\mathbb{N}, |M-M_3|\in\mathbb{N}
		\right\}
		=\bigcup_{j=1}^m R_j
		.
	\end{equation*}
	Отсюда
	\begin{multline*}
		\left\{
			M:
			|M-M_1|\in\mathbb{N},
			|M-M_2|\in\mathbb{N},
			|M-M_3|\in\mathbb{N}
		\right\}
		\subset
		\\ \subset
		\left(\bigcup_{i=1}^n Q_i\right)
		\cap
		\left(\bigcup_{j=1}^m R_j\right)
		=
		\bigcup_{i=1}^n \bigcup_{j=1}^m \left( Q_i \cap R_i \right)
		.
	\end{multline*}
	Из предположения о том, что $M_1$, $M_2$, $M_3$
	не лежат на одной прямой, вытекает,
	что $Q_1\cap R_1$ не содержит прямую.
	Поэтому $|Q_1\cap R_1|\leqslant 4$,
	и, следовательно, в силу леммы~\ref{thmQuadCurves}
	$|Q_i\cap R_j|\leqslant 4$
	для всех $i=1,2,...,n$, $j=1,2,...,m$.
	Отсюда
	\begin{equation*}
		|M| \leqslant 4nm
		,
	\end{equation*}
	т.е. $M$ конечно,
	что проиворечит предположению о бесконечности $M$.
	Полученное проиворечие доказывает, что все точки $M$
	принадлежат некоторой прямой $L$.
	В качестве $L$ можно взять прямую,
	проходящую через пару любых точек из $M$.
\end{proof}

Предположение о бесконечности $M$ в теореме~\ref{thmErdos} существенно.
Для заданного $n\in\mathbb{N}$ обозначим через $\mathfrak{M}_n$
множество таких $M\in\mathfrak{M}$, что
$|M|=n$ и $M \not\subset L$ для любой прямой $L \subset\mathbb{R}^2$.

\begin{theorem}
	\label{thm:power_exist}
	Для любого $n\in\mathbb{N}$ выполнено $\mathfrak{M}_n\neq\varnothing$.
\end{theorem}

\begin{proof}
	Из теоремы Пифагора следует, что треугольник со сторонами
	$2k+1$, $2k^2+2k$, $2k^2+2k+1$
	является прямоугольным для любого $k\in\mathbb{N}$.
	Выпишем стороны первых трёх треугольников,
	а затем стороны подобных им треугольников с совпадающей первой стороной:
	\begin{equation}
		\label{eq:PifagTriangles}
		\begin{array}{ll}
			3,  4,  5 & \quad 105, 140, 175 \\
			5, 12, 13 & \quad 105, 252, 273 \\
			7, 24, 25 & \quad 105, 360, 375.
		\end{array}
	\end{equation}
	Ясно, точки $(0,0)$, $(0,105)$, $(140,0)$, $(252,0)$, $(360,0)$
	образуют множество из $\mathfrak{M}_5$.
	Очевидно, если взять не 3, а $n$ точек, и затем найти подобные
	прямоугольные треугольники с совпадающей первой стороной,
	то можно построить множество из $\mathfrak{M}_{n+2}$.
\end{proof}

Другой пример множества из $\mathfrak{M}_{n}$ для заданного $n$
можно найти в доказательстве теоремы~\ref{thm:power_char_exist},
уверждение которой усиливает утверждение теоремы~\ref{thm:power_exist}.


\begin{lemma}
	\label{lemma_points_on_line}
	Пусть $S\in\mathfrak{M}_n$ для некоторого $n \geqslant 3$,
	$\{M_1, M_2, M_3, M_4\} \subset S$, как минимум три из этих точек различны,
	$|M_1 - M_2| = |M_3 - M_4| = 1$.
	Тогда никакая прямая $m$ не содержит точки $M_1$, $M_2$, $M_3$ и $M_4$ одновременно.
\end{lemma}

\begin{proof}
	Предположим противное.
	Обозначим через $m_{12}$ и $m_{34}$ серединные перпендикуляры к отрезкам $M_1 M_2$ и $M_3 M_4$ соответственно.

	Пусть точка $M\in S$.
	Тогда либо $|M - M_1| - |M - M_2| = 0$ и $M\in m_{12}$, либо $\Bigl||M - M_1| - |M - M_2|\Bigr| = 1$ и $M\in m$,
	т.е. в любом случае $M\in m \cup m_{12}$.
	Аналогично $M\in m \cup m_{34}$.
	Следовательно, $M\in (m \cup m_{12}) \cap (m \cup m_{34}) = m \cup (m_{12} \cap m_{34}) = m$,
	т.к. $m_{12} \cap m_{34} = \varnothing$ (перпендикуляры к одной прямой, проведённые в разных точках, не пересекаются между собой).

	В силу произвольности выбора $M \in S$ получаем $S \subset m$, что противоречит $S\in\mathfrak{M}_n$.
\end{proof}

\begin{corollary}
	\label{corollary:max_points_on_line}
	Пусть $S\in\mathfrak{M}_n$, $\operatorname{diam} S = d$.
	Тогда ни на какой прямой не лежит более $(d+3)/2$ точек из $S$.
\end{corollary}

\begin{proof}
	Предположим противное.
	Очевидно, что на прямой $m$ может быть не более $d+1$ точек системы $S$.

	Пусть сначала $d$ чётно.
	Тогда из каждой пары соседних точек, кроме, быть может, одной пары,
	нужно выбросить хотя бы одну точку в силу леммы~\ref{lemma_points_on_line}.
	Итого мы выбросим не менее $d/2$ точек,
	останется не более $(d+2)/2$ точек.

	Пусть теперь $d$ нечётно.
	Тогда нужно выбросить не менее $(d-1)/2$ точек.
	Останется не более $(d+3)/2$ точек.
\end{proof}

\begin{definition}
	Будем говорить, что подмножество плоскости $S\subset\mathbb{R}^2$
	вписано в квадрат со стороной $d$, если
	можно построить такой квадрат $Q$ со стороной $d$,
	что каждая точка $S$ окажется либо внутри $Q$, либо на одной из его сторон.
\end{definition}

\begin{lemma}
	\label{lemma:square_container}
	Пусть $S\in\mathfrak{M}_n$, $\operatorname{diam} S = d$.
	Тогда $S$ можно вписать в квадрат со стороной $d$.
\end{lemma}

\begin{proof}
	Пусть $M_1, M_2 \in S$, $|M_1 - M_2| = d$,
	т.е. диаметр системы $S$ достигается на $M_1$  $M_2$.
	Введём прямоугольную декартову систему координат на плоскости таким образом, что
	$M_1 = (-d/2; 0)$, $M_1 = (d/2; 0)$.

	Тогда прочие точки $M_i \in S$ имеют координаты $M_i=(x_i, y_i)$.
	Очевидно, что для $i>2$ имеем $|x_i| < d/2$
	(иначе $|M_i - M_1| > d$ или $|M_i - M_2| > d$).
	Более того, $|y_i| < d$ (иначе $|M_i - M_1| > d$ или $|M_i - M_2| > d$).

	Пусть $y_+ = \max_{i} y_i$, $y_- = \min_{i} y_i$, $M_+=(x_+, y_+)$, $M_-=(x_-, y_-)$
	(если максимум досигается на нескольких точках, возьмём любые).
	Тогда
	\begin{multline}
		d \geq |M_+ M_-| = \sqrt{(x_+ - x_-)^2 + (y_+ - y_-)^2}
		\geq \sqrt{(y_+ - y_-)^2} = y_+ - y_-
	\end{multline}
	Следовательно, $S$ можно вписать в квадрат со сторонами $x=\pm d/2$,
	$y=y_+$, $y=y_+ - d$.
\end{proof}

\begin{remark}
	Скорее всего, улучшить оценку на сторону квадрата, сделав её меньше диаметра,
	не получится: часто встречаются конструкции систем Эрдёша, расположенных на окружности
	~\cite{anning1915discussions,harborth1993upper,piepmeyer1996maximum,kurz2008bounds,our-vvmsh-2018}.
\end{remark}

\begin{definition}
	Крестом точек $M_1$ и $M_2$ будем называть объединение прямой,
	проходящей через эти точки,
	и серединного перпендикуляра к отрезку $M_1 M_2$
	и обозначать $cr(M_1,M_2)$.
\end{definition}

\begin{proposition}
	\label{proposition:intervals_cross}
	Если интервалы $M_1 M_2$ и $M_3 M_4$ не пересекаются,
	$M_1 \neq M_2$, $M_3 \neq M_4$,
	то $cr(M_1,M_2) \cap cr(M_3,M_4)$~--- либо не более 4 точек, либо прямая.
\end{proposition}

\begin{proof}
	Пересечением $cr(M_1,M_2) \cap cr(M_3,M_4)$ может быть либо от 2 до 4 точек, либо прямая,
	либо, если эти кресты совпадают, их пересечение совпадает с каждым из этих крестов.
	Но $cr(M_1,M_2) \neq cr(M_3,M_4)$, потому что середины отрезков $M_1 M_2$ и $M_3 M_4$
	совпасть не могут, т.к. интервалы $M_1 M_2$ и $M_3 M_4$ не пересекаются по условию.
\end{proof}

\begin{lemma}
	\label{lemma_preliminary_size}
	Пусть $S$~--- система Эрдёша,
	$M_1, M_2, M_3, M_4 \in S$,
	$M_1 \neq M_2$, $M_3 \neq M_4$,
	интервалы $M_1 M_2$ и $M_3 M_4$ не пересекаются,
	$d = \operatorname{diam} S > 5$.
	Тогда $|S| \leq 4 \cdot |M_1 - M_2| \cdot |M_3 - M_4| + \frac{d-5}{2}$.
\end{lemma}

\begin{proof}
	Если $|(cr(M_1, M_2) \cap cr(M_3, M_4))| < \infty$,
	то (см. доказательство теоремы~\ref{thmErdos} или~\cite[часть 2, неравенство (1)]{solymosi2003note})
	$|S| \leq 4 \cdot |M_1 - M_2| \cdot |M_3 - M_4|$.
	Иначе (в силу утверждения~\ref{proposition:intervals_cross}) $cr(M_1, M_2) \cap cr(M_3, M_4) = m$,
	где $m$~--- прямая.
	В силу следствия~\ref{corollary:max_points_on_line} на этой прямой лежит не более $(d+3)/2$ точек;
	кроме того, из общего количества точек нужно вычесть 4,
	которые бы дали эти 2 креста в случае дискретного переcечения.
	Получаем верхнюю оценку
	\begin{multline}
		|S| \leq 4 \cdot |M_1 - M_2| \cdot |M_3 - M_4| - 4 + \frac{d+3}{2}
		=
		\\=
		4 \cdot |M_1 - M_2| \cdot |M_3 - M_4| + \frac{d-5}{2}
		.
	\end{multline}
\end{proof}

Определим теперь коэффициенты упаковки точек в квадрат $\varphi_k$.
Пусть
\begin{equation*}
	\Phi_k = \{ P \subset [0;1]^2 : |P|=k\}
	,
\end{equation*}
где $[0;1]^2$~--- замкнутый единичный квадрат на плоскости.
Тогда
\begin{equation*}
	\varphi_k = \max_{P \in \Phi_k} \min_{A,B \in P} |A - B|
	.
\end{equation*}
Иначе говоря, через $\varphi_k$ будем обозначать наибольшее число, такое,
что в единичном квадрате нельзя разместить $k$ точек так,
чтобы расстояние между любыми двумя точками было не менее $\varphi_k$.
Проблема отыскания $\varphi_k$ носит название проблемы упаковки точек в квадрат~\cite{locatelli2002packing,costa2013valid}.


\begin{lemma}
	Пусть $S$~--- система Эрдёша,
	$d = \operatorname{diam} S > 5$,
	$k \geq 2$,
	$m \geq 2$,
	$ |S| = (k-1)m^2 + 2$.

	Тогда
	\begin{equation}
		d \geq \mu (|S| - 2),
	\end{equation}
	где
	\begin{equation}
		\mu = \frac{\sqrt{64\varphi_k^2 (k-1)+1}-1}{16\varphi_k^2 (k-1)}
	\end{equation}
\end{lemma}

\begin{proof}
	Впишем $S$ в квадрат со стороной $d$ (см. лемму~\ref{lemma:square_container})
	и разобьём этот квадрат на $m^2$ маленьких квадратов со стороной $d/m$.
	Тогда по принципу Дирихле либо:

	а) хотя бы в два маленьких квадрата $Q_1$ и $Q_2$ попало не менее чем по $k$ точек;

	б) хотя бы в один маленький квадрат $Q_1$ попало не менее чем $k+1$ точка.
	\\
	В случае (а) в $Q_1$ выберем точки $M_1$ и $M_2$,
	а в $Q_2$ выберем точки $M_3$ и $M_4$
	так, что $|M_1 - M_2| \leq \varphi_k d /m$, $|M_3 - M_4| \leq \varphi_k d/m$
	(это всегда можно сделать в силу определения $\varphi_k$).

	В случае (б) в $Q_1$ выберем точки $M_1$ и $M_2$ так, что
	$|M_1 - M_2| \leq \varphi_{k+1} d /m \leq \varphi_k d /m$.
	Из всех точек, кроме $M_1$, выберем $M_3$ и $M_4$ так, что
	$|M_3 - M_4| \leq \varphi_k d/m$.
	Возможное совпадение $M_2$ и $M_3$ не помешает дальнейшему доказательству.

	TODO: реверанс про выпуклый четырёхугольник.

	По лемме~\ref{lemma_preliminary_size}
	\begin{equation}
		|S| \leq 4 \left( \frac{d}{m} \varphi_k \right)^2 + \frac{d}{2} - \frac{5}{2}
	\end{equation}
	или, с учётом того, что $ |S| = (k-1)m^2 + 2$,
	\begin{equation}
		 4 \left( \frac{d}{m} \varphi_k \right)^2 + \frac{d}{2} - \left( (k-1)m^2 + \frac{9}{2}\right) \geq 0
	\end{equation}

	Положим $d = \lambda m$:
	\begin{equation}
		 4 \lambda^2 \varphi_k^2 + \frac{m}{2} \lambda - \left( (k-1)m^2 + \frac{9}{2}\right) \geq 0
	\end{equation}

	Решим это квадратное неравенство относительно $\lambda$,
	зная, что $\lambda > 0$.

	Непосредственно выпишем дискриминант:
	\begin{multline}
		D =
		\frac{m^2}{4} + 4 \cdot 4 \varphi_k^2 \cdot \left( (k-1)m^2 + \frac{9}{2}\right)
		=
		\frac{m^2}{4} + 16 \varphi_k^2 \cdot \left( (k-1)m^2 + \frac{9}{2}\right)
		>\\>
		\frac{m^2}{4} + 16 \varphi_k^2 \cdot (k-1)m^2
		=
		\frac{m^2}{4} \cdot (1 + 64 \varphi_k^2 \cdot (k-1))
		> 0
	\end{multline}

	Итак,
	\begin{multline}
		\frac{d}{m} = \lambda >
		\frac{-\frac{m}{2} + \sqrt{\frac{m^2}{4} \cdot (1 + 64 \varphi_k^2 \cdot (k-1))} }{8 \varphi_k^2}
		=\\=
		\frac{-m + \sqrt{m^2  (1 + 64 \varphi_k^2 \cdot (k-1))} }{16 \varphi_k^2}
		=
		m\frac{ \sqrt{ 64 \varphi_k^2 (k-1) + 1} -1 }{16 \varphi_k^2}
	\end{multline}
	Разделим обе части неравенства на положительное число $m(k-1)$:
	\begin{equation}
		\frac{d}{(k-1)m^2}
		>
		\frac{ \sqrt{ 64 \varphi_k^2 (k-1) + 1} -1 }{16 \varphi_k^2 (k-1)}
	\end{equation}
	Положим
	\begin{equation}
		\mu = \frac{ \sqrt{ 64 \varphi_k^2 (k-1) + 1} -1 }{16 \varphi_k^2 (k-1)},
	\end{equation}
	тогда
	\begin{equation}
		\frac{d}{(k-1)m^2}
		>
		\mu
	\end{equation}
	откуда незамедлительно
	\begin{equation}
		d > \mu \cdot (|S|-2)
		.
	\end{equation}
\end{proof}


В~\cite{costa2013valid} приведены значения $\varphi_k$ до $k=10$ включительно

TODO: разобраться, насколько эти значения точны и откуда они взяты =/

По их расчётам получаем $\varphi_{10} = 0.4214$,
что даёт оценку $d>0.3583(|S|-2)$.

Это, конечно, для $|S|$ специального вида, но множитель $(m+1)^2 / m^2$ съестся на бесконечности,
а итоговые коэфициенты и так будем брать с запасом (кто их там знает, как они округляли).

Это для $|S| \geq 41$, но для меньших мощностей посчитано (и не один раз, в том числе нами).

Кстати, для $|S| = 3$ не проходит (и не должно).

А особенная приятность в том, что вычисление новых $\varphi_k$ может давать автоматическое улучшение оценки
(может, правда, и не давать).
Однако, как ни печально, константу не получится сделать даже 0.42 (есть нижняя оценка на $\varphi_k$).

Следующий прикол:
де-факто нам нужна не какая попало расстановка точек в единичном квадрате,
а расстановка с рациональными расстояниями (они потенциально могут превратиться в целые при умножении на $d/m$).
А возможность аппроксимировать произвольную расстановку расстановкой с рациональными расстояниями
уже имеет прямое отношение к проблеме Улама-Эрдёша (ссылка!!).


Итак, каждому множеству $M\in\mathfrak{M}_n$ сопоставлены
два натуральных числа: мощность и диаметр.
Кемнитц~\cite{kemnitz1988punktmengen} обнаружил и третье,
которое назвал характеристикой.

\begin{definition}
	\label{def_char_classic}
	Характеристикой множества $M\in\mathfrak{M}_n$ называется свободное от квадратов~\cite[гл. 34, п. 3]{Bukhstab-number-theory}
	число $p$, такое, что площадь любого треугольника $ABC$, где $A,B,C\in M$,
	соизмерима с $\sqrt{p}$.
\end{definition}

Опираясь на работы~\cite{our-mkmitu-2016,our-ped-2017} и используя~\cite[теорема 3.1]{polygons-on-lattices},
можно доказать, что определение~\ref{def_char_classic} равносильно следующему определению.
\begin{definition}
	\label{def_char_classic}
	Характеристикой множества $M\in\mathfrak{M}_n$ называется свободное от квадратов
	число $p$, такое, что множество $M$ может быть размещена на решётке
\begin{equation}\label{grid_for_Erdosh_system}
	\left\{\left(
		\frac{a_i}{2m}
		;
		\frac{b_i\sqrt{p}}{2m}
	\right)\right\},
\end{equation}
где $a_i, b_i \in \mathbb{Z}$, в качестве $m$ можно взять длину любого ребра $M$.
\end{definition}

TODO: надо ли писать доказательство?

TODO: можно ли использовать термин "ребро $M$" или его надо определить? Где?

Характеристика для данного множества $M\in\mathfrak{M}_n$ определяется единственным образом и обозначается $\operatorname{char}M$.
Заметим, что понятие характеристики можно обобщить и на множества $M\in\mathfrak{M}$,
содержащиеся в некоторой прямой.
В таком случае логично полагать $\operatorname{char} M = 0$.



\begin{theorem}
	\label{thm:power_char_exist}
	Для любого $n\geq 3$ и любого свободного от квадратов числа $p$
	существует множество $M\in\mathfrak{M}_n$ такое, что $\operatorname{char} M = p$.
\end{theorem}

\begin{proof}
	Не теряя общности положим, что $n$ нечётно.
	Пусть
	\begin{equation*}
		M = \{M_i, i =0,1,...,n+1\},
	\end{equation*}
	где
	$M_i = (2^{n-i}-p\cdot 2^{i-1},0)$, $i=0,1,...,n$;
	$M_{n+1} = (0,\sqrt{p \cdot 2^{n+1}})$.
	Очевидно, для $0\leqslant i \leqslant j \leqslant n$ выполнено
	$|M_i- M_j|\in\mathbb{N}$.
	Заметим, что для $i=1,...,n$
	\begin{multline*}
		|M_{n+1} - M_i| =
		\sqrt{p\cdot 2^{n+1} + (2^{n-i}-p\cdot2^{i-1})^2}
		=
		\\=
		\sqrt{4p\cdot 2^{n-1} + (2^{n-i})^2 - 2 \cdot 2^{n-i} \cdot p \cdot 2^{i-1} + (p\cdot 2^{i-1})^2}
		=
		\\=
		\sqrt{4p\cdot 2^{i-1} \cdot 2^{n-i} + (2^{n-i})^2 - 2 \cdot 2^{n-i} \cdot p \cdot 2^{i-1} + (p\cdot 2^{i-1})^2}
		=
		\\=
		\sqrt{(2^{n-i})^2 + 2 \cdot 2^{n-i} \cdot p \cdot 2^{i-1} + (p\cdot 2^{i-1})^2}
		=
		%\\=
		%\sqrt{(2^{n-i})^2 + 2 \cdot 2^{n-i} \cdot p \cdot 2^{i-1} + (p\cdot 2^{i-1})^2}
		%=
		2^{n-i} + p\cdot 2^{i-1}
		\in\mathbb{N}
		.
	\end{multline*}
\end{proof}

TODO: не знаю, доказывалось ли это ранее. Скорее всего, да, но где - неизвестно.


В отличие от диаметра и мощности, характеристика в некотором смысле локальна:
если $S\in\mathfrak{M}_n$, $H \subset S$, $H\in\mathfrak{M}_{p}$, $p<n$,
то
\begin{equation*}
	|H| < |S|, \quad \operatorname{diam} H \leq \operatorname{diam} S, \quad \operatorname{char} H = \operatorname{char} S
	.
\end{equation*}


Конструкцию $\mathfrak{M}_n$ можно распространить на пространства более высокой размерности.

%В пространствах более высокой размерности можно построить множества, аналогичные $\mathfrak{M}$.

\begin{definition}
	Пусть $m \geq 2$, $n \geq m + 1$.
	Через $\mathfrak{M}(m,n)$ будем обозначать множество таких подмножеств $m$--мерного евклидова пространства
	$M\subset\mathbb{R}^m$, что $|M| = n$, для любых $M_1,M_2 \in M$ выполнено $|M_1 - M_2| \in\mathbb{N}$
	и $M$ не содержится ни в какой $(m-1)$--мерной гиперплоскости.
\end{definition}

Более того, конструкцию $\mathfrak{M}_n$ можно распространить и на $\mathbb{Z}^m_n$~\cite{Kohnert2006IntegralPS}.
%Более того, аналогичные множества были построены в $\mathbb{Z}^m_n$~\cite{Kohnert2006IntegralPS}.

Характеристика (наряду с диаметром и мощностью) обобщается на случай, когда множество точек с целочисленными расстояниями
рассматривается в $\mathbb{R}^m$~\cite{kurz2005characteristic}.

При изучении множеств $\mathfrak{M}(m,n)$ возникает закономерный вопрос
о минимальном диаметре для заданных размерности и мощности.
Говоря формально, определим функцию
\begin{equation*}
	d(m,n) = \min_{M\in\mathfrak{M}(m,n)} \operatorname{diam} M
	.
\end{equation*}

\begin{definition}
	Множество $M\in\mathfrak{M}(m,n)$ называется оптимальным,
	если $\operatorname{diam} M = d(m,n)$.
\end{definition}

Приведём известные оценки на $d(m,n)$
(за основу взят список из~\cite{kurz2008bounds}):

%, дополненный результатом работы~\cite{nozaki2013lower}
%и результатом, полученным в данной статье.

\begin{align}
	& d(m, n - 1) \leq d(m,n) \\
	& d(n, n + 1) = 1 \\
	& d(m, n) \leq \begin{cases}
		2^{n-m+1} -2 & \mbox{~для~} n-m \equiv 0 \mod 2
		\\
		3(2^{n-m} -1) & \mbox{~для~} n-m \equiv 1 \mod 2
	\end{cases} \qquad \mbox{\cite{harborth1985diameters}} \\
	& d(2,n) \geq \frac{3}{8}n \qquad \mbox{для достаточно больших $n$ (утверждение~\ref{proposition:linear_bound_29})} \\
	& d(2,n) > 0.3457 n \qquad \mbox{для $n\geq 4$ (утверждение~\ref{proposition:linear_bound_10})} \\
	& d(m,n) \leq (n-m)^{c \log \log (n-m)} \mbox{~для некоторого $c$} \qquad \mbox{\cite{harborth1993upper}} \label{upper_bound_d_m_n_log_log}\\
	& d(m,n) > \sqrt{\frac{3}{2m}} n^{1/m} \mbox{~\cite{kanold1981punktmengen}, в частных случаях улучшена в \cite{nozaki2013lower}} \\
	& d(3,n) > \frac{1}{\sqrt{14}}n^{1/2} \mbox{~для~} n \geq 5 \quad \mbox{\cite{kanold1981punktmengen}} \\
	& d(m, 2m + 1) \leq 8 \qquad \mbox{\cite{piepmeyer1988raumliche}} \\
	& d(m, 2m + 2) \leq 13 \qquad \mbox{\cite{piepmeyer1988raumliche}} \\
	& d(m, 3m    ) \leq 109 \qquad \mbox{\cite{kemnitz1988punktmengen}} \\
	& d(m, m^2 + m) \leq 17 \qquad \mbox{\cite{kurz2008bounds}} \\
	& 3 \leq d(m, n) \leq 4 \mbox{~для~} m + 2 \leq n \leq 2m \mbox{~и~} d(m, 2m) = 4 ~ \mbox{\cite{piepmeyer1988raumliche,harborth1991point}}\\
\end{align}

Кроме того, в~\cite{kurz2008bounds} выдвинуто предположение, что $d(m - 1, n) \geq d(m, n)$.




Существуют алгоритмы вычисления $d(m,n)$ на ЭВМ, однако они требуют большого количества машинного времени,
причём требуемое время быстро растёт с ростом $m$ и $n$.
Первые несколько значений $d(2,n)$ были получены в~\cite{harborth1998integral},
следующие известные значения~--- в \cite{kurz2005characteristic,kurz2006konstruktion,kurz2008minimum,kurz2008bounds,our-mz-rus}.
Алгоритм, основанный на переборе,
может быть существенно оптимизирован с применением характеристики~\cite{kurz2005characteristic,kreisel2008heptagon}.
Однако сложность вычисления максимальной возможной мощности для заданного диаметра $d$, по оценке~\cite{kreisel2008heptagon},
составлет $O(d^3)$ (в двумерном случае).
%В (TODO: ссылка!) отмечено, что важно перепроверять вычисленные значения независимо составленными программами для ЭВМ.
Приведём некоторые известные значения:

\begin{proposition}
\label{proposition:d(2,n)}
\end{proposition}
$(d(2, n))_{n=3,...,122} = 1$, 4, 7, 8, 17, 21, 29, 40, 51, 63, 74, 91, 104, 121,
134, 153, 164,
196, 212, 228, 244, 272, 288, 319, 332, 364, 396, 437, 464, 494, 524, 553, 578, 608,
642, 667, 692, 754, 816, 897, 959, 1026, 1066, 1139, 1190,  1248, 1306, 1363, 1410,
1460, 1514, 1564, 1614, 1675, 1727, 1770, 1817, 1887, 1906, 2060, 2140, 2169,
2231, 2299, 2432, 2494, 2556, 2624, 2692, 2827, 2895, 2993, 3098, 3196, 3294,
3465, 3575, 3658, 3749, 3885, 3922, 4223, 4380, 4437, 4559, 4693, 4883,
5018, 5109, 5264, 5332, 5480, 5603, 5738, 5938, 5995, 6052,
6324, 6432, 6630, 6738, 6939, 7061, 7245, 7384, 7568, 7752, 7935, 8119, 8321,
8406, 8648, 8729, 8927, 9052, 9211, 9423, 9534, 9794, 9905
.

Для многомерного случая были вычислены~\cite{piepmeyer1988raumliche,harborth1998integral,kurz2005characteristic,kurz2006konstruktion,kurz2008bounds} следующие значения.
\begin{proposition}~
\end{proposition}
$(d(3, n))_{n=4,...,24} = 1$, 3, 4, 8, 13, 16, 17, 17, 17, 56, 65, 77, 86, 99, 112, 133, 154,
195, 212, 228, 244.

\begin{hypothesis}
	Для любого $n\geq 3$ выполнено $d(2,n) < d(2,n+1)$.
\end{hypothesis}
В пользу этой гипотезы говорят известные значения $d(2,n)$.
Для больших размерностей утверждение гипотезы неверно: так, $d(3,10) = d(3,11) = 17$.

\begin{hypothesis}
	Никакое оптимальное множество $M\in\mathfrak{M}$ не лежит на целочисленной решётке.
\end{hypothesis}
В пользу этой гипотезы говорят реузльтаты вычислений для $m=2$, $3 \leq n \leq 43$.

Примечательно, что для $m=2$, $n=4$ и $n=18$ оптимальные множества не единственны.
В~\cite{kurz2008bounds} установлено, что для $9 \leq n \leq 122$ оптимальное множество является веерным, т.е. содержится
в объединении прямой и точки (все точки, кроме одной, лежат на некоторой прямой).

\begin{hypothesis}
	Все оптимальные множества из $n$ точек на плоскости, $n\geq 9$, являются веерными.
\end{hypothesis}




При изучении множеств из $\mathfrak{M}_n$ часто возникают множества специальных видов.
Наиболее простым примером являются веерные множества.
Курц и Вассерман~\cite{kurz2008minimum} показали, что диаметр веерного множества мощности $n$
не менее, чем $n^{c \log \log n}$, где $c$~--- некоторая константа.
Это результат указывает на возможность улучшения оценки в следствии~\ref{corollary:max_points_on_line}.

В работе~\cite{kurz2008minimum} высказана
\begin{hypothesis}
	\label{hypot:log_lower_bound}
	Существует такая константа $c$,
	что для любого $n\geq 3$ и $M\in\mathfrak{M}_n$ выполнено
	$\operatorname{diam} M \geq n^{c \log \log n}$.
\end{hypothesis}

Заметим, что неравенство в гипотезе~\ref{hypot:log_lower_bound}
совпадает с верхней оценкой~\eqref{upper_bound_d_m_n_log_log} на $d(2,n)$
с точностью до константы.


Возникает закономерный вопрос: а какие бывают множества из $\mathfrak{M}_n$,
кроме веерных?

Положив $p=1$ в доказательстве теоремы~\ref{thm:power_char_exist} и добавив к построеннному множеству
точку $M_{n+2} = (0, -\sqrt{2^{n+1}})$, где $n$ нечётно,
получим множество из $n+2$ точек, из которых $n$ лежат на некоторой прямой и 2 вне её.

В~\cite{our-vvmsh-2018} приведены примеры различных множеств из $\mathfrak{M}_n$:
\begin{itemize}
	\item
		$
		\left\{
		(\pm 21/2, 0);
		(\pm 4, 3\sqrt{35}/2);
		(\pm 5/2, 0);
		(\pm 11/2, 0);
		\right\}
		$~---
		содержится в двух параллельных прямых (2 точки на одной, 6 на другой);



	\item
		$
		\left\{
		\left( {57/34} ; -{12\sqrt{35}/17}\right);
		\left( {44/17} ; {39\sqrt{35}/34}\right);
		\left( -{72/17} ;
		{15\sqrt{35}/34}\right);
		\right.$ \\ $\left.
		\left( -{175/34} ; -{24\sqrt{35}/17}\right);
		\left( -{57/34} ; {12\sqrt{35}/17}\right);
		%\right.$ \\ $\left.
		\left( \pm{17/2} ; 0\right);
		\right\}
		$~---
		содержится в двух параллельных прямых (3 точки на одной, 4 на другой);

	\item
		$
		\left\{
		(0,0);(\pm 50, 0); (\pm 14, \pm 48)
		\right\}
		$~---
		содержится в объединении окружности с её центром (7~точек);

	\item
		$
		\left\{
		\left( \pm30 ; 0\right);
		%\left( 30 ; 0\right);
		\left( \pm{15/2} ; {25\sqrt{7}/2}\right);
		%\left( {15/2} ; {25\sqrt{7}/2}\right);
		%\left( {33/2} ; {9\sqrt{7}/2}\right);
		\left( 0 ; {10\sqrt{7}/1}\right);
		\left( \pm{33/2} ; {9\sqrt{7}/2}\right);
		\right\}
		$~---
		содержится в двух прямых, пересекающихся под острым углом (7~точек).
\end{itemize}

Информация о некоторых других конструкциях может быть найдена в~\cite[\S 5.11]{brass2006research},~\cite[\S D20]{guy2013unsolved}.

TODO: спрямить ссылки!


\begin{definition}
	\cite{kurz2008minimum} Множество $M\in\mathfrak{M}(m,n)$ называется множеством полуобщего положения,
	если ни через какие $m+1$ точек из $M$ нельзя провести $(m-1)$--мерную гиперплоскость.
	Множество всех множеств полуобщего положения мощности $n$ в $\mathbb{R}^m$ будем обозначать $\overline{\mathfrak{M}}(m,n)$.
	Для краткости обозначим $\overline{\mathfrak{M}}_n=\overline{\mathfrak{M}}(2,n)$.
\end{definition}

В \cite{harborth1993upper} показано, что $\overline{\mathfrak{M}}_n\neq\varnothing$ для любого $n \geq 3$
и приведены примеры множеств из $\overline{\mathfrak{M}}_n$, содержащихся в окружности.
На эти множествах достигается верхняя оценка~\eqref{upper_bound_d_m_n_log_log} на $d(2,n)$
с некоторой константой.

Определим функцию~\cite{kurz2008minimum}
\begin{equation*}
	\overline{d}(m,n) = \min_{M\in\overline{\mathfrak{M}}(m,n)} \operatorname{diam} M
	.
\end{equation*}

В работе~\cite{kurz2008minimum} вычислено, что
\\
$\bigl(\overline{d}(2, n)\bigr)_{n=3,...,36} = 1,$
4, 8, 8, 33, 56, 56, 105, 105, 105, 532, 532, 735, 735, 735, 735,
1995, 1995, 1995, 1995, 1995, 1995, 9555, 9555, 9555, 10672,
13975, 13975, 13975, 13975, 13975, 13975, 13975, 13975.

\begin{definition}
	\cite{kurz2008minimum} Множество $M\in\overline{\mathfrak{M}}(m,n)$, $n\geq 4$ называется множеством общего положения,
	если ни через какие $m+2$ точек из $M$ нельзя провести сферу.
	Множество всех множеств общего положения мощности $n$ будем обозначать $\dot{\mathfrak{M}}(m,n)$.
	Для краткости обозначим $\dot{\mathfrak{M}}_n=\dot{\mathfrak{M}}(2,n)$.
\end{definition}

В отличие от $\mathfrak{M}_n$ и $M\in\overline{\mathfrak{M}}_n$,
существование множеств из $\dot{\mathfrak{M}}_n$ является достаточно сложным вопросом.
Множество из $\dot{\mathfrak{M}}_6$ предъявлено в~\cite{noll1989nclusters},
множество из $\dot{\mathfrak{M}}_7$~--- в~\cite{kurz2008minimum}.
Определим функцию~\cite{kurz2008minimum}
\begin{equation*}
	\dot{d}(m,n) = \min_{M\in\dot{\mathfrak{M}}(m,n)} \operatorname{diam} M
	.
\end{equation*}
Согласно~\cite{kurz2008minimum},
$\bigl(\dot{d}(2,n)\bigr)_{n=3,...,7}
= 1,$ 8, 73, 174, 22270.

Вопрос о существовании множеств из $\dot{\mathfrak{M}}_8$
остаётся открытым.

\begin{hypothesis}
	\label{hypot:general_position_nonempty}
	Для любого $n\geq 4$ непусто $\dot{\mathfrak{M}}_n$.
\end{hypothesis}


\begin{definition}
	\cite{noll1989nclusters,kurz2013constructing}
	$n$--кластером называется множество $M\in\dot{\mathfrak{M}}_n$,
	содержащееся в целочисленной решётке.
\end{definition}
Существование $n$--кластера эквивалентно существованию множества $M\in\dot{\mathfrak{M}}_n$
характеристики 1.
В~\cite{noll1989nclusters} показано существование 6--кластеров,
в~\cite{kurz2013constructing}~--- 7--кластеров.


Солимоси построил~\cite{solymosi2003note} для любого $n\geq 3$ конструкцию множества $M\in\mathfrak{M}_n$
такого, что для некоторых $M_1, M_2 \in M$ выполнено $|M_1 - M_2| = 2$.
Полученное им множество также является веерным.

Известно, что для $3\leq n \leq 5$ существуют множества из $\mathfrak{M}_n$
c минимальной длиной ребра 1.
Все известные примеры являются веерными~\cite{harborth1993upper}.
Существуют ли множества произвольной мощности с длиной ребра 1,
неизвестно, однако легко доказать следующее
\begin{proposition}
	\label{prop:edge_one}
	Для $n \geq 4$ не существует таких $M\in\overline{\mathfrak{M}}_n$,
	что для некоторых $M_1, M_2$ выполнено $|M_1 - M_2| = 1$.
\end{proposition}
\begin{proof}
	Пусть $M\in\overline{\mathfrak{M}}_n$ и для некоторых $M_1, M_2$ выполнено $|M_1 - M_2| = 1$.
	Тогда все точки множества $M$ лежат на $cr(M_1,M_2)$.
	$cr(M_1,M_2)$ есть объединение двух прямых, на каждой из которых может быть не более 2 точек в силу того, что
	$M\in\overline{\mathfrak{M}}_n$,
	откуда непосредственно следует требуемое утверждение для $n\geq 5$.

	Пусть теперь $n=4$.
	Предположим противное.
	Пусть $M=\{M_1, M_2, M_3, M_4\} \in \overline{\mathfrak{M}}_4$,
	$M_1 = (-1/2, 0)$, $M_2 = (1/2, 0)$, $O=(0,0)$.
	Тогда $M_3$ и $M_4$ лежат на оси ординат (см. выше).
	Так как $|M_3 - M_4| \in \mathbb{Z}$,
	то площадь треугольника $M_1 M_3 M_4$ есть рациональное число.
	По определению~\ref{def_char_classic} это означает, что $\operatorname{char} M = 1$.
	Тогда по определению~\ref{def_char_lattice} имеем $M_3 = (0, a/2)$.
	Понятно, что $a \neq 0$.
	Не теряя общности, положим $a\in\mathbb{N}$.
	Пусть $|M_1 - M_3| = b$.
	Применяя теорему Пифагора к треугольнику $O M_1 M_3$, получим уравнение
	\begin{equation*}
		\frac{1}{4} + \frac{a^2}{4} = b^2
		,
	\end{equation*}
	или, что то же самое,
	\begin{equation*}
		(2b)^2 - a^2 = 1
		.
	\end{equation*}
	Это уравнение не имеет решений в натуральных числах.
	Полученное противоречие завершает доказательство.
\end{proof}
Условие $n \geq 4$ в утверждении~\ref{prop:edge_one} существенно.
Для $n=3$ существует контрпример: равносторонний треугольник со стороной 1.



Если $M\in\mathfrak{M}(m,n)$ и все рёбра $M$ имеют чётную длину,
то понятно, что множество $H$, полученное из $M$ сжатием в два раза,
также будет принадлежать $\mathfrak{M}(m,n)$.
Выполняя такую операцию, мы рано или поздно получим множество из $\mathfrak{M}(m,n)$ такое,
что для некоторых $M_1,M_2\in M$ расстояние $|M_1 - M_2|$ нечётно.
Возникает закономерный вопрос: сколько может быть нечётных расстояний?

В~\cite{graham1974there} доказано, что
для существования множества $M\in\mathfrak{M}(m,m+2)$ такого, что для любых
$M_i,M_j\in M$, $i\neq j$ расстояние $|M_i-M_j|$ нечётно,
необходимо и достаточно, чтобы
$m+2\equiv 0 (\operatorname{mod} 16)$.
В частности, не существует такого множества $M\in\mathfrak{M}_4$,
что все его рёбра имеют нечётную длину.
На основе этого факта в работе~\cite{piepmeyer1996maximum} с использованием результатов теории графов доказано,
что количество рёбер нечётной длины в множестве $M\in\mathfrak{M}_n$ не превосходит
\begin{equation}
	\frac{n^2}{3} + \frac{r(r - 3)}{6}, \mbox{~где~} r = 1, 2, 3 \mbox{~и~} n \equiv r (\operatorname{mod} 3)
\end{equation}
и конструктивно показана точность этой оценки.
Множество, на котром эта оценка достигается, лежит на окружности.

Отдельные оценки количества рёбер нечётной длины для множеств из $\dot{\mathfrak{M}}_n$
авторам неизвестны.








Перспективными направлениями исследований являются, кроме непосредственно отыскания как можно более точных оценок на $d(m,n)$,
например, поиск систем из $\mathfrak{M}_n$, содержащихся в двух прямых (параллельных или пересекающихся).




С исследованием множеств $\mathfrak{M}_n$ тесно связаны ещё две гипотезы о расстояниях на плоскости.
\begin{hypothesis}[Улам, \cite{ulam1960collection}]
	\label{hypot:Ulam}
	Существует всюду плотное подмножество $U\subset\mathbb{R}^2$, такое,
	что для любых $U_1, U_2 \in U$ выполнено $|U_1 - U_2|\in\mathbb{Q}$,
	где $\mathbb{Q}$~--- множество рациональных чисел.
\end{hypothesis}
Если гипотеза~\ref{hypot:Ulam} справедлива, то для любого $n\geq 4$ множество $\dot{\mathfrak{M}}_n$ непусто
(гипотеза~\ref{hypot:general_position_nonempty}).
Неизвестно, следует ли из
гипотезы~\ref{hypot:general_position_nonempty} следует гипотеза~\ref{hypot:Ulam}.


В работах~\cite{garibaldi2005erdos,garibaldi2011erdos} исследуется связь множеств $\mathfrak{M}_n$ и
проблемы Эрдёша о числе различных расстояний~\cite{erdos1946sets}.
В работе~\cite{guth2015erdos} эта проблема в своей изначальной постановке была решена и доказана следующая
\begin{theorem}
	Пусть $G\subset{R}^2$, $|G| = n$.
	Тогда количество различных расстояний $|G_i - G_j|, G_i,G_j \in G$
	не менее $cn/\log n $.
\end{theorem}




TODO: Примеры множеств - рисунки (если будет время и место).




{\small Работа выполнена в Воронежском университете при поддержке РНФ, грант 16-11-10125.}

{
	\footnotesize
	\baselineskip=11pt
	\setlength{\itemsep}{0pt}
	\setlength{\parskip}{0pt}
	\begin{thebibliography}{99}
\bibitem{erdos1945integral}
\textsl{Erdös} \textsl{P.} Integral distances /\!/ Bulletin of the
American Mathematical Society. — 1945. — Т. 51, No 12. — С. 996.
\bibitem{anning1945integral}
\textsl{Anning} \textsl{N. H.}, \textsl{Erdös} \textsl{P.}
Integral distances /\!/ Bulletin of the American Mathematical
Society. — 1945. — Т. 51, No 8. — С. 598—600.
\bibitem{solymosi2003note}
\textsl{Solymosi} \textsl{J.} Note on integral distances /\!/
Discrete \& Computational Geometry. — 2003. — Т. 30, No 2. —
С. 337—342.
\bibitem{anning1915discussions}
\textsl{Anning} \textsl{N.} Discussions: Relating to a Geometric
Representation of Integral Solutions of Certain Quadratic
Equations /\!/ The American Mathematical Monthly. — 1915. —
Т. 22, No 9. — С. 321—321.
\bibitem{harborth1993upper}
\textsl{Harborth} \textsl{H.},
\textsl{Möller} \textsl{M.} An
diameter of integral point sets
Geometry. — 1993. — Т. 9, No 4.
\textsl{Kemnitz} \textsl{A.},
upper bound for the minimum
/\!/ Discrete \& Computational Geometry
— С. 427—432.\bibitem{piepmeyer1996maximum}
\textsl{Piepmeyer} \textsl{L.} The maximum number of odd
integral distances between points in the plane /\!/ Discrete \&
Computational Geometry. — 1996. — Т. 16, No 1. — С. 113—115.
\bibitem{kurz2008bounds}
\textsl{Kurz} \textsl{S.}, \textsl{Laue} \textsl{R.} Bounds for
the minimum diameter of integral point sets /\!/ arXiv preprint
arXiv:0804.1296. — 2008.
\bibitem{our-vvmsh-2018}
\textsl{Авдеев} \textsl{Н. Н.} Некоторые примеры систем
Эрдёша /\!/ Современные методы теории краевых задач :
материалы международной конференции «Понтрягинские
чтения — XXIX», посвященной 90-летию Владимира
Александровича Ильина (2–6 мая 2018 г.) — 2018. — С. 30—31.
\bibitem{locatelli2002packing}
\textsl{Locatelli} \textsl{M.}, \textsl{Raber} \textsl{U.} Packing
equal circles in a square: a deterministic global optimization
approach /\!/ Discrete Applied Mathematics. — 2002. — Т. 122,
No 1—3. — С. 139—166.
\bibitem{costa2013valid}
\textsl{Costa} \textsl{A.} Valid constraints for the point packing
in a square problem /\!/ Discrete Applied Mathematics. — 2013. —
Т. 161, No 18. — С. 2901—2909.
\bibitem{markot2005newverified}
\textsl{Markót} \textsl{M. C.}, \textsl{Csendes} \textsl{T.} A
new verified optimization technique for the "packing circles in
a unit square" problems /\!/ SIAM Journal on Optimization. —
2005. — Т. 16, No 1. — С. 193—219.
\bibitem{degroot1990optimal10points}
\textsl{DeGroot} \textsl{C.}, \textsl{Peikert} \textsl{R.},
\textsl{Würtz} \textsl{D.} The optimal packing of ten equal
circles in a square. — Eidgenössische Technische Hochschule
Zürich, Interdisziplinäres Projektzentrum für Supercomputing,
1990.\bibitem{kemnitz1988punktmengen}
\textsl{Kemnitz} \textsl{A.} Punktmengen mit ganzzahligen
Abständen. — 1988.
\bibitem{Bukhstab-number-theory}
\textsl{Бухштаб} \textsl{А.} Теория чисел. — 1966. — С. 384.
\bibitem{our-mkmitu-2016}
\textsl{Авдеев} \textsl{Н. Н.} Об отыскании множеств точек на
плоскости с целочисленными расстояниями /\!/ Математическое
и компьютерное моделирование, информационные технологии
управления: сб. тр. Школы для студентов, аспирантов и
молодых ученых «МКМИТУ-2016». — 2016. — С. 15—19.
\bibitem{our-ped-2017}
\textsl{Авдеев} \textsl{Н. Н.} О множествах точек на
плоскости с целочисленными расстояниями и алгоритмах их
отыскания /\!/ Некоторые вопросы анализа, алгебры, геометрии
и математического образования: материалы международной
молодежной научной школы «Актуальные направления
математического анализа и смежные вопросы». — 2017. — Т. 7,
No I. — С. 9—10.
\bibitem{polygons-on-lattices}
\textsl{Вавилов} \textsl{В.}, \textsl{Устинов}
Многоугольники на решетках. — 2006.
\textsl{А.}
\bibitem{Kohnert2006IntegralPS}
\textsl{Kohnert} \textsl{A.}, \textsl{Kurz} \textsl{S.} Integral
point sets over $Z^m_n$
/\!/ Discrete Applied Mathematics. — 2006. —
Т. 157 (2009). — С. 2105—2117.
\bibitem{kurz2005characteristic}
\textsl{Kurz} \textsl{S.} On the characteristic of integral point
sets in /\!/ arXiv preprint math/0511704. — 2005.
\bibitem{harborth1985diameters}
\textsl{Harborth} \textsl{H.}, \textsl{Kemnitz} \textsl{A.}
Diameters of integral point sets /\!/ Colloquia Mathematica
Societatis Jano Bolyai. Т. 48. — 1985.
\bibitem{kanold1981punktmengen}
\textsl{Kanold}
\textsl{H.-J.}
Über
Punktmengen
im
$k$-dimensionalen euklidischen Raum /\!/ Abh. Braunschw. Wiss.
Ges. — 1981. — Т. 32, No 55—65. — С. 252.
\bibitem{nozaki2013lower}
\textsl{Nozaki} \textsl{H.} Lower bounds for the minimum
diameter of integral point sets /\!/ Australasian Journal of
Combinatorics. — 2013. — Т. 56. — С. 139—143.
\bibitem{piepmeyer1988raumliche}
\textsl{Piepmeyer}
\textsl{L.}
Räumliche
ganzzahlige
Punktmengen: дис. . . . канд. / Piepmeyer L. — Master’s
thesis, TU Braunschweig, 1988.
\bibitem{harborth1991point}
\textsl{Harborth} \textsl{H.}, \textsl{Piepmeyer} \textsl{L.}
Point sets with small integral distances /\!/ Applied geometry and
discrete mathematics: the Victor Klee Festschrift. Т. 4. — American
Mathematical Soc., 1991.
\bibitem{harborth1998integral}
\textsl{Harborth} \textsl{H.} Integral distances in point sets /\!/
Karl der Grosse und sein Nachwirken. 1200 Jahre Kultur und
Wissenschaft in Europa: Band II, Mathematisches Wissen. —
1998. — С. 213—224.
\bibitem{kurz2006konstruktion}
\textsl{Kurz} \textsl{S.} Konstruktion und Eigenschaften
ganzzahliger Punktmengen: дис. . . . канд. / Kurz Sascha. — 2006.
\bibitem{kurz2008minimum}
\textsl{Kurz} \textsl{S.}, \textsl{Wassermann} \textsl{A.} On
the minimum diameter of plane integral point sets /\!/ arXiv
preprint arXiv:0804.1307. — 2008.
\bibitem{our-mz-rus}
\textsl{Авдеев} \textsl{Н. Н.}, \textsl{Семёнов} \textsl{Е. М.}
О множествах точек на плоскости с целочисленными
расстояниями /\!/ Математические заметки. — 2016. — Т. 100,
No 5. — С. 757—761.\bibitem{kreisel2008heptagon}
\textsl{Kreisel} \textsl{T.}, \textsl{Kurz} \textsl{S.} There are
integral heptagons, no three points on a line, no four on a circle /\!/
Discrete \& Computational Geometry. — 2008. — Т. 39, No 4. —
С. 786—790.
\bibitem{brass2006research}
\textsl{Brass} \textsl{P.}, \textsl{Moser} \textsl{W. O.},
\textsl{Pach} \textsl{J.} Research problems in discrete
geometry. — Springer Science \& Business Media, 2006.
\bibitem{guy2013unsolved}
\textsl{Guy} \textsl{R.} Unsolved problems in number theory. Т.
1. — Springer Science \& Business Media, 2013.
\bibitem{noll1989nclusters}
\textsl{Noll} \textsl{L. C.}, \textsl{Bell} \textsl{D. I.} n-clusters
for $1 < n < 7$ /\!/ Mathematics of Computation. — 1989. — Т. 53,
No 187. — С. 439—444.
\bibitem{kurz2013constructing}
Constructing 7-clusters / S. Kurz [и др.] /\!/ arXiv preprint
arXiv:1312.2318. — 2013.
\bibitem{graham1974there}
\textsl{Graham} \textsl{R.}, \textsl{Rothschild} \textsl{B.},
\textsl{Straus} \textsl{E.} Are there $n + 2$ points in $E^n$ with odd
integral distances? /\!/ The American Mathematical Monthly. —
1974. — Т. 81, No 1. — С. 21—25.
\bibitem{ulam1960collection}
\textsl{Ulam} \textsl{S. M.} A collection of mathematical
problems. Т. 8. — Interscience Publishers, 1960.
\bibitem{garibaldi2005erdos}
\textsl{Garibaldi} \textsl{J.}, \textsl{Iosevich} \textsl{A.} The
Erdos distance problem: lecture notes. — 2005.
\bibitem{garibaldi2011erdos}
\textsl{Garibaldi} \textsl{J.}, \textsl{Iosevich} \textsl{A.},
\textsl{Senger} \textsl{S.} The Erdös Distance Problem. Т. 56. —
American Mathematical Soc., 2011.\bibitem{erdos1946sets}
\textsl{Erdös} \textsl{P.} On sets of distances of n points /\!/
The American Mathematical Monthly. — 1946. — Т. 53, No 5. —
С. 248—250.
\bibitem{guth2015erdos}
\textsl{Guth} \textsl{L.}, \textsl{Katz} \textsl{N. H.} On the
Erdos distinct distances problem in the plane /\!/ Annals of
Mathematics. — 2015. — Т. 181. — С. 155—190.
\end{thebibliography}

}

%\printbibliography

%\printbibliography[env=thebibliography]


%\end{document}


%%%%%%%%%%%%%%%%%%%%%%%%%%%%%%%%%%%%%%%%%%%%%%

% Для оформления теорем, пожалуйста, используйте следующий образец
%
\begin{theorem}
	Всякое
\end{theorem}


% \textbf{Теорема 1.} \textsl{Текст формулировки теоремы.}
%
% Утверждения и следствия оформляются так же, как и теоремы.
%
%
%
% Для оформления определений, замечаний и примеров, пожалуйста, используйте
% следующий образец
%

%\begin{remark*}
%	Всякое
%\end{remark*}


\textsc{Замечание.} Текст замечания.
%
%
%
% Для оформления благодарностей, пожалуйста, используйте следующий образец
%
{\small Работа выполнена в Воронежском университете при поддержке РНФ, грант 16-11-10125.}
%
%
%
% Для оформления списка литературы, пожалуйста, используйте следующий образец
%

\par\bigskip\centerline{\bf Литература}\smallskip

 \begin{enumerate}

 \itemsep=0pt\parskip=0pt


\bib{Фамилия1 И1.~О1.}{Название книги.---Город: Издательство, год.---С.~??.}

\bib{Фамилия1 И1.~О1., Фамилия2 И2.~О2.} {Название статьи в
журнале~/\!/ Название журнала.---Год.---Т.~?, No~?.---С.~??--??.}

\bib{Фамилия1 И1.~О1., Фамилия2 И2.~О2., Фамилия3 И3.~О3.}
{Название статьи в сборнике~/\!/ Название сборника.---Город:
Издательство, год.---С.~??--??.}

 \end{enumerate}




\end{document}
