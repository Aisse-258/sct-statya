Oбозначим через $\mathfrak{M}$ множество таких подмножеств
$
	M = \{M_1, M_2, ... \}\subset \mathbb{R}^2
$,
что
$
	|M_i - M_j|\in\mathbb{N}
$
для всех $i$, $j$, где $\mathbb{N}$~--- множество натуральных чисел
и $|M_i - M_j$~--- расстояние между точками $M_i$ и $M_j$.
П. Эрдёш доказал~\cite{erdos1945integral},
что справедлива следующая

\begin{theorem}
	\label{thmErdos}
	Всякое бесконечное подмножество $M\in\mathfrak{M}$
	содержится в некоторой прямой $L\in\mathbb{R}^2$.
\end{theorem}

Теорему~\ref{thmErdos} можно доказать, основываясь на двух элементарных леммах.
Интересно, что работа~\cite{erdos1945integral} была опубликована через несколько
месяцев после совместной работы Эннинга и Эрдёша~\cite{anning1945integral},
в которой было дано другое, более сложное доказательство.

\begin{lemma}
	\label{thmQuadCurves}
	Пусть $A$ и $B$~--- две кривые второго порядка на плоскости и $A$ не является парой прямых.
	Тогда $A\cap B$ содержит не более 4 точек.
\end{lemma}

\begin{proof}
	Без ограничения общности кривая $A$ определяется уравнением
	\begin{equation*}
		x^2 = ay^2 + by +c
		,
	\end{equation*}
	а кривая $B$ определяется уравнением второго порядка общего вида.
	Система этих двух уравнений легко сводится к уравнению с одной неизвестной 4-го порядка.
	Если это уравнение не является тривиальным, то оно имеет не более 4 корней.
\end{proof}

Если $A$ и $B$~--- пары прямых $xy=0$ и $xy=x$,
то $A\cap B$ содержит ось игреков.
Поэтому предположение в лемме~\ref{thmQuadCurves} о том,
что одна из кривых не является парой прямых, существенно.

\begin{lemma}
	\label{thm:quadCurveFamily}
	Пусть $M_1, M_2 \in \mathbb{R}^2$ и
	$|M_1 - M_2| = n \in \mathbb{N}$.
	Тогда
	\begin{equation}
		\label{eq:quadCurveFamily}
		\left\{
			M: |M-M_1|\in\mathbb{N}, |M-M_2|\in\mathbb{N}
		\right\}
		=\bigcup_{i=1}^n Q_i
		,
	\end{equation}
	где $Q_1$ есть пара взаимно перпендикулярных прямых (крест)
	и $Q_2, Q_3,...,Q_n$~--- гиперболы.
\end{lemma}

\begin{proof}
	По неравенству треугольника
	\begin{equation*}
		\bigl| |M-M_1| - |M-M_2| \bigr|
		\leqslant
		|M_1 - M_2| = n
		.
	\end{equation*}
	Так как $\bigl| |M-M_1| - |M-M_2| \bigr|$
	есть целое неотрицательное число, то
	\begin{equation*}
		\bigl| |M-M_1| - |M-M_2| \bigr|
		\in\{0,1,2,...,n\}
		.
	\end{equation*}
	Ясно, что множество
	\begin{equation}
		\label{eq:centrPerp}
		\left\{ M: \bigl| |M-M_1| - |M-M_2| \bigr| = 0 \right\}
	\end{equation}
	принадлежит прямой, проходящей через середину отрезка
	$[M_1, M_2]$ перпендикулярно этому отрезку,
	\begin{equation}
		\label{eq:tails}
		\left\{ M: \bigl| |M-M_1| - |M-M_2| \bigr| = n \right\}
	\end{equation}
	содержится в прямой, проходящей через точки $M_1, M_2$,
	\begin{equation}
		\label{eq:hyperbola}
		\left\{ M: \bigl| |M-M_1| - |M-M_2| \bigr| = k \right\}
	\end{equation}
	есть гипербола для любого $k=1,2,...,n-1$.
	Обозначая объединение множеств \eqref{eq:centrPerp} и \eqref {eq:tails}
	через $Q_1$,
	а множества \eqref{eq:hyperbola} через $Q_i$,
	мы получаем требуемое представление.
	Таким образом, \eqref{eq:quadCurveFamily}
	есть объединение $n$ кривых второго порядка.
\end{proof}

\begin{proof}[Доказательство теоремы \ref{thmErdos}]
	Предположим противное,
	т.е. $\{M_1, M_2, ... \}\in\mathfrak{M}$
	и точки $M_1$, $M_2$, $M_3$ не лежат на одной прямой.
	Применяя лемму~\ref{thm:quadCurveFamily}
	к паре точек $M_1, M_2$ и к паре точек $M_1, M_3$,
	найдём такие кривые 2-го порядка
	$Q_1, Q_2,...,.Q_n$ и $R_1, R_2,...,.R_m$, что
	\begin{equation*}
		\left\{
			M: |M-M_1|\in\mathbb{N}, |M-M_2|\in\mathbb{N}
		\right\}
		=\bigcup_{i=1}^n Q_i
		,
	\end{equation*}
	\begin{equation*}
		\left\{
			M: |M-M_1|\in\mathbb{N}, |M-M_3|\in\mathbb{N}
		\right\}
		=\bigcup_{j=1}^m R_j
		.
	\end{equation*}
	Отсюда
	\begin{multline*}
		\left\{
			M:
			|M-M_1|\in\mathbb{N},
			|M-M_2|\in\mathbb{N},
			|M-M_3|\in\mathbb{N}
		\right\}
		\subset
		\\ \subset
		\left(\bigcup_{i=1}^n Q_i\right)
		\cap
		\left(\bigcup_{j=1}^m R_j\right)
		=
		\bigcup_{i=1}^n \bigcup_{j=1}^m \left( Q_i \cap R_i \right)
		.
	\end{multline*}
	Из предположения о том, что $M_1$, $M_2$, $M_3$
	не лежат на одной прямой, вытекает,
	что $Q_1\cap R_1$ не содержит прямую.
	Поэтому $|Q_1\cap R_1|\leqslant 4$,
	и, следовательно, в силу леммы~\ref{thmQuadCurves}
	$|Q_i\cap R_j|\leqslant 4$
	для всех $i=1,2,...,n$, $j=1,2,...,m$.
	Отсюда
	\begin{equation*}
		|M| \leqslant 4nm
		,
	\end{equation*}
	т.е. $M$ конечно,
	что проиворечит предположению о бесконечности $M$.
	Полученное проиворечие доказывает, что все точки $M$
	принадлежат некоторой прямой $L$.
	В качестве $L$ можно взять прямую,
	проходящую через пару любых точек из $M$.
\end{proof}

Предположение о бесконечности $M$ в теореме~\ref{thmErdos} существенно.
Для заданного $n\in\mathbb{N}$ обозначим через $\mathfrak{M}_n$
множество таких $M\in\mathfrak{M}$, что
$|M|=n$ и $M \not\subset L$ для любой прямой $L \subset\mathbb{R}^2$.

\begin{theorem}
	\label{thm:power_exist}
	Для любого $n\in\mathbb{N}$ выполнено $\mathfrak{M}_n\neq\varnothing$.
\end{theorem}

\begin{proof}
	Из теоремы Пифагора следует, что треугольник со сторонами
	$2k+1$, $2k^2+2k$, $2k^2+2k+1$
	является прямоугольным для любого $k\in\mathbb{N}$.
	Выпишем стороны первых трёх треугольников,
	а затем стороны подобных им треугольников с совпадающей первой стороной:
	\begin{equation}
		\label{eq:PifagTriangles}
		\begin{array}{ll}
			3,  4,  5 & \quad 105, 140, 175 \\
			5, 12, 13 & \quad 105, 252, 273 \\
			7, 24, 25 & \quad 105, 360, 375.
		\end{array}
	\end{equation}
	Ясно, точки $(0,0)$, $(0,105)$, $(140,0)$, $(252,0)$, $(360,0)$
	образуют множество из $\mathfrak{M}_5$.
	Очевидно, если взять не 3, а $n$ точек, и затем найти подобные
	прямоугольные треугольники с совпадающей первой стороной,
	то можно построить множество из $\mathfrak{M}_{n+2}$.
\end{proof}

Другой пример множества из $\mathfrak{M}_{n}$ для заданного $n$
можно найти в доказательстве теоремы~\ref{thm:power_char_exist},
уверждение которой усиливает утверждение теоремы~\ref{thm:power_exist}.
