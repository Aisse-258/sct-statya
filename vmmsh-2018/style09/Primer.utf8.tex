\input style_konf.utf8.tex


\UDC{???}


\Title{НАЗВАНИЕ СТАТЬИ}




 \Author{ФИО}

\Abstract{Текст аннотации. \par}

{\hangindent17pt\hangafter=0\noindent\footnotesize{\bf Ключевые
 слова:} ключевое слово 1, ключевое слово 2, \ldots\par}



\section*{1. Название параграфа}



\Teorema{ 1}Текст теоремы. \Endproc

\beginproof~Текст доказательства.~\endproof


\Lema{ 1}Текст леммы.\Endproc




\begin{proof}
Очевидно.
\end{proof}


\begin{equation}
	\label{eq:main}
	\sin^2 t + \cos^2 t = 1
\end{equation}

\begin{equation}
	\label{eq:aux}
	\sin^2 t = 1 - \cos^2 t
\end{equation}











\section*{Литература}

\begin{enumerate}
 {\footnotesize
 \itemsep=0pt \parskip=0pt
 \leftskip=-10pt

%+
 \bib{Полянин А.~Д.}{Справочник по линейным уравнениям математической фи\-зи\-ки.\,---\,М.: Физматлит, 2001.\,---\,576~c.}


%+
 \bib{Киприянов И.~А., Катрахов В.~В., Ляпин В.~М.} {О краевых задачах в области общего вида для сингулярных
 параболических систем уравнений~/\!/ Докл. АН СССР.\,---\,1976.\,---\,Т.~230, No~6.\,---\,С.~1271--1274.}



%+
 \bib{Ильин В.~А., Позняк Э.~Г.}{Основы математического анализа.  Ч.~1: Учеб. для вузов~/ 7-е изд.\,---\,М.:
 Физматлит, 2005.\,---\,648~с.}

}
 \end{enumerate}

 \Adress{\textsc{ФИО}\\
Место работы\\
 %старший научный сотрудник (докторант),\\
 Адрес (страна, индекс, город, улица, номер дома)\\
 E-mail}

 \bigskip


 \begin{center}
 НАЗВАНИЕ СТАТЬИ НА АНГЛИЙСКОМ ЯЗЫКЕ
 \end{center}
 \begin{center}
  ФИО на английском языке
 \end{center}

\Abstract{Аннотация на английском языке. \par}

\vspace*{-4pt}

{\hangindent17pt\hangafter=0\noindent\footnotesize {\bf Key
 words:}~Ключевые слова на английском языке.\par}



\Title{НАЗВАНИЕ СТАТЬИ}





 \Author{ФИО}

\Abstract{Текст аннотации. \par}

{\hangindent17pt\hangafter=0\noindent\footnotesize{\bf Ключевые
 слова:} ключевое слово 1, ключевое слово 2, \ldots\par}



\section*{1. Название параграфа}



\Teorema{ 1}Текст теоремы. \Endproc

\beginproof~Текст доказательства.~\endproof


\Lema{ 1}Текст леммы.\Endproc




\begin{proof}
Очевидно.
\end{proof}

\begin{equation}
	\label{eq:aux}
	\sin^2 t = 1 - \cos^2 t
\end{equation}

\begin{equation}
	\label{eq:main}
	\sin^2 t + \cos^2 t = 1
\end{equation}







\end{document}
