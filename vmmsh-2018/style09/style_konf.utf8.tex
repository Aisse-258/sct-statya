 \documentclass[a5paper,10pt]{book}%,landscape

\usepackage[utf8]{inputenc}

 \usepackage{amsmath, amsfonts, amssymb}
\usepackage[english,russian]{babel}
\usepackage{latexsym}
\usepackage{mathrsfs}
\usepackage{calrsfs}  %красивая каллиграфия
\usepackage{bbm}
\usepackage{bbold}
\usepackage{makeidx}
\usepackage{graphicx}
\usepackage{epsfig}
\usepackage{color}
\usepackage{amscd}

%\usepackage{pst-node}
% \documentclass[11pt]{book}
%\usepackage{amsmath, amsfonts, amssymb, bbm}
%\usepackage[russian]{babel}
%\usepackage{epsfig, epic}
%%\usepackage{eepic, rotating}
%\usepackage{graphicx}
%\usepackage{wrapfig}
%\usepackage{calrsfs}
%%\usepackage {curves, eepic, epic}
%\usepackage{amsxtra,latexsym, amscd,amsthm}
%\input diagxy
%\usepackage{mathrsfs}
%\usepackage{bbm}


%Цитаты и библиография с автоматической нумерацией
\usepackage{cite}
\makeatletter
\renewcommand{\@biblabel}[1]{#1)} % Заменяем библиографию с квадратных скобок на скобку
\makeatother

% Заголовок библиографии
\newcommand{\vmmshbibcaption}[1]{
	\par\bigskip\centerline{\bf #1}\smallskip
}
\usepackage{etoolbox}
\patchcmd{\thebibliography}
  {\chapter*}
  {\vmmshbibcaption}
  {}
  {}

% Библиография не влияет на колонтитулы
\makeatletter%{@}
\newcommand{\vmmshnothing}[2]{}
\patchcmd{\thebibliography}
  {\@mkboth}
  {\vmmshnothing}
  {}
  {}
\makeatother%{@}

% Шрифты и междустрочные интервалы в библиографии
% Вероятно, можнодостичь той же цели несколько изящнее
\AtBeginEnvironment{thebibliography}{
  \footnotesize
  \baselineskip=7pt
  \setlength{\itemsep}{0pt}
  \setlength{\parskip}{0pt}
}
\patchcmd{\thebibliography}
  {\settowidth}
  {
    \footnotesize
    \setlength{\itemsep}{0pt}
    \setlength{\parskip}{0pt}
    \baselineskip=0pt
    \settowidth
  }
  {}{}

% Английский заголовок библиографии
\addto\captionsenglish{
	\renewcommand\bibname{References}
}





%Теоремы с автоматической нумерацией
\usepackage{amsthm}


 %\usepackage[mirror]{crop}% зеркальное изображение
 \input diagxy
 \sloppy
 \textheight = 165mm
 \textwidth = 110mm
 \topmargin = -7mm
 \oddsidemargin=-10mm
 \evensidemargin=-10mm
 \makeindex
 \newcommand{\indexentry}[2]{\noindent\hangindent=20pt\hangafter=1 #1, #2\par}

% \NeedsTeXFormat{LaTeX2e} \ProvidesPackage{bbold}[1994/04/06 Bbold symbol package]
% \newcommand{\bbfamily}{\fontencoding{U}\fontfamily{bbold}\selectfont}
% \newcommand{\textbb}[1]{{\bbfamily#1}}
% %\DeclareMathAlphabet{\mathbbt}{U}{bbold}{m}{n}



%Блок, отвечающий за локализацию меток в пределах одной статьи
\makeatletter
\AtBeginDocument{%
  \let\origlabel@in@display\label@in@display
  \let\origref\ref \let\origpageref\pageref \let\origlabel\label
  \newcommand\locallabels[1]{%
    \renewcommand\label[1]{\origlabel{#1##1}}% ← notice the change here:
    \renewcommand\ref[1]{\origref{#1##1}}% ##1#1, not #1##1
    \renewcommand\pageref[1]{\origpageref{#1##1}}%
    \renewcommand\label@in@display[1]{\origlabel@in@display{#1##1}}%
  }}
\makeatother

\newcounter{locallabelscnt}
\newcommand{\newlabelscope}{
	\locallabels{\arabic{locallabelscnt}}
	\addtocounter{locallabelscnt}{1}
}

 \begin{document}

 \makeatletter%{@}

 \newcommand{\Author}[2]{\begin{center}\textbf{\large #1} \end{center} \medskip
  \renewcommand{\@evenhead}
  {\raisebox{0pt}[\headheight][0pt]%
 {\vbox{\hbox to\textwidth{\thepage \hfill\strut {\sl #2}}\hrule}}}}

 %\newcommand{\shorttitle}[1]{\renewcommand{\@oddhead}{\raisebox{1mm}[\headheight][0pt]%
%                    {\vbox{\hbox to\textwidth{\strut \hfill{\small #1}\hfill\thepage}\hrule}}} }

 \newcommand{\shorttitle}[1]{\renewcommand{\@oddhead}{\raisebox{0pt}[\headheight][0pt]%
 {\vbox{\hbox to\textwidth{\strut {\sl #1}%\rightmark
 \hfill\thepage}\hrule}}} }

 \headsep=2mm

  %\newcommand{\l@abcd}[2]{\centering{\par\medskip\normalsize\par\smallskip\noindent\hangindent5pt\hangafter=1{\bf #1}}}\par\medskip}
\newcommand{\l@abcd}[2]{\begin{center}\par\normalsize\par\noindent\hangindent5pt\hangafter=1{\bf
#1}\end{center}\par\smallskip}

 \renewcommand{\section}{\@startsection{section}{1}{\parindent}
                        {3ex plus 1ex minus .2ex}{2ex plus .2ex}{\bf\centering}}
 \renewcommand{\l@section}[2]{\small\leftskip0pt\par\noindent\hangindent27pt\hangafter=1{\qquad#1} \dotfill~~#2\par}

 \renewcommand{\l@part}[2]{\normalsize\leftskip0pt\par\noindent
 \hangindent17pt\hangafter=1{\bf #1} \normalsize\dotfill~~#2\par}

 \newcommand{\Adress}[1]{\par\bigskip\baselineskip=11pt\hangindent17pt\hangafter=0\noindent{\footnotesize#1\par}\normalsize}
 \newcommand{\UDC}[1]{\begingroup\newpage\thispagestyle{empty}\begin{flushleft}УДК #1\end{flushleft}}
 \newcommand{\Abstract}[1]{\hangindent17pt\hangafter=0\noindent{\footnotesize#1}\bigskip\par\medskip}

 \renewcommand{\@cite}[2]{{[#1\if@tempswa, #2\fi]}}


 \makeatother%@


%Автоматический сброс счётчиков по заголовку

\newcommand{\prepareThesis}{
	\setcounter{equation}{0}
	\setcounter{section}{0}
	\setcounter{figure}{0}
	\setcounter{table}{0}
	\setcounter{footnote}{0}
	%\setcounter{theorem}{0}
	\renewcommand{\theenumi}{\arabic{enumi}}
	\renewcommand{\labelenumi}{\theenumi)} % Yes, we have encountered such problems!
	\selectlanguage{russian} % To avoid problems if the language was switched to English (e.g. in the bibliography).
	\newlabelscope % Изоляция меток
}


 \newcommand{\Title}[1]{\begin{center}\large\uppercase{#1}\end{center}\prepareThesis\par}

 %%%%%%%%%%%%%%%%%%%%%%%%%%%%%%%

%\renewcommand{\@cite}[2]{{#1\if@tempswa, #2\fi}}

 \def\subsec#1{\smallskip\textbf{#1}}
 \newcommand{\bib}[2]{{\leftskip-10pt\baselineskip=11pt\footnotesize\item{}\textsl{#1}~#2\par}}

\newcommand{\Endproc}{\rm}
\newcommand{\Proclaim}[1]{\smallskip{\textbf{#1\/~}}\sl}
\newcommand{\proclaim}[1]{{\textbf{#1\/~}}\sl}
\newcommand{\Teorema}[1]{\smallskip{\textbf{Теорема#1.\/~}}\sl}
\newcommand{\teorema}[1]{{\textbf{Теорема#1.\/~}}\sl}
\newcommand{\Sledstvie}[1]{\smallskip{\textbf{Следствие#1.\/~}}\sl}
\newcommand{\sledstvie}[1]{{\textbf{Следствие#1.\/~}}\sl}
\newcommand{\Lema}[1]{\smallskip\textbf{{Лемма#1}.\/~}\sl}
\newcommand{\lema}[1]{\textbf{{Лемма#1}.\/~}\sl}
\newcommand{\Predl}[1]{\smallskip\textbf{{Предложение#1}.\/~}\sl}
\newcommand{\predl}[1]{\textbf{{Предложение#1}.\/~}\sl}
\newcommand{\Utv}[1]{\smallskip\textbf{{Утверждение#1}.\/~}\sl}
\newcommand{\utv}[1]{\textbf{{Утверждение#1}.\/~}\sl}
%-----------------
\newcommand{\Def}[1]{\smallskip{\sc Определение#1.}}
\newcommand{\deff}[1]{{\sc Определение#1.}}
\newcommand{\Zam}[1]{\smallskip{\sc Замечание#1.}}
\newcommand{\zam}[1]{{\sc Замечание#1.}}
\newcommand{\Primer}[1]{\smallskip{\sc Пример#1.}}
\newcommand{\primer}[1]{{\sc Пример#1.}}
\newcommand{\Zadacha}[1]{\smallskip{\sc Задача#1.}}
\newcommand{\zadacha}[1]{{\sc Задача#1.}}
%
%%%%%%%%%%%%%%%%%%%%%%%%%%%%%%%%%%%%%%%%%%%%%%%%%%%%%%%%
%%%%%%%%%     English
\newcommand{\Lemma}[1]{\smallskip\textbf{{Lemma#1}.\/~}\sl}
\newcommand{\lemma}[1]{\textbf{{Lemma#1}.\/~}\sl}
\newcommand{\Theorem}[1]{\smallskip\textbf{{Theorem#1}.\/ }\sl}
\newcommand{\theorem}[1]{{\textbf{Theorem#1.\/~}}\sl}
\newcommand{\Corollary}[1]{\smallskip\textbf{{Corollary#1}.\/~}\sl}
\newcommand{\corollary}[1]{\textbf{{Corollary#1}.\/~}\sl}
\newcommand{\Proposition}[1]{\smallskip\textbf{{Proposition#1}.\/~}\sl}
\newcommand{\proposition}[1]{\textbf{{Proposition#1}.\/~}\sl}
\newcommand{\Crit}[1]{\smallskip\textbf{{Criterion#1}.\/ }\sl}
%-------------
\newcommand{\defin}[1]{{\sc Definition#1.}}
\newcommand{\Definition}[1]{\smallskip{\sc Definition#1.}}
\newcommand{\Remark}[1]{\smallskip{\sc Remark#1.}~}
\newcommand{\Ex}[1]{\smallskip{\sc Example#1.}}
\newcommand{\Problem}[1]{{\sc Problem#1:}}
\newcommand{\Note}[1]{{\sc Note#1.}}
%%%%%%%%%%%%%%%%%%%%%%%%%%%%%%%%%%%%%%%%%%%%%%%%%%%%%%%%%%%%%%


\def\beginproof{\par\mbox{$\vartriangleleft$}}
\def\endproof{\text{$\vartriangleright$}\smallskip}
\def\Endproof{\text{$\vartriangleright$}}

%Переопределяем стандартное окружение из amsthm
\let\proof\beginproof

%%%%%%%%%%%%%    неравенства
\renewcommand{\le}{\leqslant}
\renewcommand{\leq}{\leqslant}
\renewcommand{\ge}{\geqslant}
\renewcommand{\geq}{\geqslant}
\renewcommand{\to}{\rightarrow}

%%%%%%%%%%%%%% o-lim  и др.
\def\osum{\mathop{o\text{\/-}\!\sum}}  %\def\osum{o\text{-}\!\sum}
\def\bosum{bo\text{-}\!\sum}
\def\olim{\mathop{o\text{-}{\fam0 lim}}} %\def\olim{\mathop{o\text{-}\!\lim}}
\def\bolim{\mathop{bo\text{-}\!\lim}}
\def\rlim{\mathop{r\text{-}{\fam0 lim}}}

%%%%%%%%%%%%%%%%%%%%%%%%%%%%%%%%%%%%%%%%%%%

%Vogt
\newcommand{\R}{\mathbb{R}}        % real field  +++
\newcommand{\N}{\mathbb{N}}        % natural numbers +++
%\newcommand{\F}[1][]{Fr\'echet space#1}            % Frechet space(s) +++
\newcommand{\FH}[1][]{Fr\'echet--Hilbert space#1}  % Frechet-Hilbertspace(s) +++
\newcommand{\FHS}[1][]{Fr\'echet--Hilbert--Schwartz space#1}   % Frechet-Hilbertspace(s) +++
%\newcommand{\B}[1][]{Banach space#1}               % Banach space(s) +++
\newcommand{\Hb}[1][]{Hilbert space#1}             % Hilbert space(s) +++
\newcommand{\DN}{{\rm(DN)}}                        %+
\newcommand{\id}{\operatorname{id}}                %+
\newcommand{\Span}{\operatorname{span}}            %+
\newcommand{\eps}{\varepsilon}                     %+
\newcommand{\vp}{\varphi}                          %+
\newcommand{\Om}{{\rm (}\Omega{\rm )}}             %+
\newcommand{\Lam}{\Lambda}                         %+
\newcommand{\Lama}{\Lambda_\infty(\alpha)}         %+
\newcommand{\Lamb}{\Lambda_\infty(\beta)}          %+
\newcommand{\Lamg}{\Lambda_\infty(\gamma)}         %+
\newcommand{\lama}{\Lama}                          %+
\newcommand{\tP}{\widetilde{P}}                    %+
\newcommand{\tQ}{\widetilde{Q}}                    %+
\newcommand{\io}{\iota}                            %+
\newcommand{\To}{\longrightarrow}                  %+
%%%%%%%%%%%%%%%%%%%%%%%%%%%%%%%%%%%%%%%%%%%%%%%%%%%%%%%%%%%%%%

%%%%%%%%%%%%%%%%% Операторы
\def\card{\mathop{\fam0 card}}
\def\co{\mathop{\fam0 co}}
\def\Dom{\mathop{\fam0 Dom}}
\def\dom{\mathop{\fam0 dom}}
\def\End{\mathop{\fam0 End}}
\def\Fin{\mathop{\fam0 fin}\nolimits}
\def\fin{\mathop{\fam0 fin}\nolimits}
\def\Id{\mathop{\fam0 Id}}
\def\Im{\mathop{\fam0 Im}}
\def\ind{\mathop{\fam0 ind}}
\def\Ker{\mathop{\fam0 Ker}}
\def\loc{\mathop{\fam0 loc}}
\def\const{\mathop{\fam0 const}}
\def\Exp{\mathop{\fam0 Exp}}
\def\grad{\mathop{\fam0 grad}}
\def\proj{\mathop{\fam0 proj}}
\def\im{\mathop{\fam0 im}}
\def\rot{\mathop{\fam0 rot}}
\def\Lip{\mathop{\fam0 Lip}}
\def\mod{\mathop{\fam0 mod}}
\def\On{\mathop{\fam0 On}}
\def\Orth{\mathop{\fam0 Orth}}
\def\rank{\mathop{\fam0 rank}}
\def\sa{\mathop{\fam0 sa}}
\def\SC{\mathop{\fam0 SC}}
\def\sign{\mathop{\fam0 sign}}
\def\St{\mathop{\fam0 St}}
\def\supp{\mathop{\fam0 supp}}
\def\Re{\mathop{\fam0 Re}}
\def\Div{\mathop{\fam0 div}}
\def\tv@rt{{\vert\mkern-2mu\vert\mkern-2mu\vert}}
\def\tvert#1\tvert{\mathopen\tv@rt#1\mathclose\tv@rt}
%%%тройные нормы
\def\tnorm{|\hspace{-1pt}|\hspace{-1pt}|}                        %|||
\def\bigtnorm{\big|\hspace{-1.5pt}\big|\hspace{-1.5pt}\big|}     %\big|||
\def\biggtnorm{\bigg|\hspace{-1.5pt}\bigg|\hspace{-1.5pt}\bigg|} %\bigg|||


\def\bnorml{\mathopen{\kern1pt \vrule height9pt depth2.5pt width1pt\kern1.5pt}}
\def\bnormr{\mathclose{\kern1.5pt\vrule height9pt depth2.5pt width1pt\kern1pt}}

\let\mathbbt\mathbb
\let\mathbbm\mathbb


%%%%%%%%% математические шрифты
\def\Cal#1{\mathcal{#1}}   %калиграфия (подключить пакет calrsfs)
\def\goth#1{\mathfrak{#1}} %готика
%\mathbbm --- полые буквы (\mathbb другой полый шрифт)
\def\bbold#1{{\mathbbt #1}} %полые цифры

\newcommand{\shortpage}{\enlargethispage{-\baselineskip}}
 \newcommand{\Signatura}[3]{\begin{picture}(0,0)\put(#2,#3){\makebox{{#1}}}\end{picture}}
