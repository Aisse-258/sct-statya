В пространствах более высокой размерности можно построить множества, аналогичные $\mathfrak{M}$.

\begin{definition}
	Пусть $m \geq 2$, $n \geq m + 1$.
	Через $\mathfrak{M}(m,n)$ будем обозначать множество таких подмножеств $m$--мерного евклидова пространства
	$M\subset\mathbb{R}^m$, что $|M| = n$, для любых $M_1,M_2 \in M$ выполнено $|M_1 - M_2| \in\mathbb{N}$
	и $M$ не содержится ни в какой $(m-1)$--мерной гиперплоскости.
\end{definition}

Характеристика (наряду с диаметром и мощностью) обобщается на случай, когда множество точек с целочисленными расстояниями
рассмативается в $\mathbb{R}^k$~\cite{kurz2005characteristic}.

При изучении множеств $\mathfrak{M}(m,n)$ возникает закономерный вопрос
о минимальной диаметре для заданных размерности и мощности.
Говоря формально, определим функцию
\begin{equation*}
	d(m,n) = \min_{M\in\mathfrak{M}(m,n)} \operatorname{diam} M
	.
\end{equation*}

Имеют место быть нижеследующие оценки на $d(m,n)$
(за основу взят список из~\cite{kurz2008bounds}):

%, дополненный результатом работы~\cite{nozaki2013lower}
%и результатом, полученным в данной статье.

\begin{align}
	& d(m, n - 1) \leq d(m,n) \\
	& d(n, n + 1) = 1 \\
	& d(m, n) \leq \begin{cases}
		2^{n-m+1} -2 & \mbox{для} n-m \equiv 0 \mod 2
		\\
		3(2^{n-m} -1) & \mbox{для} n-m \equiv 1 \mod 2
	\end{cases} \qquad \mbox{\cite{harborth1985diameters}} \\
	& d(2,n) \geq cn \qquad \mbox{TODO: вставить константу} \\
	& d(m,n) \leq (n-m)^{c \log \log (n-m)} \mbox{~для некоторого $c$} \qquad \mbox{\cite{harborth1993upper}} \\
	& d(m,n) > \sqrt{\frac{3}{2m}} n^{1/m} \mbox{\cite{kanold1981punktmengen},в частных случаях улучшена в \cite{nozaki2013lower}} \\
	& d(3,n) > \frac{1}{\sqrt{14}}n^{1/2} \mbox{~для~} n \geq 5 \quad \mbox{\cite{kanold1981punktmengen}} \\
	& d(m, 2m + 1) \leq 8 \qquad \mbox{\cite{piepmeyer1988raumliche}} \\
	& d(m, 2m + 2) \leq 13 \qquad \mbox{\cite{piepmeyer1988raumliche}} \\
	& d(m, 3m    ) \leq 109 \qquad \mbox{\cite{kemnitz1988punktmengen}} \\
	& d(m, m^2 + m) \leq 17 \qquad \mbox{\cite{kurz2008bounds}} \\
	& 3 \leq d(m, n) \leq 4 \mbox{~для~} m + 2 \leq n \leq 2m \mbox{~и~} d(m, 2m) = 4 ~ \mbox{\cite{piepmeyer1988raumliche,harborth1991point}}\\
\end{align}

Кроме того, Курц~\cite{kurz2008bounds} выдвинул предположение, что $d(m - 1, n) \geq d(m, n)$.




Существуют алгоритмы вычисления $d(m,n)$ на ЭВМ, однако они требуют большого количества машинного времени,
причём требуемое время быстро растёт с ростом $m$ и $n$.
Первые несколько значений $d(2,n)$ были получены в~\cite{harborth1998integral},
следующие известные значения~--- в \cite{kurz2005characteristic,kurz2006konstruktion,kurz2008minimum,kurz2008bounds,our-mz-rus}.
Курцу удалось существенно оптимизировать алгоритм, основанный на переборе, применяя характеристику~\cite{kurz2005characteristic,kreisel2008heptagon}.
Однако сложность вычисления максимальной возможной мощности для заданного диаметра $d$, по оценке Курца~\cite{kreisel2008heptagon},
составлет $O(d^3)$ (в двумерном случае).
Курц (TODO: ссылка!) отмечает, что важно перепроверять вычисленные значения независимо составленными программами для ЭВМ.
Приведём некоторые известные значения:
\\
$
(d(2, n))_{n=3,...,122} = 1, 4, 7, 8, 17, 21, 29, 40, 51, 63, 74, 91, 104, 121,
134, $\\$153, 164,
196, 212, 228, 244, 272, 288, 319, 332, 364, 396, 437, 464, 494, 524,$\\$ 553, 578, 608,
642, 667, 692, 754, 816, 897, 959, 1026, 1066, 1139, 1190, $\\$ 1248, 1306, 1363, 1410,
1460, 1514, 1564, 1614, 1675, 1727, 1770, 1817, $\\$1887, 1906, 2060, 2140, 2169,
2231, 2299, 2432, 2494, 2556, 2624, 2692, $\\$2827, 2895, 2993, 3098, 3196, 3294,
3465, 3575, 3658, 3749, 3885, 3922, $\\$4223, 4380, 4437, 4559, 4693, 4883,
5018, 5109, 5264, 5332, 5480, 5603, $\\$5738, 5938, 5995, 6052,
6324, 6432, 6630, 6738, 6939, 7061, 7245, 7384, $\\$7568, 7752, 7935, 8119, 8321,
8406, 8648, 8729, 8927, 9052, 9211, 9423, $\\$9534, 9794, 9905
$.

Примечательно, что для $n=4$ и $n=18$ оптимальные множества не единственны.
Курц выяснил, что для $9 \leq n \leq 122$ опимальное множество содержится
в объединении прямой и точки (т.е. все точки, кроме одной, лежат на некоторой прямой).



План

2. Несколько слов про вычисление. Обзор результатов Харборса, Курца и наших.
Независимые эксперименты - это хорошо. Курц и характеристика. Что-нибудь про сложность вычислений.
Неединственные оптимальные системы. Про веерность со ссылкой на Курца.


1. Особые случаи

1.1. Нечётные расстояния (3 работы)

1.2. Круговые системы (1 работа) - оценка сверху

1.3. Веерные системы (1 работа) - и оценка на них снизу

1.4. Системы с миниметром 2 (Солимоси)

1.5. Системы общего положения (гипотеза Эрдёша, работа Курца)

1.6. 7-кластеры (Курц)



3. Связанные проблемы: гипотеза Эрдёша о числе расстояний, гипотеза Улама--Эрдёша.
Ссылка на 2 работы Гарибальди. Ссылка на доказательство гипотезы Эрдёша о числе расстояний.


