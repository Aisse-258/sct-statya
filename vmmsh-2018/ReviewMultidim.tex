В пространствах более высокой размерности можно построить множества, аналогичные $\mathfrak{M}$.

\begin{definition}
	Пусть $m \geq 2$, $n \geq m + 1$.
	Через $\mathfrak{M}(m,n)$ будем обозначать множество таких подмножеств $m$--мерного евклидова пространства
	$M\subset\mathbb{R}^m$, что $|M| = n$, для любых $M_1,M_2 \in M$ выполнено $|M_1 - M_2| \in\mathbb{N}$
	и $M$ не содержится ни в какой $(m-1)$--мерной гиперплоскости.
\end{definition}

Характеристика (наряду с диаметром и мощностью) обобщается на случай, когда множество точек с целочисленными расстояниями
рассмативается в $\mathbb{R}^k$~\cite{kurz2005characteristic}.

При изучении множеств $\mathfrak{M}(m,n)$ возникает закономерный вопрос
о минимальном диаметре для заданных размерности и мощности.
Говоря формально, определим функцию
\begin{equation*}
	d(m,n) = \min_{M\in\mathfrak{M}(m,n)} \operatorname{diam} M
	.
\end{equation*}

Приведём известные оценки на $d(m,n)$
(за основу взят список из~\cite{kurz2008bounds}):

%, дополненный результатом работы~\cite{nozaki2013lower}
%и результатом, полученным в данной статье.

\begin{align}
	& d(m, n - 1) \leq d(m,n) \\
	& d(n, n + 1) = 1 \\
	& d(m, n) \leq \begin{cases}
		2^{n-m+1} -2 & \mbox{для} n-m \equiv 0 \mod 2
		\\
		3(2^{n-m} -1) & \mbox{для} n-m \equiv 1 \mod 2
	\end{cases} \qquad \mbox{\cite{harborth1985diameters}} \\
	& d(2,n) \geq cn \qquad \mbox{TODO: вставить константу} \\
	& d(m,n) \leq (n-m)^{c \log \log (n-m)} \mbox{~для некоторого $c$} \qquad \mbox{\cite{harborth1993upper}} \label{upper_bound_d_m_n_log_log}\\
	& d(m,n) > \sqrt{\frac{3}{2m}} n^{1/m} \mbox{\cite{kanold1981punktmengen},в частных случаях улучшена в \cite{nozaki2013lower}} \\
	& d(3,n) > \frac{1}{\sqrt{14}}n^{1/2} \mbox{~для~} n \geq 5 \quad \mbox{\cite{kanold1981punktmengen}} \\
	& d(m, 2m + 1) \leq 8 \qquad \mbox{\cite{piepmeyer1988raumliche}} \\
	& d(m, 2m + 2) \leq 13 \qquad \mbox{\cite{piepmeyer1988raumliche}} \\
	& d(m, 3m    ) \leq 109 \qquad \mbox{\cite{kemnitz1988punktmengen}} \\
	& d(m, m^2 + m) \leq 17 \qquad \mbox{\cite{kurz2008bounds}} \\
	& 3 \leq d(m, n) \leq 4 \mbox{~для~} m + 2 \leq n \leq 2m \mbox{~и~} d(m, 2m) = 4 ~ \mbox{\cite{piepmeyer1988raumliche,harborth1991point}}\\
\end{align}

Кроме того, Курц~\cite{kurz2008bounds} выдвинул предположение, что $d(m - 1, n) \geq d(m, n)$.




Существуют алгоритмы вычисления $d(m,n)$ на ЭВМ, однако они требуют большого количества машинного времени,
причём требуемое время быстро растёт с ростом $m$ и $n$.
Первые несколько значений $d(2,n)$ были получены в~\cite{harborth1998integral},
следующие известные значения~--- в \cite{kurz2005characteristic,kurz2006konstruktion,kurz2008minimum,kurz2008bounds,our-mz-rus}.
Курцу удалось существенно оптимизировать алгоритм, основанный на переборе, применяя характеристику~\cite{kurz2005characteristic,kreisel2008heptagon}.
Однако сложность вычисления максимальной возможной мощности для заданного диаметра $d$, по оценке Курца~\cite{kreisel2008heptagon},
составлет $O(d^3)$ (в двумерном случае).
Курц (TODO: ссылка!) отмечает, что важно перепроверять вычисленные значения независимо составленными программами для ЭВМ.
Приведём некоторые известные значения:

\begin{proposition}
\label{proposition:d(2,n)}
\end{proposition}
$(d(2, n))_{n=3,...,122} = 1$, 4, 7, 8, 17, 21, 29, 40, 51, 63, 74, 91, 104, 121,
134, 153, 164,
196, 212, 228, 244, 272, 288, 319, 332, 364, 396, 437, 464, 494, 524, 553, 578, 608,
642, 667, 692, 754, 816, 897, 959, 1026, 1066, 1139, 1190,  1248, 1306, 1363, 1410,
1460, 1514, 1564, 1614, 1675, 1727, 1770, 1817, 1887, 1906, 2060, 2140, 2169,
2231, 2299, 2432, 2494, 2556, 2624, 2692, 2827, 2895, 2993, 3098, 3196, 3294,
3465, 3575, 3658, 3749, 3885, 3922, 4223, 4380, 4437, 4559, 4693, 4883,
5018, 5109, 5264, 5332, 5480, 5603, 5738, 5938, 5995, 6052,
6324, 6432, 6630, 6738, 6939, 7061, 7245, 7384, 7568, 7752, 7935, 8119, 8321,
8406, 8648, 8729, 8927, 9052, 9211, 9423, 9534, 9794, 9905
.

Примечательно, что для $m=2$, $n=4$ и $n=18$ оптимальные множества не единственны.
Курц выяснил, что для $9 \leq n \leq 122$ оптимальное множество содержится
в объединении прямой и точки (т.е. все точки, кроме одной, лежат на некоторой прямой).

Для многомерного случая вычисленные значения можно найти в~\cite{kurz2008bounds}.


При изучении множеств из $\mathfrak{M}_n$ часто возникают множества специальных видов.
Наиболее простым примером являются веерные множества, т.е. множества, содержащиеся в объединении прямой и точки.
Курц и Вассерман~\cite{kurz2008minimum} показали, что диаметр веерного множества мощности $n$
не менее, чем $n^{c \log \log n}$, где $c$~--- некоторая константа.

TODO: про лемму

В этой же работе высказано предположение, что подобная оценка верна для всех множеств из $\mathfrak{M}_n$.
Заметим, что эта оценка совпадает с верхней оценкой~\eqref{upper_bound_d_m_n_log_log} на $d(2,n)$
с точностью до константы.
Это вкупе с тем фактом, что большинство известных оптимальных множеств являются веерными,
позволяет предположить, что все оптимальные множества являются веерными.

Возникает закономерный вопрос: а какие бывают множества из $\mathfrak{M}_n$,
кроме веерных?

\begin{definition}
	\cite{kurz2008minimum} Множество $M\in\mathfrak{M}_n$ называется множеством полуобщего положения,
	если ни через какие три точки из $M$ нельзя провести прямую.
	Множество всех множеств полуобщего положения мощности $n$ будем обозначать $\overline{\mathfrak{M}}_n$.
\end{definition}

В \cite{harborth1993upper} показано, что $\overline{\mathfrak{M}}_n\neq\varnothing$ для любого $n \geq 3$
и приведены примеры множеств из $\overline{\mathfrak{M}}_n$, содержащихся в окружности.
На эти множествах достигается верхняя оценка~\eqref{upper_bound_d_m_n_log_log} на $d(2,n)$
с некоторой константой.

Определим функцию~\cite{kurz2008minimum}
\begin{equation*}
	\overline{d}(m,n) = \min_{M\in\overline{\mathfrak{M}}(m,n)} \operatorname{diam} M
	.
\end{equation*}

В работе~\cite{kurz2008minimum} вычислено, что
\\
$\bigl(\overline{d}(2, n)\bigr)_{n=3,...,36} = 1,$
4, 8, 8, 33, 56, 56, 105, 105, 105, 532, 532, 735, 735, 735, 735,
1995, 1995, 1995, 1995, 1995, 1995, 9555, 9555, 9555, 10672,
13975, 13975, 13975, 13975, 13975, 13975, 13975, 13975.

\begin{definition}
	\cite{kurz2008minimum} Множество $M\in\overline{\mathfrak{M}}_n$, $n\geq 4$ называется множеством общего положения,
	если ни через какие четыре точки из $M$ нельзя провести окружность.
	Множество всех множеств общего положения мощности $n$ будем обозначать $\dot{\mathfrak{M}}_n$.
\end{definition}

В отличие от $\mathfrak{M}_n$ и $M\in\overline{\mathfrak{M}}_n$,
существование множеств из $\dot{\mathfrak{M}}_n$ является достаточно сложным вопросом.
Множество из $\dot{\mathfrak{M}}_6$ предъявлено в~\cite{noll1989nclusters},
множество из $\dot{\mathfrak{M}}_7$~--- в~\cite{kurz2008minimum}.
Определим функцию~\cite{kurz2008minimum}
\begin{equation*}
	\dot{d}(m,n) = \min_{M\in\dot{\mathfrak{M}}(m,n)} \operatorname{diam} M
	.
\end{equation*}
Согласно~\cite{kurz2008minimum},
$\bigl(\dot{d}(2,n)\bigr)_{n=3,...,7}
= 1,$ 8, 73, 174, 22270.

Вопрос о существовании множеств из $\dot{\mathfrak{M}}_8$
остаётся открытым.

\begin{definition}
	\cite{noll1989nclusters,kurz2013constructing}
	$n$--кластером называется множество $M\in\dot{\mathfrak{M}}_n$,
	содержащееся в целочисленной решётке.
\end{definition}
Существование $n$--кластера эквивалентно существованию множества $M\in\dot{\mathfrak{M}}_n$
характеристики 1.
В~\cite{noll1989nclusters} показано существование 6--кластеров,
в~\cite{kurz2013constructing}~--- 7--кластеров.

1. Особые случаи

1.5. Системы общего положения (гипотеза Эрдёша, работа Курца)
\cite{kreisel2008heptagon}

1.6. 7-кластеры (Курц)


Интересны, на взгляд авторов, и ещё несколько результатов для множеств из $\mathfrak{M}_n$ специального вида.

Так, Солимоси построил~\cite{solymosi2003note} для любого $n\geq 3$ конструкцию множества $M\in\mathfrak{M}_n$
такого, что для некоторых $M_1, M_2 \in M$ выполнено $|M_1 - M_2| = 2$.
Полученное им множество также является веерным.


TODO: про сущ-е множеств с миниметром 1.

Если $M\in\mathfrak{M}(m,n)$ и все рёбра $M$ имею чётную длину,
то понятно, что множество $H$, полученное из $M$ сжатием в два раза,
также будет принадлежать $\mathfrak{M}(m,n)$.
Выполняя такую операцию, мы рано или поздно получим множество из $\mathfrak{M}(m,n)$ такое,
что для некоторых $M_1,M_2\in M$ расстояние $|M_1 - M_2|$ нечётно.
Возникает закономерный вопрос: сколько может быть нечётных расстояний?

В~\cite{graham1974there} доказано, что
для существования множества $M\in\mathfrak{M}(m,m+2)$ такого, что для любых
$M_i,M_j\in M$, $i\neq j$ расстояние $|M_i-M_j|$ нечётно,
необходимо и достаточно, чтобы
$m+2\equiv 0 (\operatorname{mod} 16)$.
В частности, не существует такого множества $M\in\mathfrak{M}_4$,
что все его рёбра имеют нечётную длину.
На основе этого факта в работе~\cite{piepmeyer1996maximum} с использованием результатов теории графов доказано,
что количество рёбер нечётной длины в множестве $M\in\mathfrak{M}_n$ не превосходит
\begin{equation}
	\frac{n^2}{3} + \frac{r(r - 3)}{6}, \mbox{~где~} r = 1, 2, 3 \mbox{~и~} n \equiv r (\operatorname{mod} 3)
\end{equation}
и конструктивно показана точность этой оценки.

Отдельные оценки количества рёбер нечётной длины для множеств из $\overline{\mathfrak{M}}_n$ и $\dot{\mathfrak{M}}_n$
авторам неизвестны.



Для всех известных значений d(2,n) < d(2,n+1).
Для больших размерностей это неверно (ссылка, пример).


3. Связанные проблемы: гипотеза Эрдёша о числе расстояний, гипотеза Улама--Эрдёша.
Ссылка на 2 работы Гарибальди. Ссылка на доказательство гипотезы Эрдёша о числе расстояний.

4. Примеры множеств.


TODO: гипотеза: оптимальное множество не лежит на решётке


Перспективными направлениями исследований являются, кроме непосредственно отыскания как можно более точных оценок на $d(m,n)$,
например, поиск систем из $\mathfrak{M}_n$, содержащихся в двух прямых (параллельных или пересекающихся).
