Конструкцию $\mathfrak{M}_n$ можно распространить на пространства более высокой размерности.

%В пространствах более высокой размерности можно построить множества, аналогичные $\mathfrak{M}$.

\begin{definition}
	Пусть $m \geq 2$, $n \geq m + 1$.
	Через $\mathfrak{M}(m,n)$ будем обозначать множество таких подмножеств $m$--мерного евклидова пространства
	$M\subset\mathbb{R}^m$, что $|M| = n$, для любых $M_1,M_2 \in M$ выполнено $|M_1 - M_2| \in\mathbb{N}$
	и $M$ не содержится ни в какой $(m-1)$--мерной гиперплоскости.
\end{definition}

Более того, конструкцию $\mathfrak{M}_n$ можно распространить и на $\mathbb{Z}^m_n$~\cite{Kohnert2006IntegralPS}.
%Более того, аналогичные множества были построены в $\mathbb{Z}^m_n$~\cite{Kohnert2006IntegralPS}.

Характеристика (наряду с диаметром и мощностью) обобщается на случай, когда множество точек с целочисленными расстояниями
рассматривается в $\mathbb{R}^m$~\cite{kurz2005characteristic}.

При изучении множеств $\mathfrak{M}(m,n)$ возникает закономерный вопрос
о минимальном диаметре для заданных размерности и мощности.
Говоря формально, определим функцию
\begin{equation*}
	d(m,n) = \min_{M\in\mathfrak{M}(m,n)} \operatorname{diam} M
	.
\end{equation*}

\begin{definition}
	Множество $M\in\mathfrak{M}(m,n)$ называется оптимальным,
	если $\operatorname{diam} M = d(m,n)$.
\end{definition}

Приведём известные оценки на $d(m,n)$
(за основу взят список из~\cite{kurz2008bounds}):

%, дополненный результатом работы~\cite{nozaki2013lower}
%и результатом, полученным в данной статье.

\begin{align}
	& d(m, n - 1) \leq d(m,n) \\
	& d(n, n + 1) = 1 \\
	& d(m, n) \leq \begin{cases}
		2^{n-m+1} -2 & \mbox{~для~} n-m \equiv 0 \mod 2
		\\
		3(2^{n-m} -1) & \mbox{~для~} n-m \equiv 1 \mod 2
	\end{cases} \qquad \mbox{\cite{harborth1985diameters}} \\
	& d(2,n) \geq \frac{3}{8}n \qquad \mbox{для достаточно больших $n$ (утверждение~\ref{proposition:linear_bound_29})} \\
	& d(2,n) > 0.3457 n \qquad \mbox{для $n\geq 4$ (утверждение~\ref{proposition:linear_bound_10})} \\
	& d(m,n) \leq (n-m)^{c \log \log (n-m)} \mbox{~для некоторого $c$} \qquad \mbox{\cite{harborth1993upper}} \label{upper_bound_d_m_n_log_log}\\
	& d(m,n) > \sqrt{\frac{3}{2m}} n^{1/m} \mbox{~\cite{kanold1981punktmengen}, в частных случаях улучшена в \cite{nozaki2013lower}} \\
	& d(3,n) > \frac{1}{\sqrt{14}}n^{1/2} \mbox{~для~} n \geq 5 \quad \mbox{\cite{kanold1981punktmengen}} \\
	& d(m, 2m + 1) \leq 8 \qquad \mbox{\cite{piepmeyer1988raumliche}} \\
	& d(m, 2m + 2) \leq 13 \qquad \mbox{\cite{piepmeyer1988raumliche}} \\
	& d(m, 3m    ) \leq 109 \qquad \mbox{\cite{kemnitz1988punktmengen}} \\
	& d(m, m^2 + m) \leq 17 \qquad \mbox{\cite{kurz2008bounds}} \\
	& 3 \leq d(m, n) \leq 4 \mbox{~для~} m + 2 \leq n \leq 2m \mbox{~и~} d(m, 2m) = 4 ~ \mbox{\cite{piepmeyer1988raumliche,harborth1991point}}\\
\end{align}

Кроме того, в~\cite{kurz2008bounds} выдвинуто предположение, что $d(m - 1, n) \geq d(m, n)$.




Существуют алгоритмы вычисления $d(m,n)$ на ЭВМ, однако они требуют большого количества машинного времени,
причём требуемое время быстро растёт с ростом $m$ и $n$.
Первые несколько значений $d(2,n)$ были получены в~\cite{harborth1998integral},
следующие известные значения~--- в \cite{kurz2005characteristic,kurz2006konstruktion,kurz2008minimum,kurz2008bounds,our-mz-rus}.
Алгоритм, основанный на переборе,
может быть существенно оптимизирован с применением характеристики~\cite{kurz2005characteristic,kreisel2008heptagon}.
Однако сложность вычисления максимальной возможной мощности для заданного диаметра $d$, по оценке~\cite{kreisel2008heptagon},
составлет $O(d^3)$ (в двумерном случае).
%В (TODO: ссылка!) отмечено, что важно перепроверять вычисленные значения независимо составленными программами для ЭВМ.
Приведём некоторые известные значения:

\begin{proposition}
\label{proposition:d(2,n)}
\end{proposition}
$(d(2, n))_{n=3,...,122} = 1$, 4, 7, 8, 17, 21, 29, 40, 51, 63, 74, 91, 104, 121,
134, 153, 164,
196, 212, 228, 244, 272, 288, 319, 332, 364, 396, 437, 464, 494, 524, 553, 578, 608,
642, 667, 692, 754, 816, 897, 959, 1026, 1066, 1139, 1190,  1248, 1306, 1363, 1410,
1460, 1514, 1564, 1614, 1675, 1727, 1770, 1817, 1887, 1906, 2060, 2140, 2169,
2231, 2299, 2432, 2494, 2556, 2624, 2692, 2827, 2895, 2993, 3098, 3196, 3294,
3465, 3575, 3658, 3749, 3885, 3922, 4223, 4380, 4437, 4559, 4693, 4883,
5018, 5109, 5264, 5332, 5480, 5603, 5738, 5938, 5995, 6052,
6324, 6432, 6630, 6738, 6939, 7061, 7245, 7384, 7568, 7752, 7935, 8119, 8321,
8406, 8648, 8729, 8927, 9052, 9211, 9423, 9534, 9794, 9905
.

Для многомерного случая были вычислены~\cite{piepmeyer1988raumliche,harborth1998integral,kurz2005characteristic,kurz2006konstruktion,kurz2008bounds} следующие значения.
\begin{proposition}~
\end{proposition}
$(d(3, n))_{n=4,...,24} = 1$, 3, 4, 8, 13, 16, 17, 17, 17, 56, 65, 77, 86, 99, 112, 133, 154,
195, 212, 228, 244.

\begin{hypothesis}
	Для любого $n\geq 3$ выполнено $d(2,n) < d(2,n+1)$.
\end{hypothesis}
В пользу этой гипотезы говорят известные значения $d(2,n)$.
Для больших размерностей утверждение гипотезы неверно: так, $d(3,10) = d(3,11) = 17$.

\begin{hypothesis}
	Никакое оптимальное множество $M\in\mathfrak{M}$ не лежит на целочисленной решётке.
\end{hypothesis}
В пользу этой гипотезы говорят реузльтаты вычислений для $m=2$, $3 \leq n \leq 43$.

Примечательно, что для $m=2$, $n=4$ и $n=18$ оптимальные множества не единственны.
В~\cite{kurz2008bounds} установлено, что для $9 \leq n \leq 122$ оптимальное множество является веерным, т.е. содержится
в объединении прямой и точки (все точки, кроме одной, лежат на некоторой прямой).

\begin{hypothesis}
	Все оптимальные множества из $n$ точек на плоскости, $n\geq 9$, являются веерными.
\end{hypothesis}




При изучении множеств из $\mathfrak{M}_n$ часто возникают множества специальных видов.
Наиболее простым примером являются веерные множества.
Курц и Вассерман~\cite{kurz2008minimum} показали, что диаметр веерного множества мощности $n$
не менее, чем $n^{c \log \log n}$, где $c$~--- некоторая константа.
Это результат указывает на возможность улучшения оценки в следствии~\ref{corollary:max_points_on_line}.

В работе~\cite{kurz2008minimum} высказана
\begin{hypothesis}
	\label{hypot:log_lower_bound}
	Существует такая константа $c$,
	что для любого $n\geq 3$ и $M\in\mathfrak{M}_n$ выполнено
	$\operatorname{diam} M \geq n^{c \log \log n}$.
\end{hypothesis}

Заметим, что неравенство в гипотезе~\ref{hypot:log_lower_bound}
совпадает с верхней оценкой~\eqref{upper_bound_d_m_n_log_log} на $d(2,n)$
с точностью до константы.


Возникает закономерный вопрос: а какие бывают множества из $\mathfrak{M}_n$,
кроме веерных?

Положив $p=1$ в доказательстве теоремы~\ref{thm:power_char_exist} и добавив к построеннному множеству
точку $M_{n+2} = (0, -\sqrt{2^{n+1}})$, где $n$ нечётно,
получим множество из $n+2$ точек, из которых $n$ лежат на некоторой прямой и 2 вне её.

В~\cite{our-vvmsh-2018} приведены примеры различных множеств из $\mathfrak{M}_n$:
\begin{itemize}
	\item
		$
		\left\{
		(\pm 21/2, 0);
		(\pm 4, 3\sqrt{35}/2);
		(\pm 5/2, 0);
		(\pm 11/2, 0);
		\right\}
		$~---
		содержится в двух параллельных прямых (2 точки на одной, 6 на другой);



	\item
		$
		\left\{
		\left( {57/34} ; -{12\sqrt{35}/17}\right);
		\left( {44/17} ; {39\sqrt{35}/34}\right);
		\left( -{72/17} ;
		{15\sqrt{35}/34}\right);
		\right.$ \\ $\left.
		\left( -{175/34} ; -{24\sqrt{35}/17}\right);
		\left( -{57/34} ; {12\sqrt{35}/17}\right);
		%\right.$ \\ $\left.
		\left( \pm{17/2} ; 0\right);
		\right\}
		$~---
		содержится в двух параллельных прямых (3 точки на одной, 4 на другой);

	\item
		$
		\left\{
		(0,0);(\pm 50, 0); (\pm 14, \pm 48)
		\right\}
		$~---
		содержится в объединении окружности с её центром (7~точек);

	\item
		$
		\left\{
		\left( \pm30 ; 0\right);
		%\left( 30 ; 0\right);
		\left( \pm{15/2} ; {25\sqrt{7}/2}\right);
		%\left( {15/2} ; {25\sqrt{7}/2}\right);
		%\left( {33/2} ; {9\sqrt{7}/2}\right);
		\left( 0 ; {10\sqrt{7}/1}\right);
		\left( \pm{33/2} ; {9\sqrt{7}/2}\right);
		\right\}
		$~---
		содержится в двух прямых, пересекающихся под острым углом (7~точек).
\end{itemize}

Информация о некоторых других конструкциях может быть найдена в~\cite[\S 5.11]{brass2006research},~\cite[\S D20]{guy2013unsolved}.

TODO: спрямить ссылки!


\begin{definition}
	\cite{kurz2008minimum} Множество $M\in\mathfrak{M}(m,n)$ называется множеством полуобщего положения,
	если ни через какие $m+1$ точек из $M$ нельзя провести $(m-1)$--мерную гиперплоскость.
	Множество всех множеств полуобщего положения мощности $n$ в $\mathbb{R}^m$ будем обозначать $\overline{\mathfrak{M}}(m,n)$.
	Для краткости обозначим $\overline{\mathfrak{M}}_n=\overline{\mathfrak{M}}(2,n)$.
\end{definition}

В \cite{harborth1993upper} показано, что $\overline{\mathfrak{M}}_n\neq\varnothing$ для любого $n \geq 3$
и приведены примеры множеств из $\overline{\mathfrak{M}}_n$, содержащихся в окружности.
На эти множествах достигается верхняя оценка~\eqref{upper_bound_d_m_n_log_log} на $d(2,n)$
с некоторой константой.

Определим функцию~\cite{kurz2008minimum}
\begin{equation*}
	\overline{d}(m,n) = \min_{M\in\overline{\mathfrak{M}}(m,n)} \operatorname{diam} M
	.
\end{equation*}

В работе~\cite{kurz2008minimum} вычислено, что
\\
$\bigl(\overline{d}(2, n)\bigr)_{n=3,...,36} = 1,$
4, 8, 8, 33, 56, 56, 105, 105, 105, 532, 532, 735, 735, 735, 735,
1995, 1995, 1995, 1995, 1995, 1995, 9555, 9555, 9555, 10672,
13975, 13975, 13975, 13975, 13975, 13975, 13975, 13975.

\begin{definition}
	\cite{kurz2008minimum} Множество $M\in\overline{\mathfrak{M}}(m,n)$, $n\geq 4$ называется множеством общего положения,
	если ни через какие $m+2$ точек из $M$ нельзя провести сферу.
	Множество всех множеств общего положения мощности $n$ будем обозначать $\dot{\mathfrak{M}}(m,n)$.
	Для краткости обозначим $\dot{\mathfrak{M}}_n=\dot{\mathfrak{M}}(2,n)$.
\end{definition}

В отличие от $\mathfrak{M}_n$ и $M\in\overline{\mathfrak{M}}_n$,
существование множеств из $\dot{\mathfrak{M}}_n$ является достаточно сложным вопросом.
Множество из $\dot{\mathfrak{M}}_6$ предъявлено в~\cite{noll1989nclusters},
множество из $\dot{\mathfrak{M}}_7$~--- в~\cite{kurz2008minimum}.
Определим функцию~\cite{kurz2008minimum}
\begin{equation*}
	\dot{d}(m,n) = \min_{M\in\dot{\mathfrak{M}}(m,n)} \operatorname{diam} M
	.
\end{equation*}
Согласно~\cite{kurz2008minimum},
$\bigl(\dot{d}(2,n)\bigr)_{n=3,...,7}
= 1,$ 8, 73, 174, 22270.

Вопрос о существовании множеств из $\dot{\mathfrak{M}}_8$
остаётся открытым.

\begin{hypothesis}
	\label{hypot:general_position_nonempty}
	Для любого $n\geq 4$ непусто $\dot{\mathfrak{M}}_n$.
\end{hypothesis}


\begin{definition}
	\cite{noll1989nclusters,kurz2013constructing}
	$n$--кластером называется множество $M\in\dot{\mathfrak{M}}_n$,
	содержащееся в целочисленной решётке.
\end{definition}
Существование $n$--кластера эквивалентно существованию множества $M\in\dot{\mathfrak{M}}_n$
характеристики 1.
В~\cite{noll1989nclusters} показано существование 6--кластеров,
в~\cite{kurz2013constructing}~--- 7--кластеров.


Солимоси построил~\cite{solymosi2003note} для любого $n\geq 3$ конструкцию множества $M\in\mathfrak{M}_n$
такого, что для некоторых $M_1, M_2 \in M$ выполнено $|M_1 - M_2| = 2$.
Полученное им множество также является веерным.

Известно, что для $3\leq n \leq 5$ существуют множества из $\mathfrak{M}_n$
c минимальной длиной ребра 1.
Все известные примеры являются веерными~\cite{harborth1993upper}.
Существуют ли множества произвольной мощности с длиной ребра 1,
неизвестно, однако легко доказать следующее
\begin{proposition}
	\label{prop:edge_one}
	Для $n \geq 4$ не существует таких $M\in\overline{\mathfrak{M}}_n$,
	что для некоторых $M_1, M_2$ выполнено $|M_1 - M_2| = 1$.
\end{proposition}
\begin{proof}
	Пусть $M\in\overline{\mathfrak{M}}_n$ и для некоторых $M_1, M_2$ выполнено $|M_1 - M_2| = 1$.
	Тогда все точки множества $M$ лежат на $cr(M_1,M_2)$.
	$cr(M_1,M_2)$ есть объединение двух прямых, на каждой из которых может быть не более 2 точек в силу того, что
	$M\in\overline{\mathfrak{M}}_n$,
	откуда непосредственно следует требуемое утверждение для $n\geq 5$.

	Пусть теперь $n=4$.
	Предположим противное.
	Пусть $M=\{M_1, M_2, M_3, M_4\} \in \overline{\mathfrak{M}}_4$,
	$M_1 = (-1/2, 0)$, $M_2 = (1/2, 0)$, $O=(0,0)$.
	Тогда $M_3$ и $M_4$ лежат на оси ординат (см. выше).
	Так как $|M_3 - M_4| \in \mathbb{Z}$,
	то площадь треугольника $M_1 M_3 M_4$ есть рациональное число.
	По определению~\ref{def_char_classic} это означает, что $\operatorname{char} M = 1$.
	Тогда по определению~\ref{def_char_lattice} имеем $M_3 = (0, a/2)$.
	Понятно, что $a \neq 0$.
	Не теряя общности, положим $a\in\mathbb{N}$.
	Пусть $|M_1 - M_3| = b$.
	Применяя теорему Пифагора к треугольнику $O M_1 M_3$, получим уравнение
	\begin{equation*}
		\frac{1}{4} + \frac{a^2}{4} = b^2
		,
	\end{equation*}
	или, что то же самое,
	\begin{equation*}
		(2b)^2 - a^2 = 1
		.
	\end{equation*}
	Это уравнение не имеет решений в натуральных числах.
	Полученное противоречие завершает доказательство.
\end{proof}
Условие $n \geq 4$ в утверждении~\ref{prop:edge_one} существенно.
Для $n=3$ существует контрпример: равносторонний треугольник со стороной 1.



Если $M\in\mathfrak{M}(m,n)$ и все рёбра $M$ имеют чётную длину,
то понятно, что множество $H$, полученное из $M$ сжатием в два раза,
также будет принадлежать $\mathfrak{M}(m,n)$.
Выполняя такую операцию, мы рано или поздно получим множество из $\mathfrak{M}(m,n)$ такое,
что для некоторых $M_1,M_2\in M$ расстояние $|M_1 - M_2|$ нечётно.
Возникает закономерный вопрос: сколько может быть нечётных расстояний?

В~\cite{graham1974there} доказано, что
для существования множества $M\in\mathfrak{M}(m,m+2)$ такого, что для любых
$M_i,M_j\in M$, $i\neq j$ расстояние $|M_i-M_j|$ нечётно,
необходимо и достаточно, чтобы
$m+2\equiv 0 (\operatorname{mod} 16)$.
В частности, не существует такого множества $M\in\mathfrak{M}_4$,
что все его рёбра имеют нечётную длину.
На основе этого факта в работе~\cite{piepmeyer1996maximum} с использованием результатов теории графов доказано,
что количество рёбер нечётной длины в множестве $M\in\mathfrak{M}_n$ не превосходит
\begin{equation}
	\frac{n^2}{3} + \frac{r(r - 3)}{6}, \mbox{~где~} r = 1, 2, 3 \mbox{~и~} n \equiv r (\operatorname{mod} 3)
\end{equation}
и конструктивно показана точность этой оценки.
Множество, на котром эта оценка достигается, лежит на окружности.

Отдельные оценки количества рёбер нечётной длины для множеств из $\dot{\mathfrak{M}}_n$
авторам неизвестны.








Перспективными направлениями исследований являются, кроме непосредственно отыскания как можно более точных оценок на $d(m,n)$,
например, поиск систем из $\mathfrak{M}_n$, содержащихся в двух прямых (параллельных или пересекающихся).




С исследованием множеств $\mathfrak{M}_n$ тесно связаны ещё две гипотезы о расстояниях на плоскости.
\begin{hypothesis}[Улам, \cite{ulam1960collection}]
	\label{hypot:Ulam}
	Существует всюду плотное подмножество $U\subset\mathbb{R}^2$, такое,
	что для любых $U_1, U_2 \in U$ выполнено $|U_1 - U_2|\in\mathbb{Q}$,
	где $\mathbb{Q}$~--- множество рациональных чисел.
\end{hypothesis}
Если гипотеза~\ref{hypot:Ulam} справедлива, то для любого $n\geq 4$ множество $\dot{\mathfrak{M}}_n$ непусто
(гипотеза~\ref{hypot:general_position_nonempty}).
Неизвестно, следует ли из
гипотезы~\ref{hypot:general_position_nonempty} гипотеза~\ref{hypot:Ulam}.


В работах~\cite{garibaldi2005erdos,garibaldi2011erdos} исследуется связь множеств $\mathfrak{M}_n$ и
проблемы Эрдёша о числе различных расстояний~\cite{erdos1946sets}.
В работе~\cite{guth2015erdos} эта проблема в своей изначальной постановке была решена и доказана следующая
\begin{theorem}
	Пусть $G\subset{R}^2$, $|G| = n$.
	Тогда количество различных расстояний $|G_i - G_j|, G_i,G_j \in G$
	не менее $cn/\log n $.
\end{theorem}




TODO: Примеры множеств - рисунки (если будет время и место).

