\documentclass[a4paper,14pt]{article} %размер бумаги устанавливаем А4, шрифт 12пунктов
\usepackage[T2A]{fontenc}
\usepackage[utf8]{inputenc}
\usepackage[english,russian]{babel} %используем русский и английский языки с переносами
\usepackage{amssymb,amsfonts,amsmath,mathtext,enumerate,float,amsthm} %подключаем нужные пакеты расширений
\usepackage[pdftex,unicode,colorlinks=true,citecolor=black,linkcolor=black]{hyperref}
%\usepackage[pdftex,unicode,colorlinks=true,linkcolor=blue]{hyperref}
\usepackage{indentfirst} % включить отступ у первого абзаца
\usepackage[dvips]{graphicx} %хотим вставлять рисунки?
\graphicspath{{illustr/}}%путь к рисункам

\makeatletter
\renewcommand{\@biblabel}[1]{#1.} % Заменяем библиографию с квадратных скобок на точку:
\makeatother %Смысл этих трёх строчек мне непонятен, но поверим "Запискам дебианщика"

\usepackage{geometry} % Меняем поля страницы.
\geometry{left=4cm}% левое поле
\geometry{right=1cm}% правое поле
\geometry{top=2cm}% верхнее поле
\geometry{bottom=2cm}% нижнее поле

\renewcommand{\theenumi}{\arabic{enumi}}% Меняем везде перечисления на цифра.цифра
\renewcommand{\labelenumi}{\arabic{enumi}}% Меняем везде перечисления на цифра.цифра
\renewcommand{\theenumii}{.\arabic{enumii}}% Меняем везде перечисления на цифра.цифра
\renewcommand{\labelenumii}{\arabic{enumi}.\arabic{enumii}.}% Меняем везде перечисления на цифра.цифра
\renewcommand{\theenumiii}{.\arabic{enumiii}}% Меняем везде перечисления на цифра.цифра
\renewcommand{\labelenumiii}{\arabic{enumi}.\arabic{enumii}.\arabic{enumiii}.}% Меняем везде перечисления на цифра.цифра

\sloppy


\renewcommand\normalsize{\fontsize{14}{25.2pt}\selectfont}

\usepackage[backend=biber,style=gost-numeric,sorting=none]{biblatex}
\addbibresource{../common/notmy.bib}
\addbibresource{../common/my.bib}


\begin{document}
\renewcommand{\bibname}{Список цитированной литературы}
\renewcommand\refname{\bibname}
% !!!
% Здесь начинается реальный ТеХ-код
% Всё, что выше - беллетристика

\paragraph{Определение.}
Системой Эрдёша называется множество точек $\{M_1, M_2, ..., M_n\}$ на плоскости, не содержащееся ни в какой прямой,
такое, что для любых $i\neq j$ расстояние $|M_i M_j| \in \mathbb{N}$,
т.е. является натуральным числом.

Для дальнейшего описания свойств систем Эрдёша нам потребуются следующие определения и обозначения.

\paragraph{Определение.}
Натуральное число $r$ называется свободным от квадратов [\cite{Bukhstab-number-theory}, гл. 34, п. 3], если оно не допускает представления в виде $r = r_1^2 \cdot r_2$,
где $r_1, r_2$~--- натуральные числа и $r_1 > 1$.

\paragraph{Обозначение.}
$(n,r)$--решёткой или $\L{}(n,r)$ будем называть решётку \cite{polygons-on-lattices},
построенную на векторах $(1/n, 0)$ и $(0, \sqrt{r}/n)$,
где $n\in\mathbb{N}$, и $r$ свободно от квадратов,
т.е. множество
\begin{equation}
	\L{}(n,r) = \left\{
		\left( \frac{a}{n}, \frac{b\sqrt{r}}{n} \right):
		a,b \in \mathbb{Z}
	\right\}.
\end{equation}

\paragraph{Обозначение.}
Через $\mathbb{Q}\sqrt{r}$ будем обозначать множество $\{q\sqrt{r} : q \in \mathbb{Q}, r\in\mathbb{N}\}$
и подразумевать, что $r$ свободно от квадратов.

\paragraph{Обозначение.}
Площадь треугольника $ABC$ будем обозначать символом $S\triangle ABC$.

Также нам потребуются следующие утверждения.

\paragraph{Утверждение 1.}
Если $a\in\mathbb{Q}\sqrt{r}$ и $a\in\mathbb{Q}\sqrt{p}$, то $p=r$.

Надо ли это доказывать?


\paragraph{Утверждение 2.}
Если $A,B,C\in\L{}(n,r)$, то $S\triangle ABC \in \mathbb{Q}\sqrt{r}$.

Это утверждение непосредственно следует из [\cite{polygons-on-lattices}, теорема 3.1].

В дальнейшем, говоря о системе координат,
мы будем подразумевать прямоугольную декартову систему координат,
в которой расстояния сохраняются.

(TODO: как эту мысль выразить по-человечеки?!)


\paragraph{Лемма 1.}

Для любой системы Эрдёша $S=\{M_1, M_2, ..., M_n\}$ существует такое целое число $m$, не превосходящее её диаметра $diam(S)$,
что можно выбрать систему координат способом, при котором координаты каждой точки системы будут иметь вид
\begin{equation}
	M_i = \left(
		\frac{p_i}{2m}
		;
		\frac{\pm\sqrt{q_i}}{2m}
	\right),
\end{equation}
где $p_i \in \mathbb{Z}, q_i \in \mathbb{N}$.


\paragraph{Доказательство.}
Пусть $|M_1 M_2| = m$, тогда, очевидно, $m \leq diam(S)$.
Введём систему координат так, что $M_1=(-m/2, 0)$, $M_2=(m/2, 0)$
Рассмотрим некоторую другую точку $M_i=(x, y)\in S$.
Обозначим $M_i M_1 = a$, $M_i M_2 = b$.
Тогда $M_i$ лежит на пересечении двух окружностей,
задаваемых уравнениями
\begin{gather}
	\left(-\frac{m}{2} - x\right)^2 + y ^2 = a^2,
\\
	\left( \frac{m}{2} - x\right)^2 + y ^2 = b^2,
\end{gather}
где $a+b\geq m$.

Решив её, имеем
\begin{gather}
	x = \frac{a^2 - b^2}{2 m} = \frac{p_i}{2m},
\\
%	y = \frac{\pm \sqrt{-(a^2 - b^2)^2 + m^2 (2 a^2 + 2 b^2  - m^2)} }{2m}.
	y = \pm\sqrt{a^2 - \left(\frac{m}{2}+x\right)^2} =
	\pm\sqrt{a^2 - \left(\frac{m}{2}+\frac{p_i}{2m}\right)^2} =
	\frac{\pm\sqrt{q_i}}{2m},
\end{gather}
что и требовалось доказать.

\paragraph{Лемма 2.}

Для любой системы Эрдёша $S=\{M_1, M_2, ..., M_n\}$ существует такое целое число $m$, не превосходящее её диаметра $diam(S)$,
и такое натуральное число $q$, свободное от квадратов,
что можно выбрать систему координат способом, при котором координаты каждой точки системы будут иметь вид
\begin{equation}\label{grid_for_Erdosh_system}
	M_i = \left(
		\frac{p_i}{2m}
		;
		\frac{s_i\sqrt{q}}{2m}
	\right),
\end{equation}
где $p_i, s_i \in \mathbb{Z}$.

\paragraph{Доказательство.}
Рассмотрим две точки $M_1, M_2 \in S$.
По лемме 1 их координаты (при введении соответствующей системы координат) выражаются в виде
\begin{equation}
	M_1 = \left(
		\frac{p_1}{2m}
		;
		\frac{\pm\sqrt{q_1}}{2m}
	\right),
	M_2 = \left(
		\frac{p_2}{2m}
		;
		\frac{\pm\sqrt{q_2}}{2m}
	\right),
\end{equation}
где $p_i \in \mathbb{Z}, q_i \in \mathbb{N}$, $i=1,2$.

Тогда
\begin{equation}
	|M_1 M_2| = \frac{1}{2m} \sqrt{p_1^2 + p_2^2 - 2p_1p_2 + q_1 + q_2 - 2\sqrt{q_1 q_2}}
\end{equation}
Для того, чтобы расстояние $|M_1 M_2| \in \mathbb{N}$, необходимо,
чтобы $|M_1 M_2|^2 \in \mathbb{N}$, для чего, в свою очередь, необходимо, чтобы
$p_1^2 + p_2^2 - 2p_1p_2 + q_1 + q_2 - 2\sqrt{q_1 q_2} \in \mathbb{N}$,
что влечёт $\sqrt{q_1 q_2} \in \mathbb{N}$,
что, в свою очередь, требует соотношения
$q_1 = s_1^2 \sqrt{q}$, $q_2 = s_2^2 \sqrt{q}$.
Лемма доказана.

\paragraph{Теорема 1.}
Любая система Эрдёша $S$ может быть помещена на $\L{}(n,q)$ в некоторой системе координат,
причём $q$ не зависит от выбора системы координат.

\paragraph{Доказательство.}
Существование $n$ и $q$ следует из леммы 2.
Покажем, что такое $q$ единственно.
Пусть нашлась система координат, в которой $S \subset \L{}(m,p)$.
Пусть $M_1, M_2, M_3 \in S$ и не лежат на одной прямой.
Тогда по утверждению 2 $S\triangle M_1 M_2 M_3 \in \mathbb{Q}\sqrt{q}$
и $S\triangle M_1 M_2 M_3 \in \mathbb{Q}\sqrt{p}$,
откуда в силу утверждения 1 $p=q$,
что и требовалось доказать.

\paragraph{Замечание.}
$n$, вообще говоря, определяется не единственным образом.

\paragraph{Определение.}
Число $q$ из предыдущей теоремы будем называть орнаментом системы Эрдёша $S$.

TODO: хороший ли это термин? На английский переводится легко (ornament).

Таким образом, у системы Эрдёша, наряду с двумя наиболее очевидными характеристиками~--- диаметром и мощностью~\cite{our-mz-rus}~---
%TODO: english mz paper
существует ещё одна характеристика, инвариантная относительно выбора системы координат~--- орнамент.
Следует отметить, что в отличие от диаметра и мощности,
орнамент есть локальная характеристика в том смысле, что при рассмотрении подсистемы
(т.е. системы Эрдёша $S'$, являющейся подмножеством данной системы $S$)
орнамент сохраняется.
С точки зрения геометрического смысла, орнамент~---
это минимальное натуральное число, с корнем из которого соизмерима
площадь любого треугольника, вершины которого принадлежат данной системе
(говоря точнее, из (\ref{grid_for_Erdosh_system}) следует,
что площадь такого треугольника кратна $\sqrt{q}/4$).

Примеры систем Эрдёша с различным орнаментом можно найти, например,
в \cite{our-vvmsh-2018,our-mkmitu-2016}

Возникает закономерный вопрос о соотношении орнамента, диаметра и мощности.

Из леммы 1 непосредственно вытекает

\paragraph{Следствие 1.}
Если диаметр системы Эрдёша равен $m$,
то её орнамент принадлежит множеству
\begin{multline}
	\mathbb{N} \cap \{
		-(m+a+b)(m-a-b)(m+a-b)(m-a+b)/c
	\\
	\mbox{где }
		a,b,c\in\mathbb{N}; a \leq m; b\leq m; a+b > m
	\},
	%\right\}
\end{multline}

\paragraph{Следствие 2.}
Для диаметра $m$ существует не более $(m^2 + m)/2$ возможных орнаментов.

TODO: доказательство.

TODO: улучшить оценку из соображений симметрии предположительно до $(m^2+3m)/4 - 1$.

\paragraph{Следствие 3.}
Если диаметр системы Эрдёша равен $m$, то её орнамент строго меньше $12m^4$.

TODO: доказательство.

Обратной оценки, т.е. оценки на диаметр, зависящей от орнамента, нет.
(Достаточно заметить, что при растяжении системы в целое количество раз её орнамент не изменяется.)

Нет и подобной взаимосвязи между орнаментом и мощностью, а именно верна следующая
\paragraph{Теорема 2.}
Для любых наперёд заданных натурального $p\leq 3$ и свободного от квадратов $q$
существует система Эрдёша с мощностью $p$ и орнаментом $q$.

TODO: доказательство (конструктивное).


Ещё сюда можно добавить табличку как в \cite{our-mz-rus},
но с орнаментами оптимальных систем.

\printbibliography

\end{document}
