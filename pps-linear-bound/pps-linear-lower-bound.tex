\documentclass[a4paper,14pt]{article} %размер бумаги устанавливаем А4, шрифт 12пунктов
\usepackage[T2A]{fontenc}
\usepackage[utf8]{inputenc}
\usepackage[russian,english]{babel} %используем русский и английский языки с переносами
\usepackage{amssymb,amsfonts,amsmath,mathtext,enumerate,float,amsthm} %подключаем нужные пакеты расширений
\usepackage[pdftex,unicode,colorlinks=true,citecolor=black,linkcolor=black]{hyperref}
%\usepackage[pdftex,unicode,colorlinks=true,linkcolor=blue]{hyperref}
\usepackage{indentfirst} % включить отступ у первого абзаца
\usepackage[dvips]{graphicx} %хотим вставлять рисунки?
\graphicspath{{illustr/}}%путь к рисункам

\makeatletter
\renewcommand{\@biblabel}[1]{#1.} % Заменяем библиографию с квадратных скобок на точку:
\makeatother %Смысл этих трёх строчек мне непонятен, но поверим "Запискам дебианщика"

\usepackage{geometry} % Меняем поля страницы.
\geometry{left=2cm}% левое поле
\geometry{right=1cm}% правое поле
\geometry{top=2cm}% верхнее поле
\geometry{bottom=2cm}% нижнее поле

\renewcommand{\theenumi}{\arabic{enumi}}% Меняем везде перечисления на цифра.цифра
\renewcommand{\labelenumi}{\arabic{enumi}}% Меняем везде перечисления на цифра.цифра
\renewcommand{\theenumii}{.\arabic{enumii}}% Меняем везде перечисления на цифра.цифра
\renewcommand{\labelenumii}{\arabic{enumi}.\arabic{enumii}.}% Меняем везде перечисления на цифра.цифра
\renewcommand{\theenumiii}{.\arabic{enumiii}}% Меняем везде перечисления на цифра.цифра
\renewcommand{\labelenumiii}{\arabic{enumi}.\arabic{enumii}.\arabic{enumiii}.}% Меняем везде перечисления на цифра.цифра

\sloppy


\renewcommand\normalsize{\fontsize{14}{25.2pt}\selectfont}

\usepackage[backend=biber,style=gost-numeric,sorting=none]{biblatex}
\addbibresource{../common/notmy.bib}
\addbibresource{../common/my.bib}


\theoremstyle{plain}
\newtheorem{theorem}{Theorem}[section]
\newtheorem{conjecture}[theorem]{Conjecture}
\newtheorem{lemma}[theorem]{Lemma}
\newtheorem{corollary}[theorem]{Corollary}
\newtheorem{observation}[theorem]{Observation}

\theoremstyle{definition}
\newtheorem{definition}[theorem]{Definition}
\newtheorem{construction}[theorem]{Construction}
\newtheorem{remark}[theorem]{Remark}
\newtheorem{problem}[theorem]{Problem}

\begin{document}

%\renewcommand{\bibname}{Список цитированной литературы}
%\renewcommand\refname{\bibname}
% !!!
% The text starts here

Avdeev N.N.

On existence and diameter bounds of integral point sets.
\footnote{
	This work was carried out at Voronezh State University and supported by the Russian Science
	Foundation under grant 19-11-00197.
}

\paragraph{Abstract.}
A point set $M$ in $m$-dimensional Euclidean space is called an integral point set if all the distances between the
elements of $M$ are integers, and $M$ is not situated on an $(m-1)$-dimensional hyperplane.
We improve the linear lower bound for diameter of planar integral point sets.
This improvement takes into account some results related to the Point Packing in a Square problem.
Then for arbitrary integers $m \geq 2$, $n \geq m+1$, $d \geq 1$
we give a construction of an integral point set $M$ of $n$ points in $m$-dimensional Euclidean space,
where $M$ contains points $M_1$ and $M_2$ such that distance between $M_1$ and $M_2$ is exactly $d$.



\section{Introduction}
Let $\mathbb{N}$ be the set of all positive integers and let $|M_1 M_2|$ denote the Euclidean distance
between points $M_1$ and $M_2$ in a finite-dimensional space $\mathbb{R}^m$
(and, more generally, let $|\Delta|$ denote the length of line segment $\Delta$).
An \textit{integral point set} in $m$-dimensional Euclidean space is a point set $M$ such that all the distances between the
points of $M$ are integers and $M$ is not situated on an $(m-1)$-dimensional hyperplane.
Erd\"os and Anning proved~\cite{anning1945integral,erdos1945integral} that every integral point set consists of a finite number of points.
Taking the fact into account, we will denote the set of all integral point sets of $n$ points in $m$-dimensional Euclidean space with
$\mathfrak{M}(m,n)$ (following the notation of~\cite{our-vmmsh-2018}).
The symbol $\# M$ will be used for cardinality of $M$, i.e. the quantity of points in $M$ in our case.

For every finite point set, its diameter is naturally defined as
\begin{equation}
	\operatorname{diam} M = \max_{A,B\in M} |AB|
	.
\end{equation}
The next emerging question is: how the diameter of an integral point set depends on its cardinality?
One can easily see that every $M\in\mathfrak{M}(m,n)$ with $\operatorname{diam} M = h$
can be dilated to $M_p\in\mathfrak{M}(m,n)$ with $\operatorname{diam} M = ph$
for every $p\in\mathbb{N}$.
So, the question above should be rephrased:
how \textit{the least possible} diameter of an integral point set depends on its cardinality?
To answer it, the following function was introduced~\cite{kurz2008bounds,kurz2008minimum}:
\begin{equation}
	d(m,n) = \min_{M\in\mathfrak{M}(m,n)} \operatorname{diam} M = \min_{M\in\mathfrak{M}(m,n)} \max_{A,B\in M} |AB|
	.
\end{equation}
We also refer to~\cite{kurz2008bounds} for a list of known exact values of $d(m,n)$
and estimations; in the present paper, the case of $m=2$ will mostly be in the focus.

The most significant breakthrough on the planar case was done by Solymosi~\cite{solymosi2003note},
who proved that $cn \leq d(2,n)$ for a sufficiently small constant $c$.
Following the details of Solymosi's proof carefully,
one can know out that the inequality holds at least for $c = 1/24$.
(See~\cite[Exercise 2.6]{garibaldi2005erdos} for some remarks.)
The constant was improved in~\cite{our-mz-rus} to $1/8$ for all $n$ and in~\cite{our-vmmsh-2018}
to $3/8$ for sufficiently big $n$.

The paper~\cite{solymosi2003note} contained one more interesting result.
Let us define a function which is ``dual'' to $d(m,n)$ in some sense:
\begin{equation}
	l(m,n) = \min_{M\in\mathfrak{M}(m,n)} \min_{A,B\in M} |AB|
	.
\end{equation}
Solymosi proved that $l(2,n)\leq 2$.

In the present paper we improve Solymosi's results:
firstly, we obtain a greater constant $c = 5/11$ in Theorem~\ref{thm:main_estimate},
using the combined approach with the Point Packing in a Square problem
(we notice that such an approach is different from Solymosi's one);
then, we prove that $l(m,n)=1$ for all possible $m$ and $n$.

\section{Lower bound for the diameter}

In this section,
we improve a better lower bound for minimum diameter of planar integral point sets by
involving Point Packing in a Square problem.
Below we introduce the problem, basic notions and results.

\begin{problem}[Point Packing in a Square (PPS)~\cite{locatelli2002packing,costa2013valid}]
	\label{problem:PPS}
	Given an integer $k > 1$, place $k$ points in the unit square $U = [ 0 , 1 ]^2$ such that their
	minimum pairwise distance $m$ is maximal.
\end{problem}

\begin{definition}
	For each $k > 1$, the corresponding maximal distance $m$ from Problem~\ref{problem:PPS}
	is called the $k$-th PPS coefficient and denoted by $\varphi_k$.
\end{definition}
So, it's impossible to place $k$ points in a unit square in such a way that each pairwise distance of the points is greater than $\varphi_k$.


\begin{theorem}
	\label{thm:varphi_k_bounds}
	\cite{costa2013valid}
	For every $k\geq 2$ the following inequality holds:
	\begin{equation*}
		\sqrt{\frac{2}{k\sqrt{3}}}
		\leq
		\varphi_k
		\leq
		\frac{1}{k-1} +
		\sqrt{
			\frac{1}{(k-1)^2}
			+
			\frac{2}{(k-1)\sqrt{3}}
		}
	\end{equation*}
\end{theorem}


For building the bound, we also require the following results and notions from~\cite{our-vmmsh-2018}.

\begin{lemma}
	\cite[Lemma 4]{our-vmmsh-2018}
	\label{lem:square_container}
	Let $M\in\mathfrak{M}(2,n)$, $\operatorname{diam} M = d$.
	Then $M$ is situated in a square of side length $d$.
\end{lemma}

\begin{definition}
	\textit{Cross} for points $M_1$ and $M_2$, denoted by $cr(M_1,M_2)$, is union of two straight lines:
	the line which the points $M_1$ and $M_2$ are situated on,
	and the bisector of line segment $M_1 M_2$.
\end{definition}

\begin{lemma}
	\label{lem:intervals_cross}
	If \textbf{open} line segments $M_1 M_2$ and $M_3 M_4$ do not intersect,
	%$M_1 \neq M_2$, $M_3 \neq M_4$,
	then the set $cr(M_1,M_2) \cap cr(M_3,M_4)$ is either a straight line or contains 2 or 4 points.
\end{lemma}

\begin{lemma}
	\label{lemma:quadr_diag_edges}
	Let $ABCD$ be a convex quadrilateral on the plane.
	Then $\max\{AC,BD\}>\min\{AB,BC,CD,DA\}$,
	i.e. at least one diagonal is greater than at least one side.
\end{lemma}


Basing on the exact values of $d(2,n)$ for $ 3 \leq n\leq 122$ and the estimate $d(2,123)>10000$
\cite{kurz2008bounds}, we derive
\begin{observation}
	\label{obs:4_leq_n_leq_21491}
	The inequality
	\begin{equation}
		d(2,n) \geq 3^{1/4}\cdot2^{-3/2} \cdot n
	\end{equation}
	holds for $4 \leq n \leq 21491$.
\end{observation}

Performing some simple manipulations with the upper bound in Theorem~\ref{thm:varphi_k_bounds},
one can make the following
\begin{observation}
	\label{varphi_n_where_n_geq_21492}
	For $n \geq 21492$ we have
	\begin{equation}
		\varphi_n \leq \varphi_{n-1} \leq \frac{\beta}{\sqrt{n-2}}
		,
	\end{equation}
	where
	\begin{equation}
		\beta = \frac{1}{\sqrt{21490}} + \sqrt{ \frac{2}{\sqrt{3}} + \frac{1}{21490} } < 1.07464
		.
	\end{equation}
\end{observation}

In [TODO:cite], TODO: theorem about sublinear quantity of points on the line.
However, we need the values of all the constants in such an estimate.
To obtain such values, we will now prove a generalization of~\cite[lemma 3]{our-vmmsh-2018}.

\begin{definition}
	For a line segment $M_1 M_2$ and an integer $k$, $-|M_1 M_2| < k < |M_1 M_2|$,
	we define a $\rho(k,M_1 M_2)$-curve as the set of points $N$
	for which the equality $|N M_1| - |N M_2| = k$ holds.
\end{definition}
So, in the planar case a $\rho(k,M_1 M_2)$-curve
is a branch of a hyperbola for $k\neq 0$ and the bisector of line segment $M_1 M_2$ for $k=0$.

\begin{observation}
	\label{obs:rho_curves}
	If points $M_1,M_2,M_3,M_4$ are situated on a straight line,
	and equality $|M_1 M_2| = |M_3 M_4|$ holds but line segments $M_1 M_2$ and $M_3 M_4$ do not coincide,
	then for a fixed $k$ the $\rho(k,M_1 M_2)$-curve and $\rho(k,M_3 M_4)$-curve
	do not intersect.
\end{observation}

\begin{lemma}
	\label{lem:2k-1_segments}
	Let $M \in \mathfrak{M}(2,n)$ for some $n$ and let $m$ be a straight line.
	Then for every $k\in\mathbb{N}$ there are at most $2k-1$ segments $\Delta_i \subset m$,
	such that $|\Delta_i| = k$.
\end{lemma}
\begin{proof}
	Consider a point $N\in M \setminus m$.
	Then for each $\Delta_i$ there is a $\rho(n_i,\Delta_i)$-curve containing $N$.
	Due to Observation~\ref{obs:rho_curves}, all $n_i$ are distinct;
	otherwise $\rho(n_i,\Delta_i)$-curve and $\rho(n_i,\Delta_j)$-curve, $j\neq i$, do not intersect.
	There can be only $2k-1$ different values for $n_i$,
	so there are at most $2k-1$ distinct segments $\Delta_i$.
\end{proof}

\begin{lemma}
	\label{lem:line_segment_with n_squared_plus_one_points}
	Let $\Delta$ be a straight line segment, $|\Delta|=l$ and $M \in \mathfrak{M}(2,n)$ for some $n$.
	Let $\#(\Delta \cap M) = n^2 + 1$.
	Then
	\begin{equation}
		l \geq \frac{2}{3}n^3+\frac{1}{2}n^2-\frac{1}{6}n
		.
	\end{equation}
\end{lemma}

\begin{proof}
	$n^2+1$ points, including the ends of $\Delta$, split the segment $\Delta$ into $n^2$
	sequential segments $\Delta_i$.
	Due to Lemma~\ref{lem:2k-1_segments}, there is at most one segment of length 1,
	at most three segments of length 2, etc.
	The following two values of sums conclude the proof:
	\begin{equation}
		1 + \sum_{k=1}^n (2k-1) = n^2 + 1
		,
	\end{equation}
	\begin{equation}
		\sum_{k=1}^n k(2k-1) = \frac{2}{3}n^3+\frac{1}{2}n^2-\frac{1}{6}n
		.
	\end{equation}
\end{proof}

Now we will estimate the length of a line segment that shares an arbitrary number of points
with an integral point set.

\begin{lemma}
	Let $\Delta$ be a straight line segment, $|\Delta| = b$ and $M \in \mathfrak{M}(2,n)$ for some $n$.
	Let $\#(\Delta \cap M) = t$.
	Then
	\begin{equation}
		\label{eq:estimate_for_segment_length}
		b\geq \frac{2}{3}t^{3/2}-\frac{3}{2}t+\frac{5}{6}t^{1/2}
		.
	\end{equation}
\end{lemma}

\begin{proof}
	Let $f(k)$ denote the mininal length of a line segment
	that shares $k$ points with and integral point set.
	We observe that $f(k) > f (k-1)$.
	Due to Lemma~\ref{lem:line_segment_with n_squared_plus_one_points},
	$f(n^2+1) \geq \frac{2}{3}n^3+\frac{1}{2}n^2-\frac{1}{6}n$.

	%Consider $t$ such that $n^2 + 1 \leq t \leq (n+1)^2$. %,
	%then $f(t) > f(n^2 +1)$.% and $n \geq \sqrt{t} - 1$.
	For $t\in\mathbb{N}$ the inequality $(\sqrt{t} - 1)^2 +1 \leq t$ holds,
	thus
	%and derive that
	\begin{equation}
		\label{eq:line_segment_length}
		f(t) \geq f((\sqrt{t} - 1)^2 +1) \geq \frac{2}{3}(\sqrt{t} - 1)^3+\frac{1}{2}(\sqrt{t} - 1)^2-\frac{1}{6}(\sqrt{t} - 1)
		=
		\frac{2}{3}t^{3/2}-\frac{3}{2}t+\frac{5 }{6}t^{1/2}
		.
	\end{equation}
\end{proof}

This lemma leads to the following observation.

\begin{observation}
	\label{obs:estimate_points_on_straight_line}
	Let $\Delta$ be a straight line segment, $|\Delta| = b$ and $M \in \mathfrak{M}(2,n)$ for some $n$.
	Let $\#(\Delta \cap M) = k$ and $b>10000$.
	Then $k \leq \gamma_2 b + 6$,
	where
	\begin{equation}
		\gamma_2 = \frac{3846}{2593 \sqrt{647}-5823}
	\end{equation}
\end{observation}

\begin{proof}
	We are interested in integral $t$ such that estimate~\eqref{eq:estimate_for_segment_length}
	yields to $b > 10000$, because we know the maximum quantity of points for integral point sets of diameters at most $10000$.
	So, let us consider $t\geq 647$ and find a coefficient $\gamma_2$,
	such that the inequality
	\begin{equation}
		\label{eq:inequality_for_linear_estimate_of_segment_length}
		\frac{2}{3}t^{3/2}-\frac{3}{2}t+\frac{5 }{6}t^{1/2} \geq \frac{t-6}{\gamma_2}
	\end{equation}
	holds for all $t\geq 647$.
	For these $t$,
	the left part of~\eqref{eq:inequality_for_linear_estimate_of_segment_length} obviously grows faster than the right one.
	Turning~\eqref{eq:inequality_for_linear_estimate_of_segment_length} into the same equality
	and solving it for $t=647$, we obtain the claimed estimate.
\end{proof}

Now we are ready to prove the main theorem of the section.

\begin{theorem}
	\label{thm:main_estimate}
	If $n\geq 4$, then $d(2,n) \geq \gamma (n - 2)$,
	where
	\begin{multline}
		\frac{3^{1/4}}{2^{3/2}} >
		\gamma = \frac{\sqrt{16 {{\left( \sqrt{\frac{2}{\sqrt{3}}+\frac{1}{21490}}+\frac{1}{\sqrt{21490}}\right) }^{2}}+\frac{14791716}{{{\left( 2593 \sqrt{647}-5823\right) }^{2}}}}-\frac{3846}{2593 \sqrt{647}-5823}}{8 {{\left( \sqrt{\frac{2}{\sqrt{3}}+\frac{1}{21490}}+\frac{1}{\sqrt{21490}}\right) }^{2}}}
		>\\
		> 0.45557
		> \frac{5}{11}
	\end{multline}
\end{theorem}

\begin{proof}
	For $4 \leq n \leq 21491$, the claim of the theorem follows immediately from Observation~\ref{obs:4_leq_n_leq_21491}.
	Let us consider $M\in \mathfrak{M}(2,n)$, $n \geq 21492$, $\operatorname{diam} M = b$.
	Lemma~\ref{lem:square_container} yields that $M$ is situated in a square of side length $b$.
	Let $M_1, M_2, M_3, M_4$ be points of $M$ such that the distances $|M_1 M_2|$ and $|M_3 M_4|$
	are minimal in $M$ ($M_2$ and $M_3$ may be the same point).
	Then $|M_1 M_2| \leq b\varphi_{n-1}$ and $|M_3 M_4| \leq b\varphi_{n-1}$.
	Due to Lemma~\ref{lemma:quadr_diag_edges}, open line segments $M_1 M_2$ and $M_3 M_4$ do not intersect
	(otherwise they are not minimal).

	Let $C = cr(M_1 M_2) \cap cr(M_3 M_4)$.
	Each point of $N\in M$ satisfies one of the following conditions:

	a) $N$ belongs to $C$~--- totally at most $\gamma_2 b + 6$ points due to Lemma~\ref{lem:intervals_cross} and Observation~\ref{obs:estimate_points_on_straight_line};

	b) $N$ belongs to the intersection of one of $|M_1 M_2| - 1$ hyperbolas
	with one of $|M_3 M_4| - 1$ hyperbolas~--- totally at most $4 (|M_1 M_2| - 1)(|M_3 M_4| - 1)$ points;

	c) $N$ belongs to the intersection of one of $|M_1 M_2| - 1$ hyperbolas and $cr(M_3 M_4)$~---
	totally at most $4 (|M_1 M_2| - 1)$ points;

	d) $N$ belongs to the intersection of one of $|M_3 M_4| - 1$ hyperbolas and $cr(M_1 M_2)$~---
	totally at most $4 (|M_3 M_4| - 1)$ points;

	(see~\cite{erdos1945integral} for details).
	Summing up the cases above, we get the following estimate:
	\begin{equation}
		\label{eq:estimate_n_varphi_n_squared}
		n \leq 4 b^2 \varphi_{n-1}^2 - 4 + \gamma_2 b + 6
		.
	\end{equation}
	Observation~\ref{varphi_n_where_n_geq_21492} turns estimate~\eqref{eq:estimate_n_varphi_n_squared}
	into the following:
	\begin{equation}
		\label{eq:estimate_n_sqrt_n-2}
		n-2 \leq 4 b^2 \frac{\beta^2}{n-2} + \gamma_2 b
		,
	\end{equation}
	which obviously leads to inequality
	\begin{equation}
		1 \leq 4\beta^2 \left(\frac{b}{n-2}\right)^2  + \gamma_2 \frac{b}{n-2}
		.
	\end{equation}
	Let us denote $\lambda = b/(n-2)$ and solve the following quadratic inequality with respect to $\lambda$:
	\begin{equation}
		\label{eq:square_inequality_lambda}
		4\beta^2 \lambda^2  + \gamma_2 \lambda - 1 \geq 0
		.
	\end{equation}
	The discriminant is $\gamma_2^2 + 16 \cdot \beta^2$,
	so we obtain the following estimate:
	\begin{equation}
		\frac{b}{n-2} = \lambda \geq \frac{-\gamma_2 + \sqrt{\gamma_2^2 + 16 \cdot \beta^2}}{8\beta^2}
		.
	\end{equation}
	Calculating the expression above for our $\beta$ and $\gamma_2$, we conclude the proof.
\end{proof}

\begin{corollary}
	If $n\geq 4$, then $d(2,n) > \frac{5}{11} n$.
\end{corollary}

\begin{proof}
	For $4 \leq n \leq 21491$, the claim follows from Observation~\ref{obs:4_leq_n_leq_21491} immediately.
	For $n > 21491$, the inequality $0.45557(n-2) > \frac{5}{11}n$ holds.
\end{proof}


We can improve the constant in Theorem~\ref{thm:main_estimate} if new exact values of $d(2,n)$ are found;
however, the obtained constant is bounded by $3^{1/4}\cdot2^{-3/2}$.
To be precise, we have the following

\begin{theorem}
	For every $\varepsilon > 0$ there exists a number $n_0$ such that the inequality
	\begin{equation}
		d(2,n) \geq n\cdot(3^{1/4}\cdot2^{-3/2} - \varepsilon)
	\end{equation}
	holds for every $n>n_0$.
\end{theorem}



\section{Constructing integral point sets}

\begin{definition}
	A set $M\in\mathfrak{M}(2,n)$ is called \textit{facher}
	if $M$ consists of $n-1$ points on a straight line
	and one point out of the line.
\end{definition}

\begin{definition}
	A set $M\in\mathfrak{M}(m,n)$ is called \textit{optimal}
	if $\operatorname{diam}M=d(m,n)$.
\end{definition}

\begin{definition}
	\cite{kurz2005characteristic}
	Let $M$ be a planar integral point set.
	A squarefree number $q$ is called the \textit{characteristic} of $M$,
	if for any points $M_1, M_2, M_3 \in M$ the area of triangle $M_1 M_2 M_3$
	is $p_{1,2,3}\sqrt{q}$ for some rational $p_{1,2,3}$.
\end{definition}
For given integral point set, the characteristic is determined uniquely.

Facher sets are the simplest ones of planar integral point sets.
It is shown that for $9\leq n \leq 122$ all the optimal sets are facher~\cite{kurz2008minimum}.
For every cardinality $n$ and every characteristic $q$
there exists a facher set $M\in\mathfrak{M}(2,n)$ with characteristic $q$~\cite[Theorem 5]{our-vmmsh-2018}.
In~\cite{antonov2008maximal}, the facher sets of characteristic 1 were investigated; they were called \textit{semi-crabs}.

For every integer $n\geq 3$ Solymosi presented~\cite{solymosi2003note} a construction of a facher integral point set
$M\in\mathfrak{M}(2,n)$
such that equality $|M_1 M_2| = 2$ holds for some $M_1, M_2 \in M$.
The constructed set has both odd and even distances.

Now we will improve the Solymosi's result.

\begin{construction}
	\label{con:planar_set_with_minimeter_1}
	Let us choose a positive integer $k > 1$ and take
	\begin{equation}
		a = 2^{2^k} - 1
		.
	\end{equation}
	Then
	\begin{equation}
		\label{eq:a_equiv_3_mod_4}
		a \equiv 3 \mod 4
	\end{equation}
	and, moreover,
	\begin{multline}
		a = \left(2^{2^{k-1}}\right)^2 - 1
		=
		\left(2^{2^{k-1}} + 1\right) \left(2^{2^{k-1}} - 1\right)
		=
		\\=
		\left(2^{2^{k-1}} + 1\right) \left(2^{2^{k-2}} + 1\right) \cdot ... \cdot \left(2^{2^1} + 1\right) \left(2^2 - 1\right)
		.
	\end{multline}
	We take $d_j = 2^{2^j} + 1$ for $1 \leq j \leq k-1$.
	Then $d_j \equiv 1 \mod 4$.

	Let $c_J = \prod_{j\in J} d_j$ for every subset of indexes $J\subset I = \{1,2,...,k-1\}$
	(and  $c_\varnothing = 1$).
	We get
	\begin{equation}
		\label{eq:c_J_equiv_1_mod_4}
		c_J\equiv 1 \mod{4}
	\end{equation}
	and, moreover, $a$ is divisible by $c_J$.


	From~\eqref{eq:a_equiv_3_mod_4} and~\eqref{eq:c_J_equiv_1_mod_4} we obtain $a/c_J \equiv 3 \mod 4$.
	Let $b_J = (c_J - a/c_J)/2$, then $b_J \equiv 1 \mod 2$.
	Let us further take $g_J = (c_J + a/c_J)/2$, then $g_J \equiv 0 \mod 2$.

	We will now define the coordinates of the points as following:
	\begin{equation}
		M_{J\pm} =\left(\pm\frac{b_J}{2}, 0\right)
		,
		~~
		N   =\left(0, \frac{\sqrt{a}}{2}\right)
		.
	\end{equation}
	Then the distances are:
	\begin{multline}
		|N M_{J\pm}|
		=
		\left(\frac{b_J^2}{4} + \frac{a}{4}\right)^{1/2}
		=
		\frac{1}{2}\left(\left(\frac{c_J - a/c_J}{2}\right)^2 + a\right)^{1/2}
		=
		\\=
		\frac{1}{2}\left( \left(\frac{c_J}{2}\right)^2 - \frac{a}{2} + \left(\frac{a}{c_j}\right)^2 + a\right)^{1/2}
		=
		\frac{1}{2}\left( \left(\frac{c_J}{2}\right)^2 + \frac{a}{2} + \left(\frac{a}{c_j}\right)^2    \right)^{1/2}
		=
		\\=
		\frac{1}{2}\left(\left(\frac{c_J + a/c_J}{2}\right)^2\right)^{1/2}
		=
		\frac{1}{2}\left(\frac{c_J + a/c_J}{2}\right)
		=
		\frac{g_J}{2}
		\in\mathbb{N}
		,
	\end{multline}
	\begin{equation}
		|M_{J_1 \pm}  M_{J_2 \pm}|
		=
		\left|\frac{b_{J_1}}{2} \pm \frac{b_{J_2}}{2}\right|
		=
		\left|\frac{b_{J_1} \pm b_{J_2}}{2}\right|
		\in\mathbb{N}
		.
	\end{equation}
	In particular, for $H = \{k-1\}$ we obtain $C_H = 2^{2^{k-1}}+1$ and
	\begin{equation}
		b_H =
		\left( 2^{2^{k-1}}+1 - \frac{a}{2^{2^{k-1}}+1} \right)/2
		=
		\left(2^{2^{k-1}}+1 - \left(  2^{2^{k-1}}-1 \right) \right)/2
		=
		1
		.
	\end{equation}
	Then, one of the distances is
	\begin{equation}
		|M_{H+}  M_{H-}|
		=
		\left|\frac{b_{H}}{2} - \frac{-b_{H}}{2}\right|
		=
		\left|\frac{1}{2} - \frac{-1}{2}\right|
		= 1
		.
	\end{equation}

	Note that all the points $M_{J\pm}$ are distinct:
	the equality $b_J =  b_K$ implies $J=K$;
	the equality $b_J = -b_K$ implies $c_J = -c_K$ or $c_J = a / c_K$.
	The first case is impossible because both $c_J$ and $c_K$ are positive;
	the second case contradicts with~\eqref{eq:a_equiv_3_mod_4} and~\eqref{eq:c_J_equiv_1_mod_4}.


	So, $M = \{ M_{J\pm}, N\}$ is indeed a planar integral point set of $2^k+1$ points,
	and distance 1 occurs in $M$.
	Reminding that $k$ can be taken arbitrary large, we conclude that we can construct
	a planar integral point set of arbitrary large count of points so that the distance 1 occurs in that set.
\end{construction}

\begin{remark}
	Applying Construction~\ref{con:planar_set_with_minimeter_1} to $k=1$ naively,
	we get $a = 3$, $I=\varnothing$, $b_\varnothing = -1$ and then
	obtain a right triangle of side length 1.
	This triangle is obviously the optimal set in $\mathfrak{M}(2,3)$.
\end{remark}

\begin{remark}
	For $k=2$ in Construction~\ref{con:planar_set_with_minimeter_1},
	we obtain the optimal set in $\mathfrak{M}(2,5)$ presented in~\cite[Fig. 1]{harborth1993upper}.
	If we truncate one point from it,
	then we get one of the two optimal sets in $\mathfrak{M}(2,4)$.
\end{remark}

\begin{conjecture}
	\label{con:no_optimal_with_edge_1}
	Every set $M\in\mathfrak{M}(2,n)$, $n\geq 6$, such that for some $M_1,M_2$ equality $|M_1 M_2|=1$ holds,
	is not optimal.
\end{conjecture}

The motivation of Conjecture~\ref{con:no_optimal_with_edge_1} is based on the results of~\cite[Section 5]{kurz2008minimum},
especially the following theorem.

\begin{theorem}
	For every $12 \leq n \leq 122$, there exist a facher optimal sets $M_n\in\mathfrak{M}(2,n)$
	that consist of $n-1$ points $\{ (b_1,0), ..., (b_{n-1},0)\}$
	and the point $(0,\sqrt{A_n})$,
	where $b_1,...,b_{n-1}, A_n$ are integer and $\sqrt{A_n}\notin\mathbb{N}$.
\end{theorem}
Such a set can not contain distance 1: due to Theorem~\ref{thm:minimeter_1_planar} below,
the points with distance 1 should be $(\pm 1/2, 0)$.




%Now we need two more definitions.
\begin{definition}
	\cite{antonov2008maximal}
	A set $M\in \mathfrak{M}(m,n)$ is \textit{maximal},
	if there is no set $M'\in \mathfrak{M}(m,n+1)$
	such that $M \subsetneq M'$.
\end{definition}

%\begin{definition}
%	A set $M\in \mathfrak{M}(m,n)$ is \textit{submaximal},
%	if there is only one maximal integral point set $M'\in \mathfrak{M}(m,n+1)$,
%	such that $M \subset M'$.
%\end{definition}



If the take $k=1$ in Lemma~\ref{lem:2k-1_segments},
we obtan the following result (which is exactly~\cite[Lemma 3]{our-vmmsh-2018}).
\begin{corollary}
	\label{cor:only_one_distance_1_on_straight_line}
	Let $M \in \mathfrak{M}(2,n)$ for some $n$ and let $m$ be a straight line.
	Then there is at most one pair of points $M_1,M_2\in M \cap m$
	such that $|M_1 M_2| = 1$.
\end{corollary}

To describe all the sets $M \in \mathfrak{M}(2,n)$ with distance 1,
we will need~\cite[Proposition 6]{our-vmmsh-2018}.
For the sake of completeness, we will give here a slightly rephrased version of it with the proof as
\begin{lemma}
	\label{lem:no_4_points_in_semigeneral_position_with_distance_1}
	Let $M=\{M_1,M_2,M_3,M_4\}\in \mathfrak{M}(2,4)$ such that $|M_1 M_2|=1$
	and let $l$ be the straight line containing points $M_1$ and $M_2$.
	Then one of the points $M_3$ or $M_4$ belongs to $l$.
\end{lemma}

\begin{proof}
	Let $m$ denote the bisector of line segment $M_1 M_2$.
	Due to the triangle inequality, there are two possibilities to place each of points $M_i$, $i=3,4$:

	a) $M_i$ belongs to $l$ and $|M_i M_1|-|M_i M_2| = \pm 1$;

	b) $M_i$ belongs to $m$ and $|M_i M_1| = |M_i M_2|$.

	If (a) is true for both points, then $M\subset l$ and thus $M$ is not an integral point set.
	If (a) is true for one point and (b) is true for another, then the claim of the lemma follows.
	So, suppose the contrary, i.e. both points $M_3$ and $M_4$ belong to $m$.

	The area of the triangle $M_1 M_3 M_4$ is rational because $|M_3 - M_4| \in \mathbb{Z}$.
	Thus, the set $M$ is of characteristic 1,
	and there is a Cartesian coordinate system such that $M_1=(-1/2,0)$, $M_2=(1/2,0)$, $M_3=(0, a/2)$.
	(TODO: ref?)
	It is clear that $a\neq 0$.
	We set $b = |M_1 - M_3|$ and $O=(0,0)$.
	Applying the Pythagorean theorem to triangle $OM_1M_3$, we obtain the following Diophantine equation:
	\begin{equation}
		\frac{1}{4} + \frac{a^2}{4} = b^2
		,
	\end{equation}
	or, which is the same,
	\begin{equation}
		1 + a^2 = (2b)^2
		,
	\end{equation}
	that has no integral solutions.
	This contradiction concludes the proof.
\end{proof}

Using Construction~\ref{con:planar_set_with_minimeter_1},
Lemma~\ref{lem:no_4_points_in_semigeneral_position_with_distance_1} and Corollary~\ref{cor:only_one_distance_1_on_straight_line}
together with results of~\cite[Section 6]{antonov2008maximal},
we obtain the following

\begin{theorem}
	\label{thm:minimeter_1_planar}
	For every $n\geq 3$ there is a planar intergal set $M$ of $n$ points
	such that for some $M_1,M_2 \in M$ equality $|M_1 M_2|=1$ holds.
	This set consists of $n-1$ points, including $M_1$ and $M_2$, on a straight line and one point out of the line.

	And vice versa, if $M$ is a planar integral point set of $n$ points
	such that for some $M_1,M_2 \in M$ equality $|M_1 M_2|=1$ holds,
	then $M$ consists of $n-1$ points, including $M_1$ and $M_2$, on a straight line,
	and one point out of the line, on the bisector of line segment $M_1 M_2$.
	There is only one maximal integral point set $M' \supseteq M$,
	and the bisector is the axis of symmetry for $M'$.
	Moreover, if $n > 3$, then the distance 1 occurs in $M$ (and $M'$) only once.
\end{theorem}

Using the ``blowing up'' procedure described in~\cite[theorem 1.3]{kurz2008bounds},
we can construct an integral point set $M$ in $m$-dimensional Euclidean space, $m\geq 3$,
with distance 1 occuring in it.
If we have an integral point set with distance 1 occuring,
then we can easily dilate it to turn 1 into the desired distance.
These facts, together with Theorem~\ref{thm:minimeter_1_planar}, give the following

\begin{theorem}
	For arbitrary integers $m \geq 2$, $n \geq m+1$, $d \geq 1$
	there exists $M\in\mathfrak{M}(m,n)$,
	such that for some $M_1, M_2\in M$ equality $|M_1 M_2| = d$ holds.
\end{theorem}


\begin{definition}
	\cite{noll1989nclusters}
	We will call an integral point set $M$ \textit{prime}, if the greatest common divisor
	of all the distances occuring in $M$ is 1.
\end{definition}

If an integral point set is prime,
then it cannot be squashed to an integral point set of the same power and structure but smaller diameter.

\begin{conjecture}
	\label{hyp:prime_planar}
	For every $m \geq 2$, $n \geq m+1$, $d \geq 1$ there exists an \textbf{prime}
	integral point set $M\in\mathfrak{M}(m,n)$ which
	contains points $M_1$ and $M_2$ such that distance between $M_1$ and $M_2$ is exactly $d$.
\end{conjecture}

For $m \geq 3$, Conjecture~\ref{hyp:prime_planar} can be easily proved:
indeed, we can just take a facher integral point set according to Construction~\ref{con:planar_set_with_minimeter_1}
with sufficiently large height and then apply the ``blowing up'' procedure described in~\cite[Theorem 1.3]{kurz2008bounds},
replacing the out-of-line point with an $(m-2)$-dimensional simplex of proper side length.
That simplex will consist of $m-1$ points;
we can throw out up to all points $M_J$, except the two points with distance 1.
So, the set consists of $m+1$ or more points and has distance 1 occuring in it.
As distance 1 occurs in the obtained set, it is prime.

So, Conjecture~\ref{hyp:prime_planar} needs proof only for $m=2$.
For $m>2$, a bit more strong result can also be claimed.

\begin{theorem}
	For arbitrary integers $m > 2$, $n \geq m+1$, $d \geq 1$
	there exists a prime set $M\in\mathfrak{M}(m,n)$,
	such that for some $M_1, M_2\in M$ equality $|M_1 M_2| = d$ holds,
	the distance $d$ is minimal and occurs in $M$ only once.
\end{theorem}

\begin{proof}
	Exploit Construction~\ref{con:planar_set_with_minimeter_1} to get $M'\in \mathfrak{M}(2,n-m+2)$
	with $M'_1, M'_2 \in M'$ and $|M'_1 M'_2| = 1$.
	Dilate $M'$ to $M''$ in such a way that $M'_1$ and $M'_2$ turn into $M''_1$ and $M''_2$ resp.
	and $|M''_1 M''_2| = d$.
	Then ``blow up'' $M''$ to $M\in \mathfrak{M}(m,n)$ using a simplex of side length $d+1$.
	The greatest common divisor of $d$ and $d+1$ is $1$, so $M$ is prime.
\end{proof}


\section{Final remarks and open problems}

\begin{remark}
	The estimate of Theorem~\ref{thm:main_estimate} is not tight.
	The derivation of tight bounds for the minimum diameter $d(2, n)$
	is still a challenging task for the future~\cite[Section 7]{kurz2008minimum}.
\end{remark}
TODO: should I give the known upper bound here?


\section{Language issues to be resolved before publication}

\begin{enumerate}
	\item
		there are at most $2k-1$ distinct segments $\Delta_i$

		different vs. distinct ?

	\item
		Summing up

		is it correct to say so ?

	\item
		... the following \\ Theorem X.

		is it correct to split the sentence in such a way?

	\item
		For the sake of completeness.

		Is the phrase correct?
		There should be an example in one of papers about Banach limits...

	\item
		to belong vs. to be situated

	\item
		intersection of ... and ...
		OR
		intersection of ... with ...
		?

\end{enumerate}

\section{Acknowledgements}
Author thanks Prof. E.M. Semenov for the fruitful discussion and ideas
and A.S. Chervinskaia for the idea of using the term ``facher'' and other linguistic advices.


\printbibliography

\end{document}
