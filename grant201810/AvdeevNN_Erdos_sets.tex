\documentclass[a4paper,12pt]{article}

\usepackage[T2A]{fontenc}
\usepackage[utf8]{inputenc}


\usepackage[backend=biber,style=gost-numeric,sorting=none]{biblatex}
\addbibresource{../common/notmy.bib}
\addbibresource{../common/my.bib}

%\usepackage{cite}

\usepackage{amsthm}
\newtheoremstyle{VMMSHtheorem}% name of the style to be used
	{\topsep}% measure of space to leave above the theorem. E.g.: 3pt
	{\topsep}% measure of space to leave below the theorem. E.g.: 3pt
	{\sl}% name of font to use in the body of the theorem
	{\parindent}% measure of space to indent
	{\bfseries}% name of head font
	{.}% punctuation between head and body
	{ }% space after theorem head; " " = normal interword space
	{\thmname{#1}\thmnumber{ #2}\thmnote{ (#3)}}

\theoremstyle{VMMSHtheorem}

\newtheorem{theorem}{Теорема}
\newtheorem*{theorem*}{Теорема}
\newtheorem{lemma}[theorem]{Лемма}
\newtheorem*{lemma*}{Лемма}
\newtheorem{proposition}[theorem]{Утверждение}
\newtheorem*{proposition*}{Утверждение}
\newtheorem{corollary}[theorem]{Следствие}
\newtheorem*{corollary*}{Следствие}
\newtheorem{hypothesis}[theorem]{Гипотеза}
\newtheorem*{hypothesis*}{Гипотеза}

\newtheoremstyle{VMMSHremark}% name of the style to be used
	{\topsep}% measure of space to leave above the theorem. E.g.: 3pt
	{\topsep}% measure of space to leave below the theorem. E.g.: 3pt
	{}% name of font to use in the body of the theorem
	{\parindent}% measure of space to indent
	{\sc}% name of head font
	{.}% punctuation between head and body
	{ }% space after theorem head; " " = normal interword space
	{\thmname{#1}\thmnumber{ #2}\thmnote{ (#3)}}

\theoremstyle{VMMSHremark}
\newtheorem{definition}[theorem]{Определение}
\newtheorem*{definition*}{Определение}
\newtheorem{designation}[theorem]{Обозначение}
\newtheorem*{designation*}{Обозначение}
\newtheorem{remark}[theorem]{Замечание}
\newtheorem*{remark*}{Замечание}
\newtheorem{example}[theorem]{Пример}
\newtheorem*{example*}{Пример}









\usepackage{amsmath, amsfonts, amssymb}
\usepackage{amscd}
\usepackage[english,russian]{babel}


\sloppy

\begin{document}



{
\center{МНОЖЕСТВА ТОЧЕК С ЦЕЛОЧИСЛЕННЫМИ РАССТОЯНИЯМИ НА ПЛОСКОСТИ И В ЕВКЛИДОВОМ ПРОСТРАНСТВЕ} % обязательное поле!
}

\vspace{2cm}

Множеством с целочисленными расстояниями будет называть такое подмножество плоскости $\mathbb{R}^2$,
не содержащееся ни в какой прямой,
что расстояние между любыми двумя точками $M$ есть целое число.

В \cite{anning1945integral,erdos1945integral} доказано, что любое МЦР конечно.
Следовательно, для МЦР естественным образом определяется диаметр:
$$
	diam(M) = \max_{A,B\in M} |A-B|,
$$
где $|A-B|$~--- обычное расстояние между $A$ и $B$.

Пусть $\mathfrak{M}_n$~--- множество всех МЦР, состоящих ровно из $n$ точек.
В~\cite{harborth1993upper} была рассмотрена функция
$$
	f(n) = \min_{M\in\mathfrak{M}_n} diam(M)
$$
и найдена верхняя оценка на эту функцию.
Кроме того, там же вычислены значения $f(n)$ для первых нескольких $n$.

В~\cite{solymosi2003note} была доказана нижняя оценка на $f(n)$:
\begin{equation}\label{Solymosi}
f(n) \geqslant \frac{1}{24} n
.
\end{equation}
В~\cite{our-mz-rus} коэффициент при $n$ был улучшен до 1/8.

Планируется:
\begin{itemize}
\item
ещё более улучшить оценку~\eqref{Solymosi};
\item
изучить связь задачи об отыскании $f(n)$ и задачи об упаковке точек в квадрат~\cite{locatelli2002packing};
\item
построить частные случаи оценки~\eqref{Solymosi}:
например, для МЦР, содержащихся в окружности, гиперболе и семействе софокусных гипербол;
\item
изучить такие МЦР $M\in\mathfrak{M}_n$, что $diam(M)=f(n)$.
\end{itemize}




\printbibliography



\end{document}
