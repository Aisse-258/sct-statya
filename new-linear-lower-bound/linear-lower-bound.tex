\documentclass[a4paper,14pt]{article} %размер бумаги устанавливаем А4, шрифт 12пунктов
\usepackage[T2A]{fontenc}
\usepackage[utf8]{inputenc}
\usepackage[english,russian]{babel} %используем русский и английский языки с переносами
\usepackage{amssymb,amsfonts,amsmath,mathtext,enumerate,float,amsthm} %подключаем нужные пакеты расширений
\usepackage[pdftex,unicode,colorlinks=true,citecolor=black,linkcolor=black]{hyperref}
%\usepackage[pdftex,unicode,colorlinks=true,linkcolor=blue]{hyperref}
\usepackage{indentfirst} % включить отступ у первого абзаца
\usepackage[dvips]{graphicx} %хотим вставлять рисунки?
\graphicspath{{illustr/}}%путь к рисункам

\makeatletter
\renewcommand{\@biblabel}[1]{#1.} % Заменяем библиографию с квадратных скобок на точку:
\makeatother %Смысл этих трёх строчек мне непонятен, но поверим "Запискам дебианщика"

\usepackage{geometry} % Меняем поля страницы.
\geometry{left=4cm}% левое поле
\geometry{right=1cm}% правое поле
\geometry{top=2cm}% верхнее поле
\geometry{bottom=2cm}% нижнее поле

\renewcommand{\theenumi}{\arabic{enumi}}% Меняем везде перечисления на цифра.цифра
\renewcommand{\labelenumi}{\arabic{enumi}}% Меняем везде перечисления на цифра.цифра
\renewcommand{\theenumii}{.\arabic{enumii}}% Меняем везде перечисления на цифра.цифра
\renewcommand{\labelenumii}{\arabic{enumi}.\arabic{enumii}.}% Меняем везде перечисления на цифра.цифра
\renewcommand{\theenumiii}{.\arabic{enumiii}}% Меняем везде перечисления на цифра.цифра
\renewcommand{\labelenumiii}{\arabic{enumi}.\arabic{enumii}.\arabic{enumiii}.}% Меняем везде перечисления на цифра.цифра

\sloppy


\renewcommand\normalsize{\fontsize{14}{25.2pt}\selectfont}

\usepackage[backend=biber,style=gost-numeric,sorting=none]{biblatex}
\addbibresource{../common/notmy.bib}
\addbibresource{../common/my.bib}


\begin{document}
\renewcommand{\bibname}{Список цитированной литературы}
\renewcommand\refname{\bibname}
% !!!
% Здесь начинается значащий текст
% Всё, что выше - беллетристика

Avdeev N.N, Semenov E.M.

Better linear lower bound for integral point sets


Это НЕ статья, это просто доказательство.
Имеются вольности языка.

\paragraph{Определение.}
Системой Эрдёша называется множество точек $\{M_1, M_2, ..., M_n\}$ на плоскости, не содержащееся ни в какой прямой,
такое, что для любых $i\neq j$ расстояние $|M_i M_j| \in \mathbb{N}$,
т.е. является натуральным числом.

\paragraph{Лемма 1.}
В системе Эрдёша $S$ не может быть точек $M_1$, $M_2$, $M_3$, $M_4$,
лежащих на одной прямой $m$ и таких, что $|M_1 M_2| = |M_3 M_4| = 1$.

\paragraph{Доказательство.}
Обозначим через $m_{12}$ и $m_{34}$ серединные перпендикуляры к отрезкам $M_1 M_2$ и $m_3 M_4$ соответственно.

Пусть точка $M\in S$.
Тогда либо $|MM_1| - |MM_2| = 0$ и $M\in m_{12}$, либо $\left||MM_1| - |MM_2|\right| = 1$ и $M\in m$,
т.е. в любом случае $M\in m \cup m_{12}$.
Аналогично $M\in m \cup m_{34}$.
Следовательно, $M\in (m \cup m_{12}) \cap (m \cup m_{34}) = m \cup (m_{12} \cap m_{34}) = m$,
т.к. $m_{12} \cap m_{34} = \varnothing$ (перпендикуляры к одной прямой, проведённые в разных точках, не пересекаются между собой).

В силу произвольности выбора $M \in S$ получаем $S \subset m$, чего быть не должно.


\paragraph{Следствие 1.1.}
Пусть $S$~--- система Эрдёша, $\mathop{diam} S = d$.
Тогда ни на какой прямой не лежит более $(d+3)/2$ точек из $S$.

\paragraph{Доказательство.}
Предположим проивное.
Очевидно, что на прямой $m$ может быть не более $d+1$ точек системы $S$.

Пусть сначала $d$ чётно.
Тогда из каждой пары соседних точек, кроме, быть может, одной пары,
нужно выбросить хотя бы одну точку в силу леммы 1.
Итого мы выбросим не менее $d/2$ точек,
останется не более $(d+2)/2$ точек.

Пусть теперь $d$ нечётно.
Тогда нужно выбросить не менее $(d-1)/2$ точек.
Останется не более $(d+3)/2$ точек.

\paragraph{Лемма 2.}
Систему Эрдёша $S$ диаметром $d$ можно вместить в квадрат со стороной $d$.

\paragraph{Доказательство.}
Пусть $M_1, M_2 \in S$, $|M_1 M_2| = d$,
т.е. диаметр системы $S$ достигается на $M_1$  $M_2$.
Введём прямоугольную декартову систему координат на плоскости таким образом, что
$M_1 = (-d/2; 0)$, $M_1 = (d/2; 0)$.

Тогда прочие точки $M_i \in S$ имеют координаты $M_i=(x_i, y_i)$.
Очевидно, что для $i>2$ имеем $|x_i| < d/2$
(иначе $|M_i M_1| > d$ или $|M_i M_2| > d$).
Более того, $|y_i| < d$ (иначе $|M_i M_1| > d$ или $|M_i M_2| > d$).

Пусть $y_+ = \max_{i} y_i$, $y_- = \min_{i} y_i$, $M_+=(x_+, y_+)$, $M_-=(x_-, y_-)$
(если максимум досигается на нескольких точках, возьмём любые).
Тогда
\begin{multline}
	d \geq |M_+ M_-| = \sqrt{(x_+ - x_-)^2 + (y_+ - y_-)^2}
	\geq \sqrt{(y_+ - y_-)^2} = y_+ - y_-
\end{multline}
Отсуюда следует, что $S$ можно вместить в квадрат со сторонами $x=\pm d/2$,
$y=y_+$, $y=y_+ - d$.

\paragraph{Примечание.}
Скорее всего, улучшить оценку на сторону квадрата, сделав её меньше диаметра,
не получится: часто встречаются конструкции систем Эрдёша, расположенных на окружности
(Ссылка!!!).

\paragraph{Ремарка.}
Терминология местами диковата,
но я на неё не особо заморачиваюсь,
потому что всё равно потом переводить на английский.

\paragraph{Определение.}
Крестом точек $M_1$ и $M_2$ будем называть объединение прямой,
проходящей через эти точки,
и серединного перпендикуляра к отрезку $M_1 M_2$
и обозначать $cr(M_1,M_2)$.

\paragraph{Ремарка.}
Возможно, в финальной версии мы крест уберём, но пока пусть будет.
Без него формулировать утверждение 3 и лемму 4 вообще как-то грустно.

\paragraph{Обозначение.}
Количесво точек в множестве $P$ будем обозначать как $\#P$
и считать его равным $\infty$, если $P$ не менее чем счётно.

Вроде как в англоязычной литературе это стандартное обозначение.

\paragraph{Утверждение 3.}
Если интервалы $M_1 M_2$ и $M_3 M_4$ не пересекаются,
$M_1 \neq M_2$, $M_3 \neq M_4$,
то $cr(M_1,M_2) \cap cr(M_3,M_4)$~--- либо не более 4 точек, либо прямая.

\paragraph{Доказательство.}
Пересечением $cr(M_1,M_2) \cap cr(M_3,M_4)$ может быть либо от 2 до 4 точек, либо прямая,
либо, если эти кресты совпадают, их пересечение совпадает с каждым из эих крестов.
Но $cr(M_1,M_2) \neq cr(M_3,M_4)$, потому что середины отрезков $M_1 M_2$ и $M_3 M_4$
совпасть не могут, т.к. интервалы $M_1 M_2$ и $M_3 M_4$ не пересекаются по условию.

\paragraph{Лемма 4.}
Пусть $S$~--- система Эрдёша,
$M_1, M_2, M_3, M_4 \in S$,
$M_1 \neq M_2$, $M_3 \neq M_4$,
интервалы $M_1 M_2$ и $M_3 M_4$ не пересекаются,
$d = \mathop{diam} S > 5$.
Тогда $\#S \leq 4 \cdot |M_1 M_2| \cdot |M_3 M_4| + \frac{d-5}{2}$.

\paragraph{Доказательство.}
Если $\#(cr(M_1, M_2) \cap cr(M_3, M_4)) < \infty$,
то (ссылка на Солимоси!)
$\#S \leq 4 \cdot |M_1 M_2| \cdot |M_3 M_4|$.
Иначе (в силу утверждения 3) $cr(M_1, M_2) \cap cr(M_3, M_4) = m$,
где $m$~--- прямая.
В силу следствия 1.1 на этой прямой лежит не более $(d+3)/2$ точек;
кроме того, из общего количества точек нужно вычесть 4,
которые бы дали эти 2 креста в случае дискретного переечения.
Получаем верхнюю оценку
\begin{equation}
	\#S \leq 4 \cdot |M_1 M_2| \cdot |M_3 M_4| - 4 + \frac{d+3}{2}
	=
	4 \cdot |M_1 M_2| \cdot |M_3 M_4| + \frac{d-5}{2},
\end{equation}
что и требовалось доказать.

\paragraph{Обозначение.}
Через $\varphi_k$ будем обозначать максимально возможный минимум попарных расстояний между $k$ точками,
упакованными в единичный квадрат (ссылка!!).

На английском это есть в статье и звучит как-то адекватно.

\paragraph{Лемма 5.}
Пусть $S$~--- система Эрдёша,
$d = \mathop{diam} S > 5$,
$k \geq 2$,
$m \geq 2$,
$ \#S = (k-1)m^2 + 2$.

Тогда
\begin{equation}
	d \geq \mu (\#S - 2),
\end{equation}
где
\begin{equation}
	\mu = \frac{\sqrt{64\varphi_k^2 (k-1)+1}-1}{16\varphi_k^2 (k-1)}
\end{equation}

\paragraph{Доказательство.}
Впишем $S$ в квадрат со стороной $d$ (см. лемму 2)
и разобьём этот квадрат на $m^2$ маленьких квадратов со стороной $d/m$.
Тогда по принципу Дирихле либо:

а) хотя бы в два маленьких квадрата $Q_1$ и $Q_2$ попало не менее чем по $k$ точек;

б) хотя бы в один маленький квадрат $Q_1$ попало не менее чем $k+1$ точка.
\\
В случае (а) в $Q_1$ выберем точки $M_1$ и $M_2$,
а в $Q_2$ выберем точки $M_3$ и $M_4$
так, что $|M_1 M_2| \leq \varphi_k d /m$, $|M_3 M_4| \leq \varphi_k d/m$
(это всегда можно сделать в силу определения $\varphi_k$).

В случае (б) в $Q_1$ выберем точки $M_1$ и $M_2$ так, что
$|M_1 M_2| \leq \varphi_{k+1} d /m \leq \varphi_k d /m$.
Из всех точек, кроме $M_1$, выберем $M_3$ и $M_4$ так, что
$|M_3 M_4| \leq \varphi_k d/m$.
Возможное совпадение $M_2$ и $M_3$ не помешает дальнейшему доказательству.

По лемме 4
\begin{equation}
	\#S \leq 4 \left( \frac{d}{m} \varphi_k \right)^2 + \frac{d}{2} - \frac{5}{2}
\end{equation}
или, с учётом того, что $ \#S = (k-1)m^2 + 2$,
\begin{equation}
	 4 \left( \frac{d}{m} \varphi_k \right)^2 + \frac{d}{2} - \left( (k-1)m^2 + \frac{9}{2}\right) \geq 0
\end{equation}

Положим $d = \lambda m$:
\begin{equation}
	 4 \lambda^2 \varphi_k^2 + \frac{m}{2} \lambda - \left( (k-1)m^2 + \frac{9}{2}\right) \geq 0
\end{equation}

Решим это квадратное неравенство относительно $\lambda$,
зная, что $\lambda > 0$.

Честно считаем дискриминант:
\begin{multline}
	D =
	\frac{m^2}{4} + 4 \cdot 4 \varphi_k^2 \cdot \left( (k-1)m^2 + \frac{9}{2}\right)
	=
	\frac{m^2}{4} + 16 \varphi_k^2 \cdot \left( (k-1)m^2 + \frac{9}{2}\right)
	>\\>
	\frac{m^2}{4} + 16 \varphi_k^2 \cdot (k-1)m^2
	=
	\frac{m^2}{4} \cdot (1 + 64 \varphi_k^2 \cdot (k-1))
	> 0
\end{multline}

Итак,
\begin{multline}
	\frac{d}{m} = \lambda >
	\frac{-\frac{m}{2} + \sqrt{\frac{m^2}{4} \cdot (1 + 64 \varphi_k^2 \cdot (k-1))} }{8 \varphi_k^2}
	=\\=
	\frac{-m + \sqrt{m^2  (1 + 64 \varphi_k^2 \cdot (k-1))} }{16 \varphi_k^2}
	=
	m\frac{ \sqrt{ 64 \varphi_k^2 (k-1) + 1} -1 }{16 \varphi_k^2}
\end{multline}
Разделим обе части неравенства на положительное число $m(k-1)$:
\begin{equation}
	\frac{d}{(k-1)m^2}
	>
	\frac{ \sqrt{ 64 \varphi_k^2 (k-1) + 1} -1 }{16 \varphi_k^2 (k-1)}
\end{equation}
Положим
\begin{equation}
	\mu = \frac{ \sqrt{ 64 \varphi_k^2 (k-1) + 1} -1 }{16 \varphi_k^2 (k-1)},
\end{equation}
тогда
\begin{equation}
	\frac{d}{(k-1)m^2}
	>
	\mu
\end{equation}
откуда незамедлительно
\begin{equation}
	d > \mu \cdot (\#S-2)
\end{equation}
что и требовалось доказать.

Какие-то умные и добрые люди вычислили $\varphi_k$ до $k=10$ включительно (ссылка!) с точностью 4 знака после запятой.
(Умные, потому что вычислили, а добрые, потому что выложили.)

По их расчётам получаем $\varphi_{10} = 0.4214$,
что даёт оценку $d>0.3583(\#S-2)$.

Это, конечно, для $\#S$ специального вида, но множитель $(m+1)^2 / m^2$ съестся на бесконечности,
а итоговые коэфициенты и так будем брать с запасом (кто их там знает, как они округляли).

Это для $\#S \geq 41$, но для меньших мощностей посчитано (и не один раз, в том числе нами).

Кстати, для $\#S = 3$ не проходит (и не должно).

А особенная приятность в том, что вычисление новых $\phi_k$ может давать автоматическое улучшение оценки
(может, правда, и не давать).
Однако, как ни печально, константу не получится сделать даже 0.42 (есть нижняя оценка на $\varphi_k$).

Следующий прикол:
де-факто нам нужна не какая попало расстановка точек в единичном квадрате,
а расстановка с рациональными расстояниями (они потенциально могут превратиться в целые при умножении на $d/m$).
А возможность аппроксимировать произвольную расстановку расстановкой с рациональными расстояниями
уже имеет прямое отношение к проблеме Улама-Эрдёша (ссылка!!).



Просто чтобы список литературы был непуст: \cite{our-mz-rus}
\printbibliography

\end{document}
