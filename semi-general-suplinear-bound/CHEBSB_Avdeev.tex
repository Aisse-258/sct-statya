\documentclass[11pt,twoside,draft
]{article}
\usepackage{amsmath,amsfonts,amssymb,amsthm,indentfirst,enumerate,textcomp}
\usepackage[utf8]{inputenc}
\usepackage[T2A]{fontenc}
\usepackage{chebsb}
\usepackage[english,russian]{babel}
\usepackage{indentfirst, array}
\usepackage{amscd,latexsym}
\usepackage{mathrsfs}
%\usepackage[ruled, linesnumbered]{algorithm2e}
\usepackage{tabularx}
\usepackage{multirow}

\usepackage{graphicx}
\usepackage{textcase}% Оформление страниц




%%%%%%%%%%%%%%%%%%%%%%%%%%%%%%%%%%
%% Автоматическая нумерация теорем (коль скоро подключен amsthm)

% К сожалению, английского ``Definition'' в стилевом файле нет.
% Добавляю по аналогии.
\newtheorem{Definition}{\indent {\sc Definition}}
\renewcommand{\theDefinition}{\rm \arabic{Definition}}

\newtheorem{Corollary}{\indent {\sc Corollary}}
\renewcommand{\theCorollary}{\rm \arabic{Corollary}}

\newtheorem{Remark}{\indent {\sc Remark}}
\renewcommand{\theRemark}{\rm \arabic{Remark}}

%%%%%%%%%%%%%%%%%%%%%%%%%%%%%%%%%%





\label{beg}
% заглавие стати и аннотация

\levkolonttl{%левый колонтитул - авторы
~}%N.~N.~Avdeev}
\prvkolonttl{%правый колонтитул - сокращенное название статьи
On diameter bounds for planar integral point sets \ldots}

\UDK{% УДК статьи
519.146}

\DOI{
10.22405/2226-8383-2018-\tom-\iss-\pageref{beg}-\pageref{end}
}

\title
{%название статьи на русском языке с указанием источника финансирования при необходимости
Об оценках на диаметр плоских множеств с целочисленными расстояниями полуобщего положения%
~}%\footnote{Исследование выполнено за счет гранта Российского научного фонда (проект 19-11-00197).}}
{%название статьи на английском языке
On diameter bounds for planar integral point sets in semi-general position}

\author
{%авторы статьи с указанием города на русском языке
~}%Н. Н. Авдеев (г. Воронеж)}
{%авторы статьи с указанием города на английском языке
~}%N. N. Avdeev (Voronezh)}

\Cite{
%Н. Н. Авдеев
Об оценках на диаметр плоских множеств с целочисленными расстояниями полуобщего положения // Чебышевcкий сборник, 2019, т.~\tom, вып.~\iss, с.~\pageref{beg}--\pageref{end}.
}
{
%N. N. Avdeev,
2019, "On diameter bounds for planar integral point sets in semi-general position"\,, {\it Che\-by\-shev\-skii sbornik}, vol.~\tom, no.~\iss, pp.~\pageref{beg}--\pageref{end}.
}

\info
{%авторы статьи с указанием аффилиации на русском языке
%\noindent {\bf Н. Н. Авдеев}~--- Воронежский государственный университет, кафедра теории функций и геометрии
deleted
\noindent
%\emph{e-mail: nickkolok@mail.ru, avdeev@math.vsu.ru}


}
{%авторы статьи с указанием аффилиации на английском языке
%\noindent {\bf N. N. Avdeev}~--- Voronezh State University
deleted
\noindent
%\emph{e-mail: nickkolok@mail.ru, avdeev@math.vsu.ru}

}

\Abstract
{%Аннотация статьи на русском языке 150-250 слов с учетом ключевых слов
Множество точек $M$ на плоскости называется плоским множеством с целочисленными расстояниями,
если все расстояния между точками $M$ суть целые числа,
и при этом $M$ не содержится ни в какой прямой.
Говорят, что плоское множество с целочисленными расстояниями есть множество полуобщего положения,
если никакие три его точки не лежат на одной прямой.
Известная оценка снизу на диаметр плоского множества с целочисленными расстояниями
линейна относительно его мощности.
Ранее не были известны отдельные оценки снизу на диаметр плоских множеств с целочисленными расстояниями полуобщего положения
заданной мощности
(известная конструктивная оценка сверху на диаметр плоских множеств с целочисленными расстояниями
использует как раз множества полуобщего положения).
В статье доказывается надлинейная оценка снизу
на диаметр плоского множества с целочисленными расстояниями полуобщего положения
(полиномиальная с показателем $5/4$).
Доказательство основано на относительно большом количестве лемм и наблюдений,
включая результаты Солимоси
из статьи, в которой была впервые доказана линейная оценка снизу
на диаметр плоских множеств с целочисленными расстояниями.
Полученная в данной статье оценка всё ещё не точна.
}
{%Аннотация статьи на английском языке
A point set $M$ in the Euclidean plane is said to be a planar integral point set if all the distances between the
elements of $M$ are integers, and $M$ is not situated on a straight line.
A planar integral point set is said to be a set in semi-general position, if it does not contain collinear triples.
The existing lower bound for mininal diameter of a planar integral point set is linear with respect to its cardinality.
There were no known special diameter bounds for planar integral point sets in semi-general position of given cardinality
(the known upper bound for planar integral point sets is constructive
and employs planar integral point sets in semi-general position).
We prove a new lower bound for minimal diameter of planar integral point sets in semi-general position
that is better than linear (polynomial of power $5/4$).
The proof is based on several lemmas and observations, including the ones established by Solymosi
to prove the first linear lower bound for diameter of a planar integral point set.
We have to admit that our bound is still not exact.
}

\keywords
{%ключевые слова на русском языке
комбинаторная геометрия, диаметр множества, множество с целочисленными расстояниями.
}
{%ключевые слова на английском языке
combinatorial geometry, diameter of a set, integral point set.
}

%число наименований в библиографии
\Bibliography{21 название.}{21 titles.}

\begin{document}

%генерация заглавия статьи
\maketitle

\enmaketitle



\section{Introduction}



An \textit{integral point set} in a plane is a point set $M$ such that all the usual (Euclidean) distances between the
points of $M$ are integers and $M$ is not situated on a straight line.
Every integral point set consists of a finite number of points~\cite{anning1945integral,erdos1945integral};
thus, we denote the set of all planar integral point sets of $n$ points by
$\mathfrak{M}(2,n)$ (using the notation in~\cite{our-vmmsh-2018})
and define the diameter of $M\in\mathfrak{M}(2,n)$ in the following natural way:
\begin{equation}
	\operatorname{diam} M = \max_{A,B\in M} |AB|
	,
\end{equation}
where $|AB|$ denotes the Euclidean distance.
The symbol $\# M$ will be used for cardinality of $M$, that is the number of points in $M$ in our case.

Since every integral point set can obviously be dilated to a set of larger diameter,
minimal possible diameters of sets of given cardinality are of interest.
To be precise,
the following function was introduced in~\cite{kurz2008bounds,kurz2008minimum}:
\begin{equation}
	d(2,n) = \min_{M\in\mathfrak{M}(2,n)} \operatorname{diam} M
	.
\end{equation}

It turned out to be very easy to construct a planar integral point set of $n$ points with $n-1$ collinear ones and one point out of the line
(so-called \textit{facher} sets);
the same holds for 2 points out of the line (we refer the reader to~\cite{antonov2008maximal}, where some of such sets are called \textit{crabs})
and even for 4 points out of the line~\cite{huff1948diophantine}.
For $9\leq n\leq 122$, the minimal possible diameter is attained at a facher set~\cite{kurz2008bounds}.


\begin{Definition}
	A set $M\in\mathfrak{M}(2,n)$ is called to be in \textit{semi-general position},
	if no three points of $M$ are collinear.
	The set of all planar integral point sets in semi-general position
	is denoted by $\overline{\mathfrak{M}}(2,n)$.
\end{Definition}

Furthermore, the constructions of integral point sets in semi-general position of arbitrary cardinality
appeared in~\cite{harborth1993upper};
such sets are situated on a circle.
Also, there is a sophisticated construction of a circular integral point set of arbitrary cardinality
that gives the possible numbers of odd integral distances
between points in the plane~\cite{piepmeyer1996maximum}.


\begin{Definition}
	A set $M\in\overline{\mathfrak{M}}(2,n)$ is said to be a set in \textit{general position},
	if no four points of $M$ are concyclic.
	The set of all planar integral point sets in general position
	is denoted by $\dot{\mathfrak{M}}(2,n)$.
\end{Definition}

It remains unknown whether there are integral points sets in general position of arbitrary cardinality;
however, some sets $M\in \dot{\mathfrak{M}}(2,7)$ are known~\cite{kreisel2008heptagon,kurz2013constructing}.

The inequality
\begin{equation*}
	d(2,n) \leq \overline{d}(2,n) \leq \dot{d}(2,n)
	,
\end{equation*}
where
$
	\overline{d}(2,n) = \min_{M\in\overline{\mathfrak{M}}(2,n)} \operatorname{diam} M
$
and
$
	\dot{d}(2,n) = \min_{M\in\dot{\mathfrak{M}}(2,n)} \operatorname{diam} M
$,
is obvious; however, a more interesting relation holds:
\begin{equation*}
	c_1 n \leq d(2,n) \leq \overline{d}(2,n) \leq n^{c_2 \log \log n}
	.
\end{equation*}
The upper bound is established in~\cite{harborth1993upper}.
The lower bound firstly appeared in~\cite{solymosi2003note};
the largest known value for $c_1$ is $5/11$ for $n\geq 4$~\cite{my-pps-linear-bound-2019}.


There are some bounds for minimal diameter of planar integral point sets in some special positions.
Assuming that a planar integral point set contains many collinear points,
the following result holds.
\begin{Theorem}~\cite[Theorem 4]{kurz2008minimum}
	For $\delta > 0$, $\varepsilon > 0$, and $P\in\mathfrak{M}(2,n)$ with
	at least $n^\delta$ collinear points, there exists a $n_0 (\varepsilon)$
	such that for all $n \geq n_0 (\varepsilon)$ we have
	\begin{equation}
		\operatorname{diam} P \geq n^{\frac{\delta}{4 \log 2(1+\varepsilon)}\log \log n}
		.
	\end{equation}
\end{Theorem}
For diameter bounds for circular sets, we refer the reader to~\cite{bat2018number}.

Particular cases of planar integral point sets are also discussed
in~\cite[\S 5.11]{brass2006research},~\cite[\S D20]{guy2013unsolved},~\cite{our-pmm-2018},~\cite{our-ped-2018}.
For generalization to higher dimensions and the appropriate bounds, see~\cite{kurz2005characteristic,nozaki2013lower}.

In the present paper we give a special bound for planar integral point sets in semi-general position.
The condition of semi-general position is crucial in the given proof.



\section{Preliminary results}

In this section, we give some lemmas which will be used for the proof.


\begin{lemma}
	\cite[Observation 1]{solymosi2003note}
	If a triangle $T$ has integer side lengths $a \leq b \leq c$,
	then the minimal height $m$ of it is at least $\left(a - \frac{1}{4}\right)^{1/2}$.
\end{lemma}

\begin{Definition}
	The part of a plane between two parallel straight lines with distance $\rho$ between them
	is called a strip of width $\rho$.
\end{Definition}

\begin{lemma}
	\cite{smurov1998stripcoverings}
	If a triangle $T$ with minimal height $\rho$ is situated in a strip,
	then the width of the strip is at least $\rho$.
\end{lemma}

\begin{Corollary}
	\label{cor:solymosi_strip}
	If a triangle $T$ with integer side lengths $a \leq b \leq c$ is situated in a strip,
	then the width of the strip is at least $\left(a - \frac{1}{4}\right)^{1/2}$.
\end{Corollary}


\begin{lemma}
	\cite[Lemma 4]{our-vmmsh-2018};
	\cite[Lemma 2.4]{my-pps-linear-bound-2019}
	\label{lem:square_container}
	Let $M\in\mathfrak{M}(2,n)$, $\operatorname{diam} M = d$.
	Then $M$ is situated in a square of side length $d$.
\end{lemma}

\begin{Definition}
	\cite[Definition 2.5]{my-pps-linear-bound-2019}
	A \textit{cross} for points $M_1$ and $M_2$, denoted by $cr(M_1,M_2)$, is the union of two straight lines:
	the line through $M_1$ and $M_2$,
	and the perpendicular bisector of line segment $M_1 M_2$.
\end{Definition}

\begin{lemma}
	\cite[Theorem 3.10]{my-pps-linear-bound-2019}
	\label{lem:no_distance_one}
	Each set $M\in\mathfrak{M}(2,n)$
	such that for some $M_1,M_2 \in M$ equality $|M_1 M_2|=1$ holds,
	consists of $n-1$ points, including $M_1$ and $M_2$, on a straight line,
	and one point out of the line, on the perpendicular bisector of line segment $M_1 M_2$.
\end{lemma}


\begin{lemma}
	\label{lem:count_of_points_on_hyperbolas}
	Let $\{M_1, M_2, M_3, M_4\} \subset M\in\overline{\mathfrak{M}}(2,n)$
	(points $M_2$ and $M_3$ may coincide, other points are distinct), $n\geq 4$.
	Then $\# M \leq 4 \cdot |M_1 M_2| \cdot |M_3 M_4|$.
\end{lemma}

\begin{Remark}
	Lemma~\ref{lem:count_of_points_on_hyperbolas} is one of the variations of~\cite{erdos1945integral}.
\end{Remark}

\begin{proof}[Proof]
	Each point $N\in M$ satisfies one of the following conditions:

	a) $N$ belongs to $cr(M_1,M_2)$~--- overall at most 4 points;

	b) $N$ belongs to $cr(M_3,M_4)$~--- overall at most 4 points;

	c) $N$ belongs to the intersection of one of $|M_1 M_2| - 1$ hyperbolas
	with one of $|M_3 M_4| - 1$ hyperbolas~--- overall at most $4 (|M_1 M_2| - 1)(|M_3 M_4| - 1)$ points;

	Due to Lemma~\ref{lem:no_distance_one} we have $|M_1 M_2| \geq 2$ and $|M_3 M_4| \geq 2$.
	Since
	\begin{multline}
		4 (|M_1 M_2| - 1)(|M_3 M_4| - 1) + 4 + 4
		=
		4 ( (|M_1 M_2| - 1)(|M_3 M_4| - 1) + 2)
		=
		\\=
		4 ( |M_1 M_2| \cdot |M_3 M_4| + 1 - |M_1 M_2| - |M_3 M_4| + 2)
		=
		\\=
		4 ( |M_1 M_2| \cdot |M_3 M_4| + (1 - |M_1 M_2|) + (2 - |M_3 M_4|))
		\leq
		4 |M_1 M_2| \cdot |M_3 M_4|
		,
	\end{multline}
	the assertion follows.
\end{proof}


\section{The main result}

\begin{Theorem}
	\label{thm:main_result}
	For every integer $n \geq 3$ we have
	\begin{equation}
		\overline{d}(2,n) \geq (n/5)^{5/4}
		.
	\end{equation}
\end{Theorem}

\begin{proof}[Proof]
	For $n = 3$ we have $\overline{d}(2,3) = 1$.
	Consider $M\in\overline{\mathfrak{M}}(2,n)$, $n \geq 4$, $\operatorname{diam} M = p$.

	Let us choose points $M_1, M_2, M_3, M_4 \in M$
	(points $M_2$ and $M_3$ may coincide, other points are distinct), such that
	\begin{equation}
		\min_{A, B \in M} |AB| = |M_1 M_2|
		,
	\end{equation}
	\begin{equation}
		\min_{A, B \in M \setminus \{M_1\}} |AB| = |M_3 M_4| = m
		.
	\end{equation}

	For $m \leq p^{2/5}$, Lemma~\ref{lem:count_of_points_on_hyperbolas} yields that
	\begin{equation}
		n \leq 4 \cdot |M_1 M_2| \cdot |M_3 M_4| \leq  4 p^{4/5}
		.
	\end{equation}
	Equivalently,
	\begin{equation}
		\label{eq:hyperbolas_5_4}
		p \geq (n/4) ^ {5/4} > (n/5) ^ {5/4}
		.
	\end{equation}

	So, let us consider $m > p^{2/5}$.
	Then for any $A,B \in M\setminus\{M_1\}$ the inequality $|AB| > p^{2/5}$ holds.
	Due to Corollary~\ref{cor:solymosi_strip}, no three points of $M\setminus\{M_1\}$
	are located in a strip of width $p^{1/5} / 2$.

%TODO: Solymosi's proof accepts a=b; in our case, a=b leads to a better bound (?)
% and the condition a < b may (??) remove -1/4 from the lemma about triangle


	Lemma~\ref{lem:square_container} yields that $M$ is situated in a square with side length $p$.
	Let us partition this square into $q$ strips, $2p^{4/5} \leq q < 2p^{4/5} + 1$, each of width at most $p^{1/5} / 2$.
	Every strip contains at most two points of  $M\setminus\{M_1\}$,
	thus
	\begin{equation}
		\label{eq:strips_4_5}
		n \leq 2(2p^{4/5} + 1) + 1
		= 4p^{4/5}+3
		\leq 5 p^{4/5}
		.
	\end{equation}
	The latter inequality holds because $\overline{d}(2,n) \geq 4$ for $n\geq 4$~~\cite{kurz2008minimum}
	% TODO: add two more links from the kurz2008minimum's intro in case of lack of references
	and $4^{4/5}>3$.
	From the inequality~\eqref{eq:strips_4_5} one can easily derive that
	\begin{equation}
		\label{eq:strips_5_4}
		p \geq (n/5) ^ {5/4}
		.
	\end{equation}
\end{proof}



\begin{Remark}
	The following result appeared in~\cite{solymosi2003note} as Corollary 1:
\end{Remark}

\begin{lemma}
	For $H \in \overline {\mathfrak{M}}(2,n)$, the minimal
	distance in H is at least $n^{1/3}$.
\end{lemma}
Applying the same technique, one can easily derive that
\begin{equation}
	n \leq 3 \frac{\operatorname{diam} H }{n^{1/6}}
	,
\end{equation}
which leads to the bound
\begin{equation}
	\overline{d}(2,n) \geq c_3 n^{7/6}
	,
\end{equation}
which is less than the one from Theorem~\ref{thm:main_result}.


\section{Conclusion}
The presented bound is the first special lower bound for sets in semi-general position.
Thus, we have not accepted the challenge to make the constant in Theorem~\ref{thm:main_result} as large as possible,
in order to keep the ideas of the proof clear and understandable.
The more thorough research can be done in the future to improve the constant.
However, the upper and lower bounds are still not exact.

\section{Acknowledgements}
Author thanks Dr. Prof. E.M. Semenov for proofreading and valuable advices and Dr. A.S. Usachev for proofreading.


%библиография по ГОСТу

% Мне библиографию в требуемом формате сгенерировал biblatex:
% https://github.com/odomanov/biblatex-gost/issues/20

\begin{thebibliography}{99}
\bibitem{anning1945integral}
Anning N. H., Erdös P. Integral distances // Bulletin of the American Mathematical
Society. — 1945. — т. 51, No 8. — с. 598—600.
\bibitem{erdos1945integral}
Erdös P. Integral distances // Bulletin of the American Mathematical Society. — 1945. —
т. 51, No 12. — с. 996.
\bibitem{our-vmmsh-2018}
Авдеев Н. Н., Семёнов Е. М. Множества точек с целочисленными расстояниями на
плоскости и в евклидовом пространстве // Математический форум (Итоги науки. Юг
России). — 2018. — с. 217—236.
\bibitem{kurz2008bounds}
Kurz S., Laue R. Bounds for the minimum diameter of integral point sets // Australasian
Journal of Combinatorics. — 2007. — т. 39. — с. 233—240. — arXiv: 0804.1296.
\bibitem{kurz2008minimum}
Kurz S., Wassermann A. On the minimum diameter of plane integral point sets // Ars
Combinatoria. — 2011. — т. 101. — с. 265—287. — arXiv: 0804.1307.
\bibitem{antonov2008maximal}
Antonov A. R., Kurz S. Maximal integral point sets over $\mathbb{Z}^2$ // International Journal of
Computer Mathematics. — 2008. — т. 87, No 12. — с. 2653—2676. — arXiv: 0804.1280.
\bibitem{huff1948diophantine}
Huff G. B. Diophantine problems in geometry and elliptic ternary forms // Duke
Mathematical Journal. — 1948. — т. 15, No 2. — с. 443—453.
\bibitem{harborth1993upper}
Harborth H., Kemnitz A., Möller M. An upper bound for the minimum diameter of integral
point sets // Discrete \& Computational Geometry. — 1993. — т. 9, No 4. — с. 427—432.
\bibitem{piepmeyer1996maximum}
Piepmeyer L. The maximum number of odd integral distances between points in the plane //
Discrete \& Computational Geometry. — 1996. — т. 16, No 1. — с. 113—115.
\bibitem{kreisel2008heptagon}
Kreisel T., Kurz S. There are integral heptagons, no three points on a line, no four on a
circle // Discrete \& Computational Geometry. — 2008. — т. 39, No 4. — с. 786—790.
\bibitem{kurz2013constructing}
Constructing 7-clusters / S. Kurz [и др.] // Serdica Journal of Computing. — 2014. — т. 8,
No 1. — с. 47—70. — arXiv: 1312.2318.
\bibitem{solymosi2003note}
Solymosi J. Note on integral distances // Discrete \& Computational Geometry. — 2003. —
т. 30, No 2. — с. 337—342.
\bibitem{my-pps-linear-bound-2019}
Avdeev N. On existence of integral point sets and their diameter bounds. — 2019. — arXiv:
1906.11926.
\bibitem{bat2018number}
Bat-Ochir G. On the number of points with pairwise integral distances on a circle // Discrete
Applied Mathematics. — 2018. — т. 254. — с. 17—32.
\bibitem{brass2006research}
Brass P., Moser W. O., Pach J. Research problems in discrete geometry. — Springer Science
\& Business Media, 2006.
\bibitem{guy2013unsolved}
Guy R. Unsolved problems in number theory. т. 1. — Springer Science \& Business Media,
2013.
\bibitem{our-pmm-2018}
Авдеев Н. Н. Об отыскании целоудалённых множеств специального вида // Актуальные
проблемы прикладной математики, информатики и механики - сборник трудов Международной
научной конференции. — Научно-исследовательские публикации, 2018. — с. 492—498.
\bibitem{our-ped-2018}
Авдеев Н. Н. On integral point sets in special position // Некоторые вопросы анализа,
алгебры, геометрии и математического образования: материалы международной молодежной
научной школы «Актуальные направления математического анализа и смежные вопросы». —
2018. — т. 8. — с. 5—6.
\bibitem{kurz2005characteristic}
Kurz S. On the characteristic of integral point sets in $E^m$ // Australasian Journal of
Combinatorics. — 2006. — т. 36. — с. 241. — arXiv: math/0511704.
\bibitem{nozaki2013lower}
Nozaki H. Lower bounds for the minimum diameter of integral point sets // Australasian
Journal of Combinatorics. — 2013. — т. 56. — с. 139—143.
\bibitem{smurov1998stripcoverings}
Смуров М., Спивак А. Покрытия полосками // Квант. — 1998. — No 5. — с. 6.
\end{thebibliography}




%библиография по Гарвардскому стандарту

\begin{engbibliography}{99}

\bibitem{ENGanning1945integral}
	Anning, N.H. \& Erdös, P. 1945.
	“Integral distances”.
	\emph{Bulletin of the American Mathematical Society}, vol. 51.8, pp. 598—600.
	doi: 10.1090/S0002-9904-1945-08407-9

\bibitem{ENGerdos1945integral}
	Erdös, P. 1945.
	“Integral distances”.
	\emph{Bulletin of the American Mathematical Society}, vol. 51.12, p. 996.
	doi: 10.1090/S0002-9904-1945-08490-0

\bibitem{ENGour-vmmsh-2018}
	Avdeev, N.N. \& Semenov, E.M. 2018.
	``Mnozhestva tochek c tselochislennymi rasstoyaniyami na ploskosti i v evklidovom prostranstve''
	(``Integral point sets on the plane and in Euclidean space'')
	\emph{Matematicheskiy forum (Itogi nauki. Yug Rossii)}, pp. 217—236.
	% No DOI

\bibitem{ENGkurz2008bounds}
	Kurz, S. \& Laue, R. 2007.
	“Bounds for the minimum diameter of integral point sets”.
	\emph{Australasian Journal of Combinatorics}, vol. 39, pp. 233—240. arXiv: 0804.1296.

\bibitem{ENGkurz2008minimum}
	Kurz, S. \& Wassermann, A. 2011.
	“On the minimum diameter of plane integral point sets”.
	\emph{Ars Combinatoria}, vol. 101, pp. 265—287. arXiv: 0804.1307.

\bibitem{ENGantonov2008maximal}
	Antonov, A.R. \& Kurz, S. 2008.
	“Maximal integral point sets over $Z^2$”.
	\emph{International Journal of Computer Mathematics}, vol. 87.12, pp. 2653—2676. arXiv: 0804.1280.
	doi: 10.1080/00207160902993636

\bibitem{ENGhuff1948diophantine}
	Huff, G.B. 1948.
	“Diophantine problems in geometry and elliptic ternary forms”.
	\emph{Duke Mathematical Journal}, vol. 15.2, pp. 443—453.
	doi:10.1215/S0012-7094-48-01543-9

\bibitem{ENGharborth1993upper}
	Harborth, H., Kemnitz, A. \& Möller, M. 1993.
	“An upper bound for the minimum diameter of integral point sets”.
	\emph{Discrete \& Computational Geometry} vol. 9.4, pp. 427—432.
	doi: 10.1007/bf02189331

\bibitem{ENGpiepmeyer1996maximum}
	Piepmeyer, L. 1996.
	“The maximum number of odd integral distances between points in the plane”.
	\emph{Discrete \& Computational Geometry}, vol. 16.1, pp. 113—115.
	doi: 10.1007/bf02711135

\bibitem{ENGkreisel2008heptagon}
	Kreisel, T. \& Kurz, S. 2008.
	“There are integral heptagons, no three points on a line, no four on a circle”.
	\emph{Discrete \& Computational Geometry}, vol. 39.4, pp. 786—790.
	doi: 10.1007/s00454-007-9038-6

\bibitem{ENGkurz2013constructing}
	Kurz, S., Noll, L.C., Rathbun, R, \& Simmons, C. 2014.
	“Constructing 7-clusters”.
	\emph{Serdica Journal of Computing}, vol. 8.1, pp. 47—70. arXiv: 1312.2318.

\bibitem{ENGsolymosi2003note}
	Solymosi, J. 2003.
	“Note on integral distances”.
	\emph{Discrete \& Computational Geometry}, vol. 30.2, pp. 337—342.
	doi: 10.1007/s00454-003-0014-7

\bibitem{ENGmy-pps-linear-bound-2019}
	Avdeev, N. 2019.
	“On existence of integral point sets and their diameter bounds”.
	arXiv: 1906.11926

\bibitem{ENGbat2018number}
	Bat-Ochir, G. 2018.
	“On the number of points with pairwise integral distances on a circle”.
	\emph{Discrete Applied Mathematics}, vol. 254, pp. 17—32.
	doi: 10.1016/j.dam.2018.07.004

\bibitem{ENGbrass2006research}
	Brass, P., Moser, W.O.J. \& Pach, J. 2006.
	\emph{Research problems in discrete geometry}. Springer Science \& Business Media.
	doi: 10.1007/0-387-29929-7

\bibitem{ENGguy2013unsolved}
	Guy, R. 2013.
	\emph{Unsolved problems in number theory}. Vol. 1.
	Springer Science \& Business Media.
	doi: 10.1007/978-1-4757-1738-9

\bibitem{ENGour-pmm-2018}
	Avdeev, N.N. 2018.
	“Ob otyskanii tseloudalennykh mnozhestv spetsial'nogo vida”
	(``On the search of integral point sets of a special type'').
	\emph{
		Aktual'nye problemy prikladnoj matematiki, informatiki i mekhaniki
		- sbornik trudov Mezhdunarodnoj nauchnoj konferencii.
	}
	(Actual problems of applied mathematics, informatics and mechanics
	- proc. of the int. conf.).
	Voronezh, pp. 492—498.

\bibitem{ENGour-ped-2018}
	Avdeev, N.N. 2018.
	“On integral point sets in special position”.
	\emph{Nekotorye voprosy analiza, algebry, geometrii i matematicheskogo obrazovaniya}
	(Some problems of analysis, algebra, geometry and mathematical education), vol. 8, pp. 5—6.

\bibitem{ENGkurz2005characteristic}
	Kurz, S. 2006.
	“On the characteristic of integral point sets in $E^m$”.
	\emph{Australasian Journal of Combinatorics}, vol. 36, pp. 241–248.
	arXiv: math/0511704.

\bibitem{ENGnozaki2013lower}
	Nozaki, H. 2013.
	“Lower bounds for the minimum diameter of integral point sets”.
	\emph{Australasian Journal of Combinatorics}, vol. 56, pp. 139—143.

\bibitem{ENGsmurov1998stripcoverings}
	Smurov, M. \& Spivak, A. 1998.
	“Pokrytiya poloskami” (``Covering by strips'').
	\emph{Kvant}, vol. 5, pp. 6--12

\end{engbibliography}




\label{end}

\end{document}
