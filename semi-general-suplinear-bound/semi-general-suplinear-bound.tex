\documentclass[a4paper,14pt]{article} %размер бумаги устанавливаем А4, шрифт 12пунктов
\usepackage[T2A]{fontenc}
\usepackage[utf8]{inputenc}
\usepackage[russian,english]{babel} %используем русский и английский языки с переносами
\usepackage{amssymb,amsfonts,amsmath,mathtext,enumerate,float,amsthm} %подключаем нужные пакеты расширений
\usepackage[unicode,colorlinks=true,citecolor=black,linkcolor=black]{hyperref}
%\usepackage[pdftex,unicode,colorlinks=true,linkcolor=blue]{hyperref}
\usepackage{indentfirst} % включить отступ у первого абзаца
\usepackage[dvips]{graphicx} %хотим вставлять рисунки?
\graphicspath{{illustr/}}%путь к рисункам

\makeatletter
\renewcommand{\@biblabel}[1]{#1.} % Заменяем библиографию с квадратных скобок на точку:
\makeatother %Смысл этих трёх строчек мне непонятен, но поверим "Запискам дебианщика"

\usepackage{geometry} % Меняем поля страницы.
\geometry{left=2cm}% левое поле
\geometry{right=1cm}% правое поле
\geometry{top=2cm}% верхнее поле
\geometry{bottom=2cm}% нижнее поле

\renewcommand{\theenumi}{\arabic{enumi}}% Меняем везде перечисления на цифра.цифра
\renewcommand{\labelenumi}{\arabic{enumi}}% Меняем везде перечисления на цифра.цифра
\renewcommand{\theenumii}{.\arabic{enumii}}% Меняем везде перечисления на цифра.цифра
\renewcommand{\labelenumii}{\arabic{enumi}.\arabic{enumii}.}% Меняем везде перечисления на цифра.цифра
\renewcommand{\theenumiii}{.\arabic{enumiii}}% Меняем везде перечисления на цифра.цифра
\renewcommand{\labelenumiii}{\arabic{enumi}.\arabic{enumii}.\arabic{enumiii}.}% Меняем везде перечисления на цифра.цифра

\sloppy




\renewcommand\normalsize{\fontsize{14}{25.2pt}\selectfont}

\usepackage[backend=biber,style=gost-numeric,sorting=none]{biblatex}
\addbibresource{../common/notmy.bib}
\addbibresource{../common/my.bib}


\theoremstyle{plain}
\newtheorem{theorem}{Theorem}[section]
\newtheorem{conjecture}[theorem]{Conjecture}
\newtheorem{lemma}[theorem]{Lemma}
\newtheorem{corollary}[theorem]{Corollary}
\newtheorem{proposition}[theorem]{Proposition}

\theoremstyle{definition}
\newtheorem{definition}[theorem]{Definition}
\newtheorem{construction}[theorem]{Construction}
\newtheorem{remark}[theorem]{Remark}
\newtheorem{problem}[theorem]{Problem}

\begin{document}

%\renewcommand{\bibname}{Список цитированной литературы}
%\renewcommand\refname{\bibname}
% !!!
% The text starts here

\title{
	On diameter bounds for planar integral point sets in semi-general position
	\footnote{
		This work was carried out at Voronezh State University and supported by the Russian Science
		Foundation grant 19-11-00197.
	}
}

\author{
	N.N. Avdeev
	\footnote{nickkolok@mail.ru, avdeev@math.vsu.ru}
}

\maketitle

\paragraph{Abstract.}
A point set $M$ in the Euclidean plane is called a planar integral point set if all the distances between the
elements of $M$ are integers, and $M$ is not situated on a straight line.
A planar integral point set is called to be in semi-general position, if it does not contain collinear triples.
The existing bound for mininum diameter of planar integral point sets is linear.
We prove a new bound for mininum diameter of planar integral point sets in semi-general position
which is greater than linear.



\section{Introduction}

Kurz - facher
Bat-Ochir - circular
Kurz - precise values
Kurz - about many collinear points

Kurz - semi-general and general position, precise values

Piepmeyer(?) - upper bound \cite{harborth1993upper}

Particular cases - Huff, Monographies (2?), my-amm, my-ped


\section{Preliminary results}

In this section, we give some lemmas which will be used for the proof.


\begin{lemma}
	\cite[Observation 1]{solymosi2003note}
	If a triangle $T$ has integer side-lengths $a \leq b \leq c$,
	then the minimal height $m$ of it is at least $\left(a - \frac{1}{4}\right)^{1/2}$.
\end{lemma}

\begin{definition}
	The part of a plane between two parallel straight lines with distance $\rho$ between the lines
	is called a strip of width $\rho$.
\end{definition}

\begin{lemma}
	TODO: cite Kvant

	If a triange $T$ with minimal height $\rho$ is situated in a strip,
	then the width of a strip is at least $\rho$.
\end{lemma}

\begin{lemma}
	\cite[Lemma 4]{our-vmmsh-2018};
	\cite[TODO]{my-pps-linear-bound-2019}
	\label{lem:square_container}
	Let $M\in\mathfrak{M}(2,n)$, $\operatorname{diam} M = d$.
	Then $M$ is situated in a square of side length $d$.
\end{lemma}

\begin{definition}
	\cite[TODO]{my-pps-linear-bound-2019}
	A \textit{cross} for points $M_1$ and $M_2$, denoted by $cr(M_1,M_2)$, is the union of two straight lines:
	the line through $M_1$ and $M_2$,
	and the perpendicular bisector of line segment $M_1 M_2$.
\end{definition}

\begin{lemma}
	\cite[Theorem TODO]{my-pps-linear-bound-2019}
	\label{lem:no_distance_one}
	Each set $M\in\mathfrak{M}(2,n)$
	such that for some $M_1,M_2 \in M$ equality $|M_1 M_2|=1$ holds,
	consists of $n-1$ points, including $M_1$ and $M_2$, on a straight line,
	and one point out of the line, on the perpendicular bisector of line segment $M_1 M_2$.
\end{lemma}


\begin{lemma}
	\label{lem:count_of_points_on_hyperbolas}
	Let $\{M_1, M_2, M_3, M_4\} \subset M\in\overline\mathfrak{M}(2,n)$
	(points $M_2$ and $M_3$ may coincide, other points may not), $n\geq 4$.
	Then $\# M \leq 4 \cdot |M_1 M_2| \cdot |M_3 M_4|$.
\end{lemma}

\begin{remark}
	Lemma~\ref{lem:count_of_points_on_hyperbolas} is one of the variations of~\cite{erdos1945integral},
	\cite[TODO]{my-pps-linear-bound-2019}, etc.
\end{remark}

\begin{proof}
	Each point $N\in M$ satisfies one of the following conditions:

	a) $N$ belongs to $cr(M_1,M_2)$~--- overall at most 4 points;

	b) $N$ belongs to $cr(M_3,M_4)$~--- overall at most 4 points;

	c) $N$ belongs to the intersection of one of $|M_1 M_2| - 1$ hyperbolas
	with one of $|M_3 M_4| - 1$ hyperbolas~--- overall at most $4 (|M_1 M_2| - 1)(|M_3 M_4| - 1)$ points;

	Due to Lemma~\ref{lem:no_distance_one} we have $|M_1 M_2| \geq 2$ and $|M_3 M_4| \geq 2$.
	Since
	\begin{multline}
		4 (|M_1 M_2| - 1)(|M_3 M_4| - 1) + 4 + 4
		=
		4 ( (|M_1 M_2| - 1)(|M_3 M_4| - 1) + 2)
		=
		4 ( |M_1 M_2| \cdot |M_3 M_4| + 1 - |M_1 M_2| - |M_3 M_4| + 2)
		=
		4 ( |M_1 M_2| \cdot |M_3 M_4| + 1 - |M_1 M_2| - |M_3 M_4| + 2)
		=
		4 ( |M_1 M_2| \cdot |M_3 M_4| + (1 - |M_1 M_2|) + (2 - |M_3 M_4|))
		\leq
		4 |M_1 M_2| \cdot |M_3 M_4|
		,
	\end{multline}
	we are done.
\end{proof}


\section{The main result}

\begin{theorem}
	For every $n\in\mathbb{N}$ we have
	\begin{equation}
		\overline{d}(2,n) \geq c n^{5/4}
		,
	\end{equation}
	where $c = $ TODO!
\end{theorem}

\begin{proof}
	For $n = 3$ we have $\overline{d}(2,n) = 1$.
	Consider $M\in\overline\mathfrak{M}(2,n)$, $n \geq 4$, $\operatorname{diam} M = p$.

	Let us choose points $M_1, M_2, M_3, M_4 \in M$
	(points $M_2$ and $M_3$ may coincide, other points may not), such that
	\begin{equation}
		\min_{A, B \in M} |AB| = |M_1 M_2|
		,
	\end{equation}
	\begin{equation}
		\min_{A, B \in M \setminus \{M_1\}} |AB| = |M_3 M_4| = m
		.
	\end{equation}

	For $m \leq p^{2/5}$, Lemma~\ref{lem:count_of_points_on_hyperbolas} yields that
	\begin{equation}
		n \leq 4 \cdot |M_1 M_2| \cdot |M_3 M_4| \leq  4 p^{4/5}
		,
	\end{equation}
	or, that is the same,
	\begin{equation}
		p \geq (n/4) ^ {5/4}
		.
	\end{equation}

	So, let us consider $m > p^{2/5}$.
	Due to Lemma~\ref{}

\end{proof}

\section{Conclusion}
The presented bound is the first special lower bound for sets in semi-general position.
Thus, we did not accepted the challenge to make the constant in Theorem~\ref{thm:main_result} as large as possible,
in order to keep the ideas of the proof clear and understandable.
A more thorough research can be done in the future to enlarge the constant.
However, the upper and lower bounds are still not tight.

\section{Acknowledgements}
Author thanks Dr. Prof. E.M. Semenov for proofreading and valuable advice.


\printbibliography

\end{document}
