\documentclass[a4paper,14pt]{article} %размер бумаги устанавливаем А4, шрифт 12пунктов
\usepackage[T2A]{fontenc}
\usepackage[utf8]{inputenc}
\usepackage[english,russian]{babel} %используем русский и английский языки с переносами
\usepackage{amssymb,amsfonts,amsmath,mathtext,cite,enumerate,float,amsthm} %подключаем нужные пакеты расширений
\usepackage[pdftex,unicode,colorlinks=true,linkcolor=blue]{hyperref}
\usepackage{indentfirst} % включить отступ у первого абзаца
\usepackage[dvips]{graphicx} %хотим вставлять рисунки?
\graphicspath{{illustr/}}%путь к рисункам

\makeatletter
\renewcommand{\@biblabel}[1]{#1.} % Заменяем библиографию с квадратных скобок на точку:
\makeatother %Смысл этих трёх строчек мне непонятен, но поверим "Запискам дебианщика"

\usepackage{geometry} % Меняем поля страницы. 
\geometry{left=1cm}% левое поле
\geometry{right=1cm}% правое поле
\geometry{top=1cm}% верхнее поле
\geometry{bottom=2cm}% нижнее поле

\renewcommand{\theenumi}{\arabic{enumi}}% Меняем везде перечисления на цифра.цифра
\renewcommand{\labelenumi}{\arabic{enumi}}% Меняем везде перечисления на цифра.цифра
\renewcommand{\theenumii}{.\arabic{enumii}}% Меняем везде перечисления на цифра.цифра
\renewcommand{\labelenumii}{\arabic{enumi}.\arabic{enumii}.}% Меняем везде перечисления на цифра.цифра
\renewcommand{\theenumiii}{.\arabic{enumiii}}% Меняем везде перечисления на цифра.цифра
\renewcommand{\labelenumiii}{\arabic{enumi}.\arabic{enumii}.\arabic{enumiii}.}% Меняем везде перечисления на цифра.цифра

\begin{document}\large


УДК ???

О множествах точек на плоскости с целочисленными расстояниями

Н.Н. Авдеев, Е.М. Семёнов.


Известна следующая

\paragraph{Теорема 1.} Пусть $\{M_1,M_2\}$ --- счётное множество точек на плоскости и расстояние $|M_i,M_j|\in \mathbb{N}$ для всех $1\leq i < j <\infty$, где $\mathbb{N} - $множество натуральных чисел. Тогда найдется такая прямая на плоскости $l$, что $M_i\in l$ для всех $i \in \mathbb{N}$.

Формулировка теоремы и идея её доказательства приведены в \cite{Newman}. Полное доказательство можно найти в \cite{angem1kurs}. Там же показан, что для любого $N\in \mathbb{N}$ существует такое множество $\{M_1,M_2,...,M_n\}\subset \mathbb{R}^2$, что $|M_i,M_j|$ для всех $1\leq i < j  \leq n$ и $M_1, M_2, ..., M_n$ не лежат на прямой. Изучению таких подмножеств посвящена настоящая работа.


Для заданного $n\in \mathbb{N}$ обозначим через $C_n$ множество таких последовательностей $M_1,M_2,...,M_n \in \mathbb{R}^2$, что $|M_i,M_j|\in\mathbb{N}$ ля всех $1\leq i < j  \leq n$ и  $M_1,M_2,...,M_n$ не принадлежат никакой прямой. Положим
$$
F(n)=\min\limits_{A\in C_n} d(A),
$$
где $d(A)$ --- диаметр $A$, т. е.
$$
d(A)=\max\limits_{x,y\in A}|x,y|
$$

Точную асимптотику последовательности  $F(n)$ найти не удалось, получены лишь верхняя и нижняя оценки.

Число элементов множества $A$ обозначим через $|A|$.
В \cite{angem1kurs} была доказана

\paragraph{Лемма 2.}

Пусть $A=(M_1, M_2, ..., M_n) \in C_n$ для некоторого $n\in N$ и $M_1, M_2, M_3$ не принадлежат прямой.
Тогда $n\leq (a+1)(b+1)+3$, где $a=|M_1,M_2|$, $|M_2,M_3|$.

Аналогичное утверждение справедливо, когда $M_1,M_2,M_3$ принадлежат некоторой прямой и $M_2$ лежит между $M_1$ и $M_3$.
В этом случае
$n\leq (a+1)(b+1)+3+d(A)$.



(Евгений Михайлович! Здесь вылезло $+d(A)$ в правой части.
В следующей лемме из-за этого может испортиться константа перед $m^2$.
Чтобы её сберечь, повысил $m$ в условии с 2 до 4.
Надо всё пересчитывать $=($ ~~ )

\paragraph{Лемма 3.}
Пусть $m\in\mathbb{N}, m\geq 4$, последовательность $(M_1,M_2,...,M_{2m^2+1})$ принадлежит $C_{2m^2+1}$ и содержится в квадрате со стороной $d$.
Тогда $d>\frac{1}{2}m^2$.

\paragraph{Доказательство.} Разобьём квадрат со стороной $d$ на $m^2$ квадратов со стороной $\frac{d}{m}$.
Тогда по крайней мере один из маленьких квадратов содержит некоторые три точки исходной последовательности.
Без ограничения общности $M_1,M_2,M_3$ содержатся в квадрате со стороной $\frac{d}{m}$.
Поэтому $|M_1,M_2|,|M_2,M_3| \leq \frac{d}{m}\sqrt{2}$.
В силу леммы 2
$$
2m^2+1 \leq\left(\frac{d}{m}\sqrt{2}+1\right)^2+3+d\sqrt{2}
$$

Положим $d=\lambda m^2$.
Тогда
$$
2m^2+1 \leq \left(\lambda m \sqrt2 + 1 \right)^2 + 3 + \lambda m^2 \sqrt2
$$
и
$$
0\leq \left( 2\lambda^2 + \sqrt2 \lambda - 2 \right)m^2 + 2\sqrt2 \lambda m + 3
$$

Для $m\geq 4$ это неравенство не выполнено, если $\lambda \leq \frac{1}{2}$.
Поэтому $\lambda > \frac{1}{2}$ и $d>\frac{m^2}{2}$.

Лемма доказана.

(Здесь вставляю кусок с примориальной конструкцией)


Обозначим через $p_i, i\in\mathbb{N}$ простые числа, начиная с 3.
По теореме Чебышева (\cite{Buhshtab}) $p_i \leq bi\ln(i+1)$ для некоторого $b>0$ и всех $i\in\mathbb{N}$.
Обозначим
$$
A_n=\prod_{i=1}^{n} p_i, n\in\mathbb{N}
$$

Тогда
$$
A_n\leq b^n n! \prod_{i=1}^{n} \ln(i+1)
$$

и по формуле Стирлинга
\begin{equation*}\label{ocenka_Stirling}
A_n <
b^n \sqrt{2\pi n} \left(\frac{n}{e}\right)^n \left(1+\frac{1}{n}\right) \prod_{i=1}^{n} \ln(i+1)
\leq \left(\frac{bn\ln (n+1)}{e}\right)^n
\end{equation*}


\paragraph{Теорема 4.}
$F(2^n)<A_n$ для всех $n\in\mathbb{N}$.

\paragraph{Доказательство.} Через $S$ обозначим множество подмножеств $\{1,2,..,n\}$ и каждому $I\in S$ поставим в соответствие числа $c_I=\prod\limits_{c\in I}p_i, b_I=\frac{1}{2}\left(c_I-\frac{A_n}{c_I}\right)$.
Так как $c_I$ и $\frac{A_n}{c_I}$ нечётны, то $b_I$ --- целые числа.

Рассмотрим подмножество точек на плоскости
\begin{equation}\label{konstrukcia_primorialy}
M_I=\{(b_I,0), I\in S\}, N=(0, \sqrt{A_n})
\end{equation}


Если $I,J \in S, I \neq J$, то $M_I \neq M_J$.
Поэтому множество (\ref{konstrukcia_primorialy}) содержит $2^n+2$ элементов.
Так как
\begin{multline*}
	|M_I,N|=\left(\frac{1}{4}\left( c_I^2 - 2 A_n + \frac{A_n^2}{c_I^2} \right)+A_n\right)^\frac{1}{2}=\\
	\left(\frac{1}{4}\left( c_I^2 - 2 A_n + \frac{A_n^2}{c_I^2} \right)+A_n\right)^\frac{1}{2}=
	\frac{c_I+\frac{A_n}{C_I}}{2} \in \mathbb{N},
\end{multline*}

то все расстояния между точками множества (\ref{konstrukcia_primorialy}) есть целые числа.
Диаметр множества (\ref{konstrukcia_primorialy}) достигается на паре точек $M_{(1,2,...,n)}$ и $M_{\varnothing}$, для которых
$$
|M_{(1,2,...,n)},M_{\varnothing}|=\frac{1}{2}(A_n-1)-\frac{1}{2}(1-A_n) = A_n-1 < A_n
$$

Поэтому 
$$
F(2^n+2) < A_n
$$

Отсюда, из \ref{ocenka_Stirling} и монотонности $F(n)$ вытекает, что
$$
F(n) \leq \left( \frac{b \log_2(n+1) \ln \log_2 (n+1)}{e}\right)^{\log_2(n+1)}
$$
для всех $n\in \mathbb{N}$.


Заметим, что подобная конструкция не позволяет получить оценку сверху в качестве полинома.
Это вызвано тем, что количество точек в множестве подобного типа (подмножество прямой и точка, на ней не лежащая) ограничено числом  делителей числа $2\mu^2+2$ (эту оценку обосновать?), где $\mu$ --- расстояние от точки до прямой.
С другой стороны, в \cite{Ramanujan} (формулы 198-200) показано, что
$$
p(n)=\max\limits_{1\leq k \leq n} d(k),
$$
где $d(k)$ --- количество делителей числа $k$ (дивизор-функция Рамануджана?),
растёт медленнее любого полинома.



(Осталось свести оба ограничения в кучку)

В ходе численного расчёта были вычислены значения $F(n)$ для $4\leq n \leq 41$.
Время, требуемое для вычисления $F(n)$ разработанными алгоритмами, растёт, как установлено эмпирически, не медленнее, чем $n^4$.
Так, расчёт для $n=5$ занимает меньше секунды, а для вычисления $F(41)$ при известном (вычисленном ранее) $F(40)$ ушло больше суток.

\addcontentsline{toc}{chapter}{Литература}
\begin{thebibliography}{99}

\bibitem{Newman} Donald J. Newman. A Problem Seminar. Springer - Verlag.1982. Problem 29.

\bibitem{angem1kurs} Аналитическая геометрия на плоскости / Е.М. Семенов, С.Н. Уксусов. – Воронеж : Воронежский государственный университет, 2016. – 100с.

\bibitem{Ramanujan} Highly composite numbers. Proceedings of the London Mathematical Society, 2, XIV, 1915, 347 – 409 (http://ramanujan.sirinudi.org/Volumes/published/ram15.pdf)

\bibitem{Buhshtab} Теория чисел / А.А. Бухштаб. --- М., 1966 (теорема 325)


\end{thebibliography}

\end{document}
