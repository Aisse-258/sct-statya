\documentclass[12pt]{article}
\usepackage[utf8]{inputenc}
\usepackage[T2A]{fontenc}
\usepackage[russian,english]{babel}
\usepackage{amsmath,amsfonts,amssymb,euscript,graphicx,wrapfig,multirow}
\usepackage{dsfont}
\usepackage{amsthm}
\inputencoding{utf8}
%\bibliographystyle{unsrt}
\textheight=240mm \textwidth=170mm
\hoffset=-17mm
\voffset=-17mm


\usepackage[backend=biber,style=gost-numeric,sorting=none]{biblatex}
\addbibresource{../common/notmy.bib}
\addbibresource{../common/my.bib}


\usepackage{hyperref}

\makeatletter
%\renewcommand{\fnum@figure}{Figure \thefigure}
\renewcommand{\@biblabel}[1]{#1.}
\makeatother

\theoremstyle{theorem}
\newtheorem{theorem}{Theorem}
\theoremstyle{dfn}
\newtheorem{dfn}{Definition}
\theoremstyle{remark}
\newtheorem{remark}{Remark}

\begin{document}
%\renewcommand\refname{\centering References}


% Переключаем язык на английский.
% Очень полезно как в плане типографики (в том числе расстановки переносов),
% так и в плане того, что не надо переименовывать "Рисунок" в "Figure"
\selectlanguage{english}




\title{
	On particular examples of planar integral point sets and their classification
	\footnote{
		This work was carried out at Voronezh State University and supported by the Russian Science
		Foundation grant 19-11-00197.
	}
}

%% Прекрасно понимаю, что следующая команда - дичь и вордописчество, но время не ждёт, время жмёт
\author{
	Avdeev N.N.
	\footnote{nickkolok@mail.ru, avdeev@math.vsu.ru}
	, Momot E.A., Zvolinsky A.E.
	\\
	\\
	\emph{Voronezh State University}
}

\maketitle


\section{Introduction}



\begin{dfn}\label{dfn1}
	A planar integral point set (PIPS) is a set $\mathcal{P}$
	of non-collinear points in plane $\mathbb{R}^{2}$ such that
	for any pair of points $P_{1}, P_{2} \in \mathcal{P}$
	the Euclidean distance $|P_{1}P_{2}|$
	between points $P_{1}$ and $P_{2}$ is integral.
\end{dfn}

How do we characterize an integral point set?
For example, we can count the number of points in it, which is always finite~\cite{anning1945integral,erdos1945integral}
and is further called the cardinality;
alternatively, we can naturally% TODO: naturally/essentially? cf. arxiv/1
define the diameter of a finite point set.

\begin{dfn}
	A diameter of the integral point set $\mathcal{P}$ is defined as
	\begin{equation}
		\operatorname{diam(\mathcal{P})} = \underset{P_{1}, P_{2} \in
		\mathcal{P}}{\max} |P_{1}P_{2}|
	\end{equation}
\end{dfn}

Every integral point set also has a characteristic~\cite{kemnitz1988punktmengen,kurz2005characteristic}.









\begin{dfn}
	The function $d(m, n)$ is the minimum possible diameter of
	the integral point set $\mathcal{P}$ of $n$ points in
	$m$-dimensional Euclidean space $\mathbb{R}^{m}$.
\end{dfn}


%For a bit more sophisticated configurations, another functions are presented~\cite{kurz2008minimum,kreisel2008heptagon}.
%
%\begin{dfn}
%	The function $\overline{d}(m, n)$ is the minimum possible diameter of
%	the integral point set $\mathcal{P}$ of $n$ points in
%	$m$-dimensional Euclidean space $\mathbb{R}^{m}$
%	such that no $m+1$ points are located on an $(m-1)$-dimensional hyperplane.
%\end{dfn}

%\begin{dfn}
%	The function $\dot{d}(m, n)$ is the minimum possible diameter of
%	the integral point set $\mathcal{P}$ of $n$ points in
%	$m$-dimensional Euclidean space $\mathbb{R}^{m}$
%	such that no three points are located on a straight line
%	and no four points are located on a circle.
%\end{dfn}

For the list of known bounds for $d(m, n)$,
we refer the reader to~\cite[Theorem 1]{kurz2008bounds} or to~\cite{our-vmmsh-2018}.
We will discuss the following estimations presented at~\cite{kurz2008bounds}:
\begin{align}
	d(m, 2m + 1) \leq 8\\
	d(m, 2m + 2) \leq 13 \hypertarget{d2}\\
	d(m, 3m) \leq 109
\end{align}
and the following theorem \cite[Theorem 2.1]{kurz2008bounds}.

\begin{theorem}
	Let $\mathcal{P}$ be a planar integral point set consisting of
	$n - 2$ points on line $l_1$ and two points $P_{1}$ and $P_{2}$ on a
	parallel line $l_2$ with distance $r$ between $l_{1}$ and $l_{2}$. If there
	exist positive integers $v$, $w$ with $f^{2} + v^{2}
	= w^{2}$ and $v < 2r$, where $|P_{1}P_{2}| = f$,
	then

	\begin{equation}\label{formula1}
		d(m, n + 2(m - 2)) \leq \max(w, \operatorname{diam(\mathcal{P})})
	\end{equation}

\end{theorem}

Firstly, we discuss the classification of planar integral points sets;
then, we present some bounds for $d(m,n)$ based on planar integral point sets of particular types
and provide some general constructions for such bounds.

\section{Classification of planar integral point sets}

\subsection{Integral point sets situated in two straight lines}

\begin{dfn}
	A planar integral point sets of $n$ points with $n-1$ points on a straight line is called
	a \textit{facher} set.
\end{dfn}
Facher sets are very dominating examples of planar integral pont sets.
In~\cite{antonov2008maximal}, facher sets of characteristic 1 are called \textit{semi-crabs}.
For $9 \leq n \leq 122$, the diameter $d(2,n)$ is reached on a facher point set~\cite{kurz2008minimum}.

For non-facher integral point sets situated in two straight lines,
we can easily distinct the following three cases:

\begin{dfn}
	A planar integral point sets situated in two parallel straight lines
	is called a \textit{rails} set.
\end{dfn}

Among the rails sets, sets with 2 points on one line and all the other on another line dominate.

\begin{dfn}
	A planar integral point sets situated in two perpendicular straight lines
	is called a \textit{cross} set.
\end{dfn}
Every cross set has characteristic 1;
in~\cite{antonov2008maximal}, cross sets with only 2 points out of one of the lines are called \textit{crabs}.

\begin{dfn}
	A planar integral point sets situated in two straight lines
	that are not parallel nor perpendicular,
	is called a \textit{sciccors} set.
\end{dfn}

There is an important subclassof scissors sets.

\begin{dfn}
	A scissors set with an axis of symmetry,
	which is the angle bisector for the straight lines,
	is called a \textit{pyramid} set.
\end{dfn}

%TODO: rephrase?

%TODO: can a scissors set have another axis of symmetry?

\subsection{Other integral point sets}

\begin{dfn}
	A planar integral point sets that is situated on a circle is called a \textit{circular}
	point set.
\end{dfn}

Circular sets are very important examples
of integral point sets~\cite{harborth1993upper,piepmeyer1996maximum,bat2018number}.


\begin{dfn}
	A planar integral point sets that is situated on the conjunction of a circle with its center,
	is called a \textit{centered-circular} point set.
\end{dfn}

These six classes dominate among all the known planar integral point sets;
however, some sophisticated constructions are also known.

%TODO: pictures!

\section{Rails sets}




\begin{figure}[h!]
%\center{\includegraphics[width=1\linewidth]{picture_12.pdf}}
\parbox{1\linewidth}{\caption{IPS of cardinality 10 and diameter 56}
\label{picture_12.pdf}}
\end{figure}

\begin{figure}[h!]
%\center{\includegraphics[width=1\linewidth]{picture_1260_R3.pdf}}
\parbox{1\linewidth}{\caption{IPS of cardinality 13 and diameter 1260}
\label{picture_1260_R3.pdf}}
\end{figure}

\begin{itemize}
\setlength{\itemsep}{-1mm}


\item
$\mathcal{P}=\sqrt{3}/{2} * \{ (\pm 56, 0),
(14, 0),
(-34, 0),
(-10, 24),
(-21 , 35),
(35, -21)\}
$
(Figure~\ref{picture_56.png}),

where $n = 7$, $\operatorname{diam(\mathcal{P})} = 56$. Using the ``blowing up''
construction, we obtain
\begin{equation}\label{result2}
d(m, 3m + 1) \leq 56
\end{equation}
The resulting estimate improves the estimate for $d(m, 3m)$, which is presented
in \cite{kemnitz1988punktmengen}. Figure~\ref{picture_12.pdf} shows an example
for $m = 3$.


\item
$\mathcal{P}=\sqrt{39}/{8} * \{ (\pm 5040, 0),
(1911, 315),
(2352, 0),
(944, 0),
(336, 0),
(2940, -420),
(2044, 220),
$

$
(735, 1155)\}
$
(Figure~\ref{picture_1260.png}),

where $n = 9$, $\operatorname{diam(\mathcal{P})} = 1260$. Using the ``blowing up''
construction, we obtain
\begin{equation}\label{result3}
d(m, 4m + 1) \leq 1260
\end{equation}
Figure~\ref{picture_1260_R3.pdf} shows an example for $m = 3$.

\end{itemize}

\section{Final remarks}
All the given PIPSs were obtained through a combination of computer search an intuition of the authors;

There is still no general construction for a rails or scissors PIPS of given cardinality.
For rails PIPSs, we can conjecture that there exists a set of arbitrary cardinality with 2 points on one line
and all the rest on the other;
on the other hand, we still have not found any rails PIPSs with 4 and 5 points on the lines.


The source code can be obtained at https://gitlab.com/Nickkolok/ips-algo

\printbibliography
%\bibliography{literature}

\end{document}
